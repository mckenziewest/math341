\documentclass[t]{beamer}

\subtitle{Representations of Integers and Fast Exponentiation}

\usepackage{amsthm,amsmath,amsfonts,hyperref,graphicx,color,multicol,soul}
\usepackage{enumitem,tikz,tikz-cd,setspace,mathtools}

%%%%%%%%%%
%Beamer Template Customization
%%%%%%%%%%
\setbeamertemplate{navigation symbols}{}
\setbeamertemplate{theorems}[ams style]
\setbeamertemplate{blocks}[rounded]

\definecolor{Blu}{RGB}{43,62,133} % UWEC Blue
\setbeamercolor{structure}{fg=Blu} % Titles

%Unnumbered footnotes:
\newcommand{\blfootnote}[1]{%
	\begingroup
	\renewcommand\thefootnote{}\footnote{#1}%
	\addtocounter{footnote}{-1}%
	\endgroup
}

%%%%%%%%%%
%TikZ Stuff
%%%%%%%%%%
\usetikzlibrary{arrows}
\usetikzlibrary{shapes.geometric}
\tikzset{
	smaller/.style={
		draw,
		regular polygon,
		regular polygon sides=3,
		fill=white,
		node distance=2cm,
		minimum height=1in,
		line width = 2pt
	}
}
\tikzset{
	smsquare/.style={
		draw,
		regular polygon,
		regular polygon sides=4,
		fill=white,
		node distance=2cm,
		minimum height=1in,
		line width = 2pt
	}
}


%%%%%%%%%%
%Custom Commands
%%%%%%%%%%

\newcommand{\C}{\mathbb{C}}
\newcommand{\quats}{\mathbb{H}}
\newcommand{\N}{\mathbb{N}}
\newcommand{\Q}{\mathbb{Q}}
\newcommand{\R}{\mathbb{R}}
\newcommand{\Z}{\mathbb{Z}}

\newcommand{\ds}{\displaystyle}

\newcommand{\fn}{\insertframenumber}

\newcommand{\id}{\operatorname{id}}
\newcommand{\im}{\operatorname{im}}
\newcommand{\lcm}{\operatorname{lcm}}
\newcommand{\Aut}{\operatorname{Aut}}
\newcommand{\Inn}{\operatorname{Inn}}

\newcommand{\blank}[1]{\underline{\hspace*{#1}}}

\newcommand{\abar}{\overline{a}}
\newcommand{\bbar}{\overline{b}}
\newcommand{\cbar}{\overline{c}}

\newcommand{\nml}{\unlhd}

%%%%%%%%%%
%Custom Theorem Environments
%%%%%%%%%%
\theoremstyle{definition}
\newtheorem{exercise}{Exercise}
\newtheorem{question}[exercise]{Question}
\newtheorem{warmup}{Warm-Up}
\newtheorem*{exa}{Example}
\newtheorem*{disc}{Group Discussion}
\newtheorem*{recall}{Recall}
\renewcommand{\emph}[1]{{\color{blue}\texttt{#1}}}

\definecolor{Gold}{RGB}{237, 172, 26}
%Statement block
\newenvironment{statementblock}[1]{%
	\setbeamercolor{block body}{bg=Gold!20}
	\setbeamercolor{block title}{bg=Gold}
	\begin{block}{\textbf{#1.}}}{\end{block}}
\newenvironment{goldblock}{%
	\setbeamercolor{block body}{bg=Gold!20}
	\setbeamercolor{block title}{bg=Gold}
	\setbeamertemplate{blocks}[shadow=true]
	\begin{block}{}}{\end{block}}
\newenvironment{defn}{%
	\setbeamercolor{block body}{bg=gray!20}
	\setbeamercolor{block title}{bg=violet, fg=white}
	\setbeamertemplate{blocks}[shadow=true]
	\begin{block}{\textbf{Definition.}}}{\end{block}}
\newenvironment{nb}{%
	\setbeamercolor{block body}{bg=gray!20}
	\setbeamercolor{block title}{bg=teal, fg=white}
	\setbeamertemplate{blocks}[shadow=true]
	\begin{block}{\textbf{Note.}}}{\end{block}}
\newenvironment{blockexample}{%
	\setbeamercolor{block body}{bg=gray!20}
	\setbeamercolor{block title}{bg=Blu, fg=white}
	\setbeamertemplate{blocks}[shadow=true]
	\begin{block}{\textbf{Example.}}}{\end{block}}
\newenvironment{blocknonexample}{%
	\setbeamercolor{block body}{bg=gray!20}
	\setbeamercolor{block title}{bg=purple, fg=white}
	\setbeamertemplate{blocks}[shadow=true]
	\begin{block}{\textbf{Non-Example.}}}{\end{block}}
\newenvironment{thm}[1]{%
	\setbeamercolor{block body}{bg=Gold!20}
	\setbeamercolor{block title}{bg=Gold}
	\begin{block}{\textbf{Theorem #1.}}}{\end{block}}


%%%%%%%%%%
%Custom Environment Wrappers
%%%%%%%%%%
\newcommand{\exer}[1]{
	\begin{exercise}
		#1
	\end{exercise}
}
\newcommand{\exam}[1]{
\begin{blockexample}
	#1
\end{blockexample}
}
\newcommand{\nexam}[1]{
\begin{blocknonexample}
	#1
\end{blocknonexample}
}
\newcommand{\enumarabic}[1]{
	\begin{enumerate}[label=\textbf{\arabic*.}]
		#1
	\end{enumerate}
}
\newcommand{\enumalph}[1]{
	\begin{enumerate}[label=(\alph*)]
		#1
	\end{enumerate}
}
\newcommand{\bulletize}[1]{
	\begin{itemize}[label=$\bullet$]
		#1
	\end{itemize}
}
\newcommand{\circtize}[1]{
	\begin{itemize}[label=$\circ$]
		#1
	\end{itemize}
}
\newcommand{\slide}[1]{
	\begin{frame}{\fn}
		#1
	\end{frame}
}
\newcommand{\slidec}[1]{
\begin{frame}[c]{\fn}
	#1
\end{frame}
}
\newcommand{\slidet}[2]{
	\begin{frame}{\fn\ - #1}
		#2
	\end{frame}
}


\newcommand{\startdoc}{
		\title{Math 341: Classical Number Theory}
		\author{Mckenzie West}
		\date{Last Updated: \today}
		\begin{frame}
			\maketitle
		\end{frame}
}

\newcommand{\topics}[2]{
	\begin{frame}[c]{\insertframenumber}
		\begin{block}{\textbf{Last Section.}}
			\begin{itemize}[label=--]
				#1
			\end{itemize}
		\end{block}
		\begin{block}{\textbf{This Section.}}
			\begin{itemize}[label=--]
				#2
			\end{itemize}
		\end{block}
	\end{frame}
}

\begin{document} 
	\startdoc
	
	\topics{
		% Last Time
		\item Properties of Congruence
	}
	{
		% This time
		\item Zero Divisors
		\item Representations of Integers beyond decimal
		\item Fast exponentiation algorithm.
	}

\slide{
	\begin{defn}
		Let $n$ be a positive integer. We will say the integers $a$ and $b$ are \emph{congruent modulo $n$} if $n$ divides $a-b$.  We denote this by
			\[a\equiv b\pmod n.\]
	\end{defn}
	\begin{statementblock}{Theorem 4.1}
		For arbitrary integers $a$ and $b$, $a\equiv b\pmod n$ if and only if $a$ and $b$ leave the same remainder when divided by $n$.
	\end{statementblock}
}

\slide{
	\begin{defn}
		Let $n$ be a positive integer.
		We will call an integer $a$ a \emph{zero divisor modulo $n$} if there is some integer $b$ such that
			\[a\not\equiv 0\pmod n, \quad b\not\equiv 0\pmod n,\quad\text{and}\quad ab\equiv 0\pmod n.\]
	\end{defn}
	\begin{exa}
		If $n=6$, then $2,3\not\equiv 0\pmod 6$ but $2\cdot3\equiv 0\pmod 6$.
	\end{exa}
	\begin{exercise}
		What are the zero divisors modulo $12$ between 0 and 12? What are some zero divisors outside this range?
	\end{exercise}
}

\slide{
	\begin{nb}
		Similar to our decimal system where every digit represents how many of a given power of 10 goes into a number, there are other numerical systems.  For example, in the binary system:
		
		\[(100111)_2= 1\cdot 2^5+0\cdot 2^4+0\cdot 2^3+1\cdot 2^2+1\cdot 2^1+1\cdot 2^0=39.\]
		
		If $b$ is the base, in this example $b=2$, then the digits $a_i$ must satisfy $0\leq a_i <b$. This should match up with what we know about decimal aka base 10 integers.
	\end{nb}
}
\slide{	
	\begin{exercise}
		What decimal integer is represented by
		\enumalph{
			\item $(11011)_2$\vskip .5in
			\item $(1000)_2$\vskip .5in
			\item $(1212)_3$\vskip .5in
		}
	\end{exercise}
}
\slide{
	\begin{defn}
		If $b\geq 2$ is an integer and $N\geq 0$ is an integer, the \emph{base $b$} representation of $N$ is
			\[(a_ra_{r-1}\dots a_2a_1a_0)_b,\]
		where $0\leq a_i< b$ for all $0\leq i< r$ and $0< a_r<b$ and in decimal form,
			\[N=a_rb^r+a_{r-1}b^{r-1}+\cdots+a_2 b^2+a_1 b+a_0.\]
	\end{defn}
	\begin{statementblock}{Theorem}
		Let $b\geq 2$ be an integer. Every non-negative integer $N$ has a \textit{unique} representation with respect to the basis $b$. 
	\end{statementblock}
}
\slide{
	\begin{exa}
		Consider $b=4$ and $N=328$.  We see that $4^4$ is the largest power of 4 that is less than or equal to 328, thus $328=(a_4a_3a_2a_1a_0)_4$. Now to find the digits:
			$$\begin{array}{rcl}
				{\color{red}328} &=& \textbf{1}\cdot 4^4 + {\color{orange}72}\\
				{\color{orange}72} &=& \textbf{1}\cdot 4^3 + {\color{yellow}8}\\
				{\color{yellow}8}&=& \textbf{0}\cdot 4^2 + {\color{green} 8}\\
				{\color{green} 8}&=& \textbf{2}\cdot 4^1 + {\color{blue}0}\\
				{\color{blue}0}&=& \textbf{0}\cdot 4^0
			\end{array}$$
		Therefore,
			\[344 = (\textbf{11020})_4.\]
	\end{exa}
}
\slide{
	\begin{exercise}
		Determine the base $b$ representation of $N$,
		\enumalph{
			\item $N=110$, $b=2$\vskip .5in
			\item $N=282$, $b=3$\vskip .5in
			\item $N=345$, $b=5$\vskip .5in
			\item $N=498$, $b=8$\vskip .5in
		}
	\end{exercise}
}
\slide{
	\begin{exa}
		What's the point? Well hopefully it helps us compute integers modulo $n$ more quickly.  For example, on the last slide you found
			\[110=64+32+8+4+2=(1101110)_2.\]
		Thus if we wanted to know
			\[5^{110}\pmod n,\]
		then we could instead compute
			\[5^{64}\cdot5^{32}\cdot 5^8\cdot 5^4\cdot 5^2\pmod n.\]
			
		(Continued on next slide.)
		\end{exa}
}
\slide{ \begin{exa}
		Though in actuality each individual one of these is going to be easy to compute because
			\[5^2\equiv 25\pmod n\]
			\[5^4\equiv (5^2)^2\pmod n\]
			\[5^8\equiv (5^4)^2\pmod n\]
			\[5^{16}\equiv (5^{8})^2\pmod n\]
			\[5^{32}\equiv (5^{16})^2\pmod n\]
			\[5^{64}\equiv (5^{32})^2\pmod n\]
		and squaring is way easier and quicker than raising to higher powers.
	\end{exa}	
}
\slide{
	\begin{exercise}
		Use the method from the example to compute $5^{110}\pmod{16}$.
	\end{exercise}
}
\slide{
	\begin{exercise}
		Compute
		\enumalph{
			\item $7^{47}\pmod{412}$\vskip 1in
			\item $23^{36}\pmod{1001}$
		}
	\end{exercise}
}

\slide{
	\begin{nb}
		The following is true for all $n>0$:
		$$x^n=\begin{cases}
			(x^2)^{n/2}&\text{if }n\text{ even}\\
			x(x^2)^{(n-1)/2}&\text{if }n\text{odd}
		\end{cases}$$
		And this is really the fact that we're using in this computation.
	\end{nb}
	\begin{exercise}
		Use this formula to compute $5^{110}\pmod{16}$, writing out each step sequentially.
	\end{exercise}
}
\end{document}

