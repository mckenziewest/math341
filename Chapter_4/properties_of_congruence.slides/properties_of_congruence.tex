\documentclass[t]{beamer}

\subtitle{Properties of Congruence}

\usepackage{amsthm,amsmath,amsfonts,hyperref,graphicx,color,multicol,soul}
\usepackage{enumitem,tikz,tikz-cd,setspace,mathtools}

%%%%%%%%%%
%Beamer Template Customization
%%%%%%%%%%
\setbeamertemplate{navigation symbols}{}
\setbeamertemplate{theorems}[ams style]
\setbeamertemplate{blocks}[rounded]

\definecolor{Blu}{RGB}{43,62,133} % UWEC Blue
\setbeamercolor{structure}{fg=Blu} % Titles

%Unnumbered footnotes:
\newcommand{\blfootnote}[1]{%
	\begingroup
	\renewcommand\thefootnote{}\footnote{#1}%
	\addtocounter{footnote}{-1}%
	\endgroup
}

%%%%%%%%%%
%TikZ Stuff
%%%%%%%%%%
\usetikzlibrary{arrows}
\usetikzlibrary{shapes.geometric}
\tikzset{
	smaller/.style={
		draw,
		regular polygon,
		regular polygon sides=3,
		fill=white,
		node distance=2cm,
		minimum height=1in,
		line width = 2pt
	}
}
\tikzset{
	smsquare/.style={
		draw,
		regular polygon,
		regular polygon sides=4,
		fill=white,
		node distance=2cm,
		minimum height=1in,
		line width = 2pt
	}
}


%%%%%%%%%%
%Custom Commands
%%%%%%%%%%

\newcommand{\C}{\mathbb{C}}
\newcommand{\quats}{\mathbb{H}}
\newcommand{\N}{\mathbb{N}}
\newcommand{\Q}{\mathbb{Q}}
\newcommand{\R}{\mathbb{R}}
\newcommand{\Z}{\mathbb{Z}}

\newcommand{\ds}{\displaystyle}

\newcommand{\fn}{\insertframenumber}

\newcommand{\id}{\operatorname{id}}
\newcommand{\im}{\operatorname{im}}
\newcommand{\Aut}{\operatorname{Aut}}
\newcommand{\Inn}{\operatorname{Inn}}

\newcommand{\blank}[1]{\underline{\hspace*{#1}}}

\newcommand{\abar}{\overline{a}}
\newcommand{\bbar}{\overline{b}}
\newcommand{\cbar}{\overline{c}}

\newcommand{\nml}{\unlhd}

%%%%%%%%%%
%Custom Theorem Environments
%%%%%%%%%%
\theoremstyle{definition}
\newtheorem{exercise}{Exercise}
\newtheorem{question}[exercise]{Question}
\newtheorem{warmup}{Warm-Up}
\newtheorem*{defn}{Definition}
\newtheorem*{exa}{Example}
\newtheorem*{disc}{Group Discussion}
\newtheorem*{nb}{Note}
\newtheorem*{recall}{Recall}
\renewcommand{\emph}[1]{{\color{blue}\texttt{#1}}}

\definecolor{Gold}{RGB}{237, 172, 26}
%Statement block
\newenvironment{statementblock}[1]{%
	\setbeamercolor{block body}{bg=Gold!20}
	\setbeamercolor{block title}{bg=Gold}
	\begin{block}{\textbf{#1.}}}{\end{block}}
\newenvironment{thm}[1]{%
	\setbeamercolor{block body}{bg=Gold!20}
	\setbeamercolor{block title}{bg=Gold}
	\begin{block}{\textbf{Theorem #1.}}}{\end{block}}


%%%%%%%%%%
%Custom Environment Wrappers
%%%%%%%%%%
\newcommand{\enumarabic}[1]{
	\begin{enumerate}[label=\textbf{\arabic*.}]
		#1
	\end{enumerate}
}
\newcommand{\enumalph}[1]{
	\begin{enumerate}[label=(\alph*)]
		#1
	\end{enumerate}
}
\newcommand{\bulletize}[1]{
	\begin{itemize}[label=$\bullet$]
		#1
	\end{itemize}
}
\newcommand{\circtize}[1]{
	\begin{itemize}[label=$\circ$]
		#1
	\end{itemize}
}
\newcommand{\slide}[1]{
	\begin{frame}{\fn}
		#1
	\end{frame}
}
\newcommand{\slidec}[1]{
\begin{frame}[c]{\fn}
	#1
\end{frame}
}
\newcommand{\slidet}[2]{
	\begin{frame}{\fn\ - #1}
		#2
	\end{frame}
}


\newcommand{\startdoc}{
		\title{Math 425: Abstract Algebra 1}
		\author{Mckenzie West}
		\date{Last Updated: \today}
		\begin{frame}
			\maketitle
		\end{frame}
}

\newcommand{\topics}[2]{
	\begin{frame}{\insertframenumber}
		\begin{block}{\textbf{Last Section.}}
			\begin{itemize}[label=--]
				#1
			\end{itemize}
		\end{block}
		\begin{block}{\textbf{This Section.}}
			\begin{itemize}[label=--]
				#2
			\end{itemize}
		\end{block}
	\end{frame}
}
\usepackage{pgfplots}

\begin{document} 
	\startdoc
	
	\topics{
		% Last Time
		\item Introduction to Congruence
		\item Going Modulo First
	}
	{
		% This time
		\item Properties of Congruence
	}

\slide{
	\begin{defn}
		Let $n$ be a positive integer. We will say the integers $a$ and $b$ are \emph{congruent modulo $n$} if $n$ divides $a-b$.  We denote this by
			\[a\equiv b\pmod n.\]
	\end{defn}
}
\slide{
	\begin{thm}{4.2 (Equivalence Relation)}
		Let $n>1$ be fixed and $a, b, c$ be arbitrary integers.  Then the following properties hold:
		\enumalph{
			\item $a\equiv a\pmod n$
			\item If $a\equiv b\pmod n$ then $b\equiv a\pmod n$.
			\item If $a\equiv b\pmod n$ and $b\equiv c\pmod n$, then $a\equiv c\pmod n$.
		}
	\end{thm}
}
\slide{
	\begin{thm}{4.2 (Algebraic properties)}
		Let $n>1$ be fixed and $a, b, c, d$ be arbitrary integers.  Then the following properties hold:
		\enumalph{
			\item[(d)] If $a\equiv b\pmod n$ and $c\equiv d\pmod n$, then $a+c\equiv b+d\pmod n$ and $ac\equiv bd\pmod n$.
			\item[(e)] If $a\equiv b\pmod n$, then $a+b\equiv b+c\pmod n$ and $ac\equiv bc\pmod n$.
			\item[(f)] If $a\equiv b\pmod n$, then $a^k\equiv b^k\pmod n$ for all positive integers $k$.
		}
	\end{thm}
}
\slide{
	\begin{statementblock}{Theorem 4.3}
		If $ca\equiv cb\pmod n$, then $a\equiv b\pmod{n/d}$, where $d=\gcd(n,c)$.
	\end{statementblock}
	\begin{exa}
		\enumalph{
			\item $2\cdot 4\equiv 2\cdot 1\pmod 6$\vskip .25in
			\item $-35\equiv 45\pmod 8$\vskip .25in
		}
	\end{exa}
}
\slide{
	\begin{statementblock}{Corollary 1}
		If $ca\equiv cb\pmod n$ and $\gcd(c,n)=1$, then $a\equiv b\pmod n$.
	\end{statementblock}
	\begin{exercise}
		Find all $x$ such that $3498x\equiv 0\pmod{175}$.
	\end{exercise}
}
\slide{
	\begin{exercise}
		Find all $x$ such that $3498x\equiv 4\pmod{175}$.
		
		(It'll be helpful to note $3498\cdot 87\equiv 1\pmod{175}$.)
	\end{exercise}
}
\slide{
	\begin{statementblock}{Corollary 2}
		If $ca\equiv cb\pmod p$ and $p\nmid c$, where $p$ is a prime number, then $a\equiv b\pmod p$.
	\end{statementblock}
}
\end{document}

