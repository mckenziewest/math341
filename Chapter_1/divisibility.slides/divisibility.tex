\documentclass[t]{beamer}

\subtitle{Divisibility and Statements}

\usepackage{amsthm,amsmath,amsfonts,hyperref,graphicx,color,multicol,soul}
\usepackage{enumitem,tikz,tikz-cd,setspace,mathtools}

%%%%%%%%%%
%Beamer Template Customization
%%%%%%%%%%
\setbeamertemplate{navigation symbols}{}
\setbeamertemplate{theorems}[ams style]
\setbeamertemplate{blocks}[rounded]

\definecolor{Blu}{RGB}{43,62,133} % UWEC Blue
\setbeamercolor{structure}{fg=Blu} % Titles

%Unnumbered footnotes:
\newcommand{\blfootnote}[1]{%
	\begingroup
	\renewcommand\thefootnote{}\footnote{#1}%
	\addtocounter{footnote}{-1}%
	\endgroup
}

%%%%%%%%%%
%TikZ Stuff
%%%%%%%%%%
\usetikzlibrary{arrows}
\usetikzlibrary{shapes.geometric}
\tikzset{
	smaller/.style={
		draw,
		regular polygon,
		regular polygon sides=3,
		fill=white,
		node distance=2cm,
		minimum height=1in,
		line width = 2pt
	}
}
\tikzset{
	smsquare/.style={
		draw,
		regular polygon,
		regular polygon sides=4,
		fill=white,
		node distance=2cm,
		minimum height=1in,
		line width = 2pt
	}
}


%%%%%%%%%%
%Custom Commands
%%%%%%%%%%

\newcommand{\C}{\mathbb{C}}
\newcommand{\quats}{\mathbb{H}}
\newcommand{\N}{\mathbb{N}}
\newcommand{\Q}{\mathbb{Q}}
\newcommand{\R}{\mathbb{R}}
\newcommand{\Z}{\mathbb{Z}}

\newcommand{\ds}{\displaystyle}

\newcommand{\fn}{\insertframenumber}

\newcommand{\id}{\operatorname{id}}
\newcommand{\im}{\operatorname{im}}
\newcommand{\lcm}{\operatorname{lcm}}
\newcommand{\Aut}{\operatorname{Aut}}
\newcommand{\Inn}{\operatorname{Inn}}

\newcommand{\blank}[1]{\underline{\hspace*{#1}}}

\newcommand{\abar}{\overline{a}}
\newcommand{\bbar}{\overline{b}}
\newcommand{\cbar}{\overline{c}}

\newcommand{\nml}{\unlhd}

%%%%%%%%%%
%Custom Theorem Environments
%%%%%%%%%%
\theoremstyle{definition}
\newtheorem{exercise}{Exercise}
\newtheorem{question}[exercise]{Question}
\newtheorem{warmup}{Warm-Up}
\newtheorem*{exa}{Example}
\newtheorem*{disc}{Group Discussion}
\newtheorem*{recall}{Recall}
\renewcommand{\emph}[1]{{\color{blue}\texttt{#1}}}

\definecolor{Gold}{RGB}{237, 172, 26}
%Statement block
\newenvironment{statementblock}[1]{%
	\setbeamercolor{block body}{bg=Gold!20}
	\setbeamercolor{block title}{bg=Gold}
	\begin{block}{\textbf{#1.}}}{\end{block}}
\newenvironment{goldblock}{%
	\setbeamercolor{block body}{bg=Gold!20}
	\setbeamercolor{block title}{bg=Gold}
	\setbeamertemplate{blocks}[shadow=true]
	\begin{block}{}}{\end{block}}
\newenvironment{defn}{%
	\setbeamercolor{block body}{bg=gray!20}
	\setbeamercolor{block title}{bg=violet, fg=white}
	\setbeamertemplate{blocks}[shadow=true]
	\begin{block}{\textbf{Definition.}}}{\end{block}}
\newenvironment{nb}{%
	\setbeamercolor{block body}{bg=gray!20}
	\setbeamercolor{block title}{bg=teal, fg=white}
	\setbeamertemplate{blocks}[shadow=true]
	\begin{block}{\textbf{Note.}}}{\end{block}}
\newenvironment{blockexample}{%
	\setbeamercolor{block body}{bg=gray!20}
	\setbeamercolor{block title}{bg=Blu, fg=white}
	\setbeamertemplate{blocks}[shadow=true]
	\begin{block}{\textbf{Example.}}}{\end{block}}
\newenvironment{blocknonexample}{%
	\setbeamercolor{block body}{bg=gray!20}
	\setbeamercolor{block title}{bg=purple, fg=white}
	\setbeamertemplate{blocks}[shadow=true]
	\begin{block}{\textbf{Non-Example.}}}{\end{block}}
\newenvironment{thm}[1]{%
	\setbeamercolor{block body}{bg=Gold!20}
	\setbeamercolor{block title}{bg=Gold}
	\begin{block}{\textbf{Theorem #1.}}}{\end{block}}


%%%%%%%%%%
%Custom Environment Wrappers
%%%%%%%%%%
\newcommand{\exer}[1]{
	\begin{exercise}
		#1
	\end{exercise}
}
\newcommand{\exam}[1]{
\begin{blockexample}
	#1
\end{blockexample}
}
\newcommand{\nexam}[1]{
\begin{blocknonexample}
	#1
\end{blocknonexample}
}
\newcommand{\enumarabic}[1]{
	\begin{enumerate}[label=\textbf{\arabic*.}]
		#1
	\end{enumerate}
}
\newcommand{\enumalph}[1]{
	\begin{enumerate}[label=(\alph*)]
		#1
	\end{enumerate}
}
\newcommand{\bulletize}[1]{
	\begin{itemize}[label=$\bullet$]
		#1
	\end{itemize}
}
\newcommand{\circtize}[1]{
	\begin{itemize}[label=$\circ$]
		#1
	\end{itemize}
}
\newcommand{\slide}[1]{
	\begin{frame}{\fn}
		#1
	\end{frame}
}
\newcommand{\slidec}[1]{
\begin{frame}[c]{\fn}
	#1
\end{frame}
}
\newcommand{\slidet}[2]{
	\begin{frame}{\fn\ - #1}
		#2
	\end{frame}
}


\newcommand{\startdoc}{
		\title{Math 341: Classical Number Theory}
		\author{Mckenzie West}
		\date{Last Updated: \today}
		\begin{frame}
			\maketitle
		\end{frame}
}

\newcommand{\topics}[2]{
	\begin{frame}[c]{\insertframenumber}
		\begin{block}{\textbf{Last Section.}}
			\begin{itemize}[label=--]
				#1
			\end{itemize}
		\end{block}
		\begin{block}{\textbf{This Section.}}
			\begin{itemize}[label=--]
				#2
			\end{itemize}
		\end{block}
	\end{frame}
}

\begin{document} 
	\startdoc
	
	\topics{
		% Last Time
		\item Introduction to Number Theory
	}
	{
		% This time
		\item Divisibility
		\item Statements
		\item Direct Proof
		\item Induction
	}
\slide{
	\begin{statementblock}{Well Ordering Principle}
		Any nonempty set of positive integers has a least/smallest element.
	\end{statementblock}
	\begin{corollary}
		There are no integers between 0 and 1.
	\end{corollary}
}
\slide{
	\begin{defn}
		The integer $d$ \emph{divides} the integer $n$ if there is some integer $k$ for which
			$$dk=n.$$
			
		In this case, we write \emph{$d|n$} to mean 
		\begin{center}
			``$d$ divides $n$'' or
			 ``$n$ is divisible by $d$'' or ``$d$ is a divisor of $n$.
		\end{center}
		The statement \emph{$d\nmid n$} means that ``$d$ does not divide $n$''.
	\end{defn}
	\exam{
		We have the following
			$$8|32\quad 8\nmid 18\quad 8\mid -8\quad -8\mid 16.$$
	}
}
\slide{
	\exer{
		Find all divisors of $15$. 
	}\vfill
	\exer{
		Find all divisors of $27$.
	}\vfill
	\exer{
		What's the relationship between the prime divisors of a number and its divisors?
	}\vfill
}
\slide{  
	\begin{statementblock}{Divisibility Facts}
		For integers $a,b,c$, and $d$, the following hold
		\enumalph{ 
			\item $a|0$, $1|a$, $a|a$.
			\item $a|1$ if and only if $a=\pm 1$.
			\item If $a|b$ and $c|d$, then $ac|bd$.
			\item If $a|b$ and $b|c$, then $a|c$.
			\item $a|b$ and $b|a$ if and only if $|a|=|b|$.
			\item If $a|b$ and $b\neq 0$, then $|a|\leq |b|$.
			\item If $a|b$ and $a|c$, then $a|(bx+cy)$ for all integers $x$ and $y$.
		} 
	\end{statementblock}
	\begin{nb}
		Burton proves parts (f) and (g) on page 20. Some of these will be on homework 2 for you to prove, (a), (c), and (e), We'll do some here to get our proof chops working.
	\end{nb}
}
\slide{  
	Space reserved for at least one proof of divisibility facts.
}
\slide{
	\begin{defn}
		A \emph{proper divisor} of a positive integer $n$ is a positive integer $d$ such that $d|n$ and $d<n$.
	\end{defn}
	\exam{The proper divisors of 15 are 1, 3, and 5.}
	\begin{statementblock}{Proper Divisibility Fact}
		If $a$ is a proper divisor of $n$, then for all positive integers $b$, we have $ab$ is a proper divisor of $nb$.
	\end{statementblock}
}
\slide{
	\begin{defn}
		A \emph{statement} is a declarative sentence that is either true or false but not both. 
	\end{defn}
	\exam{
		\bulletize{ 
			\item For all integers $a$, we have $a|0$.
			\item $2|3$
		}
	}
	\nexam{  
		\bulletize{\item Find a divisor of 3.\item $1|b$}
	}
	\exer{Come up with some of your own examples and non-examples.}
}
\slide{
	\begin{defn}
		The \emph{negation} of a statement is the opposite statement. 
		
		If $P$ denotes the original statement, $\sim P$ denotes its negation.
	\end{defn}
	\exam{The negation of \begin{center}
			``For all integers $a$, we have $a|9$''
		\end{center} is\begin{center}
		 ``There is some integer $a$ with $a\nmid 9$.''
	\end{center}}
	
}
\slide{  
	\begin{statementblock}{Negations}
		\bulletize{
			\item``for all'' $\xleftrightarrow{negates\ to}$ ``for some''
			\item ``$P$ and $Q$'' $\xlongrightarrow{negates\ to}$ ``$\sim P$ or $\sim Q$''
			\item ``$P$ or $Q$'' $\xlongrightarrow{negates\ to}$ ``$\sim P$ and $\sim Q$''.
			\item``$=$'' $\xleftrightarrow{negates\ to}$ ``$\neq$''
			\item``$\leq$'' $\xleftrightarrow{negates\ to}$ ``$>$''
			\item``$\geq$'' $\xleftrightarrow{negates\ to}$ ``$<$''
		}
	\end{statementblock}
	\exer{Let $x$ represent an integer. What is the negation of the following statement?
		\begin{center}
			``$2|x$ or $3|x$.''
		\end{center}}
}
\slide{  
	\begin{defn}
		A \emph{conditional statement} is a statement that can be written in the form ``If $P$ then $Q$'' where $P$ and $Q$ are also statements, called the \emph{hypothesis} and \emph{conclusion}, respectively.
	\end{defn}
	\exam{Let $a,b$ be integers.
		\begin{center}
			``If $a|b$ and $b\neq 0$, then $|a|\leq |b|$.''
		\end{center}
	}
	\exer{Write the following statement in the form ``If $P$ then $Q$.''\begin{center}
			``No integers satisfy $2x=3$.''
	\end{center}}
}
\slide{
	\begin{statementblock}{Negation}
		The negation of ``If $P$ then $Q$'' is ``$P$ and $\sim Q$''.
	\end{statementblock}
	\exer{What is the negation of the conditional statement
		\begin{center}
			``If $x$ and $y$ are integers, then $4x+6y\neq 1$.''
		\end{center}
		Note, there is a secret quantifier here: $x$ and $y$ are real numbers.
	}
}
\slide{  
	\begin{defn}
		The \emph{contrapositive} of the statement $P\Rightarrow Q$ is $\sim Q\Rightarrow \sim P$.
		
		The \emph{converse} of the statement $P\Rightarrow Q$ is $Q\Rightarrow P$.
	\end{defn}
	\begin{nb}
		The contrapositive of a statement is logically equivalent to it and is sometimes easier to prove!
		
		The converse of a statement is NOT logically equivalent to it.  
	\end{nb}
	\exer{What is the contrapositive and the converse of the following statement about integers $a,b$?
		\begin{center}
			``If $|a|=|b|$, then $a|b$ and $b|a$.''
	\end{center}}
}
\slide{
	\begin{defn}
		The most common form of proof, the \emph{direct proof} follows a sequence of definitions and known results to show that the given result is true.
	\end{defn}
	\begin{nb}
		Direct proofs of conditional statements should generally start by ``let''ing variables exist and ``assume''ing properties of those variables. 
		\vskip .15in
		\fbox{\begin{minipage}{.9\textwidth}
				Prove that for all integers $a,b,c$ if $a|b$ and $b|c$, then $a|c$.
		
		
			\textbf{Proof.} Let $a,b,c$ be integers. Assume $a|b$ and $b|c$.
		\end{minipage}}
	\end{nb}
}
\slide{
	\begin{defn}
		A form of direct proof for conditional statements, called \emph{proof by contrapositive}, uses the fact that the contrapositive is logically equivalent to the statement itself to prove the result.
	\end{defn}
	\exam{It would be nice to use proof by contrapositive for proving 
		\begin{center}
			``If $x$ and $y$ are integers, then $4x+6y\neq 1$.''
	\end{center}}
}
\frame{
	\begin{defn}
		A \emph{proof by contradiction} is a proof of a statement that involves 
		\enumarabic{
			\item Assume the statement is false (aka that its negation is true).
			\item Following a sequence of definitions and known results, finds an impossibility.
			\item Concluding that the assumption about the statement being false could not be possible.
		}
	\end{defn}
	\exam{
		The most common example of proof by contradiction is for the statement 
			\begin{center}
				``$\sqrt{2}$ is irrational.''
			\end{center}
	}
}
\slide{
	\begin{thm}{1.2 (Burton)}
		Let $S$ be a set of positive integers with the following properties
		\enumalph{\item The integer 1 belongs to $S$.\item Whenever the integer $k$ is in $S$, the next integer $k+1$ must also be in $S$.}
		Then $S$ is the set of all positive integers.
	\end{thm}
}
\slide{
	\exer{Use Theorem 1.2 here to prove 
			$$1+3+5+7+\cdots+(2n-1)=n^2$$
		for all $n\geq 1$.}
}
\slide{
	\begin{defn}
		A \emph{proof by induction} is a proof that employs Burton's Theorem 1.2.
	\end{defn}
	\begin{nb}
		\bulletize{
			\item Often the  set $S$ is omitted from conversation.  
			\item This method can be used to prove a statement ``for all $n\geq N$'' where $N$ is some fixed lower bound.  For example
				\begin{center}
					``$n^2<n!$ for all $n\geq 4$''
				\end{center}
		}
	\end{nb}
}
\end{document}

