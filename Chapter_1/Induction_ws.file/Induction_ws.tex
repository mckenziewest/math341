\documentclass[12pt]{article}
\usepackage{amsmath,amsthm,amssymb,hyperref,fancyhdr,color,pgffor}
\usepackage[margin=.75in]{geometry}

\newcommand{\R}{\mathbb{R}}
\newcommand{\Z}{\mathbb{Z}}

\pagestyle{fancy} 
\fancyhf{}\renewcommand{\headrulewidth}{0pt}

\newcommand{\one}{$6^n+4$ is divisible by 5 for all $n\geq 1$.}
\newcommand{\two}{$2^{2n}-1$ is divisible by 3 for all $n\geq 1$.}
\newcommand{\three}{$9^n+3$ is divisible by 4 for all $n\geq 1$}
\newcommand{\four}{$8^n-3^n$ is divisible by \underline{\hspace*{1cm}} for all $n\geq 1$}
\newcommand{\five}{$9^n-2^n$ is divisible by \underline{\hspace*{1cm}} for all $n\geq 1$}
\newcommand{\six}{$5^n+2\cdot 11^n$ is divisible by \underline{\hspace*{1cm}} for all $n\geq 1$}

\begin{document}
	\foreach\version in {1,2,3}{
	\noindent Math 341: Classical Number Theory\hfill Introduction to Number Theory\\
	\mbox{}\hfill Induction
	\setcounter{page}{1}
	\fancyfoot[C]{\version-\thepage}
\begin{enumerate}
		\item Use proof by induction to show 
		\if\version1
			\one
		\else
		\if\version2
			\two
		\else
			\three
		\fi\fi
		\newpage
		\item Do some computation to first conjecture the value for the blank, then use induction to prove your conjecture.
		\if\version1
		\four
		\else
		\if\version2
		\five
		\else
		\six
		\fi\fi
			
			
\end{enumerate}
\newpage
}
\end{document}