\documentclass[t]{beamer}

\subtitle{Fermat's Little Theorem and Wilson's Theorem}

\usepackage{amsthm,amsmath,amsfonts,hyperref,graphicx,color,multicol,soul}
\usepackage{enumitem,tikz,tikz-cd,setspace,mathtools}

%%%%%%%%%%
%Beamer Template Customization
%%%%%%%%%%
\setbeamertemplate{navigation symbols}{}
\setbeamertemplate{theorems}[ams style]
\setbeamertemplate{blocks}[rounded]

\definecolor{Blu}{RGB}{43,62,133} % UWEC Blue
\setbeamercolor{structure}{fg=Blu} % Titles

%Unnumbered footnotes:
\newcommand{\blfootnote}[1]{%
	\begingroup
	\renewcommand\thefootnote{}\footnote{#1}%
	\addtocounter{footnote}{-1}%
	\endgroup
}

%%%%%%%%%%
%TikZ Stuff
%%%%%%%%%%
\usetikzlibrary{arrows}
\usetikzlibrary{shapes.geometric}
\tikzset{
	smaller/.style={
		draw,
		regular polygon,
		regular polygon sides=3,
		fill=white,
		node distance=2cm,
		minimum height=1in,
		line width = 2pt
	}
}
\tikzset{
	smsquare/.style={
		draw,
		regular polygon,
		regular polygon sides=4,
		fill=white,
		node distance=2cm,
		minimum height=1in,
		line width = 2pt
	}
}


%%%%%%%%%%
%Custom Commands
%%%%%%%%%%

\newcommand{\C}{\mathbb{C}}
\newcommand{\quats}{\mathbb{H}}
\newcommand{\N}{\mathbb{N}}
\newcommand{\Q}{\mathbb{Q}}
\newcommand{\R}{\mathbb{R}}
\newcommand{\Z}{\mathbb{Z}}

\newcommand{\ds}{\displaystyle}

\newcommand{\fn}{\insertframenumber}

\newcommand{\id}{\operatorname{id}}
\newcommand{\im}{\operatorname{im}}
\newcommand{\Aut}{\operatorname{Aut}}
\newcommand{\Inn}{\operatorname{Inn}}

\newcommand{\blank}[1]{\underline{\hspace*{#1}}}

\newcommand{\abar}{\overline{a}}
\newcommand{\bbar}{\overline{b}}
\newcommand{\cbar}{\overline{c}}

\newcommand{\nml}{\unlhd}

%%%%%%%%%%
%Custom Theorem Environments
%%%%%%%%%%
\theoremstyle{definition}
\newtheorem{exercise}{Exercise}
\newtheorem{question}[exercise]{Question}
\newtheorem{warmup}{Warm-Up}
\newtheorem*{defn}{Definition}
\newtheorem*{exa}{Example}
\newtheorem*{disc}{Group Discussion}
\newtheorem*{nb}{Note}
\newtheorem*{recall}{Recall}
\renewcommand{\emph}[1]{{\color{blue}\texttt{#1}}}

\definecolor{Gold}{RGB}{237, 172, 26}
%Statement block
\newenvironment{statementblock}[1]{%
	\setbeamercolor{block body}{bg=Gold!20}
	\setbeamercolor{block title}{bg=Gold}
	\begin{block}{\textbf{#1.}}}{\end{block}}
\newenvironment{thm}[1]{%
	\setbeamercolor{block body}{bg=Gold!20}
	\setbeamercolor{block title}{bg=Gold}
	\begin{block}{\textbf{Theorem #1.}}}{\end{block}}


%%%%%%%%%%
%Custom Environment Wrappers
%%%%%%%%%%
\newcommand{\enumarabic}[1]{
	\begin{enumerate}[label=\textbf{\arabic*.}]
		#1
	\end{enumerate}
}
\newcommand{\enumalph}[1]{
	\begin{enumerate}[label=(\alph*)]
		#1
	\end{enumerate}
}
\newcommand{\bulletize}[1]{
	\begin{itemize}[label=$\bullet$]
		#1
	\end{itemize}
}
\newcommand{\circtize}[1]{
	\begin{itemize}[label=$\circ$]
		#1
	\end{itemize}
}
\newcommand{\slide}[1]{
	\begin{frame}{\fn}
		#1
	\end{frame}
}
\newcommand{\slidec}[1]{
\begin{frame}[c]{\fn}
	#1
\end{frame}
}
\newcommand{\slidet}[2]{
	\begin{frame}{\fn\ - #1}
		#2
	\end{frame}
}


\newcommand{\startdoc}{
		\title{Math 425: Abstract Algebra 1}
		\author{Mckenzie West}
		\date{Last Updated: \today}
		\begin{frame}
			\maketitle
		\end{frame}
}

\newcommand{\topics}[2]{
	\begin{frame}{\insertframenumber}
		\begin{block}{\textbf{Last Section.}}
			\begin{itemize}[label=--]
				#1
			\end{itemize}
		\end{block}
		\begin{block}{\textbf{This Section.}}
			\begin{itemize}[label=--]
				#2
			\end{itemize}
		\end{block}
	\end{frame}
}

\begin{document} 
	\startdoc
	
	\topics{
		% Last Time
		\item Hensel's Lemma
		\item p-adic numbers
	}
	{
		% This time
		\item Fermat's Little Theorem
		\item Wilson's Theorem
	}

\slide{
	\begin{thm}{5.1 (Fermat's [Little] Theorem)}
		Let $p$ be a prime and suppose $p\nmid a$.  Then 
			\[a^{p-1}\equiv 1\pmod p.\]
	\end{thm}
}
\slide{
\begin{exercise}
	Use Fermat's Little Theorem to compute $9^{794}\pmod{73}$.
\end{exercise}
}
\slide{
	\begin{exercise}
		Use Fermat's Little Theorem to solve the congruence $x^{86}\equiv 6\pmod{29}$.
	\end{exercise}
}
\slide{
\begin{exercise}
	Use Fermat's Little Theorem to solve the congruence $x^{39}\equiv 3\pmod{13}$.
\end{exercise}
}
\slide{
	\begin{exercise}
		Use Fermat's Little Theorem to show that $x^6+y^{12}=700003$ has no integer solutions.  (Hint look modulo 7.)
	\end{exercise}
}
\slide{
	To prove Ferma't Little Theorem, we use the second Corollary to Theorem 4.3:  
	\begin{statementblock}{Corollary 2}
		If $p$ is a prime not dividing $c$ and $ac\equiv bc\pmod p$, then $a\equiv b\pmod p$.  That is, we have ``cancellation modulo $p$'' of non-zero integers.
	\end{statementblock}
	
	We also remind ourselves of the fact that $a\equiv b\pmod p$ if and only if $a$ and $b$ have the same remainder when divided by $p$.
	
	Therefore, no two numbers $0\leq r<p$ are equivalent modulo $p$.
}
\slide{ \vskip -.25in
	\begin{proof}
		Let $p$ be a prime and $a$ and integer not divisible by $p$.
		\bulletize{
			\item Consider the integers $a,2a,3a,4a,\dots,(p-1)a$. \textbf{Claim:} None of these are congruent modulo $p$.\vskip .5in
			\item \textbf{Claim:} Thus the integers $a,2a,3a,4a,\dots,(p-1)a$ correspond to the integers $1,2,3,\dots,p-1$ modulo $p$, possibly in a different order.\vskip .5in
			\item \textbf{Claim:} $a\cdot 2a\cdot 3a\cdots (p-1)a\equiv (p-1)!\pmod p$\vskip .5in
			\item \textbf{Claim:} $a^{p-1}\equiv 1\pmod p$ 
		}
	\end{proof}
}
\slide{
	\begin{exa}
		For verification, let $p=5$ and $a=7$.
		\enumalph{
			\item $a$, $2a$, $3a$, $4a$?\vskip .5in
			\item Correspondence with $1,2,3,4$?\vskip .5in
			\item $a\cdot 2a\cdot 3a\cdot 4a$ vs $4!$?\vskip .5in
			\item $7^4\equiv 1\pmod 5$?
		}
	\end{exa}
}
\slide{
	\begin{statementblock}{Corollary}
		If $p$ is prime, then $a^p\equiv a\pmod p$ for all $a\in\Z$.
	\end{statementblock}
}
\slide{
	\begin{exercise}
		The congruence $7^{1734250}\equiv1660565\pmod{1734251}$ is true. Can we conclude 1734251 is composite?
	\end{exercise}\vskip 1in
	\begin{exercise}
		The congruence $2^{52632}\equiv 1\pmod{52633}$ is true. Can we conclude that 52633 is a prime number?
	\end{exercise}
}
\slide{
	Since $a^{p-1}\equiv 1\pmod p$ for all $p\nmid a$ if $p$ is prime, it might be natural to use this as a primality test:
	
	\begin{statementblock}{A Contrapositive of Fermat's Little Theorem}
		If there is some $a$ for which $n\nmid a$ and $a^{n-1}\not\equiv 1\pmod n$ then $n$ is necessarily composite.
	\end{statementblock}
	\begin{exercise}
		Show that $117$ is composite by simplifying $2^{116}\pmod{117}$.  
		
		(Hint: $2^{7}\equiv 11\pmod{117}$.)
	\end{exercise}
}
\slide{
	\begin{defn}
		Here we call 2 a \emph{witness} to the non-primality of 117.  That is,  if $n$ is a positive integer not dividing $a$ and $a^{n-1}\not\equiv 1\pmod{n}$ then we say $a$ \emph{witnesses} the non-primality of $n$.
	\end{defn}
	\begin{exercise}
		Does $2$ witness the non-primality of 91?
	\end{exercise}
	\vskip .5in
	\begin{exercise}
		Does $3$ witness the non-primality of 91?
	\end{exercise}
}
\slide{
	One related quick check on primality of an integer $n$ is whether $2^{n}\equiv 2\pmod n$.  Equivalently, does $n| 2^n-2$? If not, then $n$ can't be prime.
	\vskip .05in
	
	It was believed, for centuries that $n$ is prime if and only if $n|2^n-2$. This was debunked in 1819 when finally it was found that $341|2^{341}-2$.
	\vskip .05in
	
	\begin{defn}
		An composite integer $n$ is called a \emph{pseudoprime} if $n|2^n-2$.
	\end{defn}
	
	\begin{exa}
		The four smallest Pseudoprimes are 341, 561, 645, 1105.
		
		Notice that these are spaced decently far apart and rare.  So this really is a \textit{decent} way to check.
	\end{exa}
}
\slide{
	\begin{statementblock}{Theorem 5.2}
		If $n$ is an odd pseudoprime, then $M_n=2^n-1$ is a larger one.
	\end{statementblock}
	\begin{question}
		How many pseudoprimes are there?
	\end{question}
}
\slide{
	\begin{nb}
		In 1994, it was shown that there are infinitely many composite integer $n$ for which $a^n\equiv a\pmod n$ for all integers $a$.
		
		Such integers are called \emph{absolute pseudoprimes} or \emph{Carmichael numbers}.
	\end{nb}
	\begin{statementblock}{Theorem 5.3}
		Let $n$ be a composite square-free integer, say $n=p_1p_2\cdots p_r$ where the $p_i$ are distinct primes.  If $p_i-1|n-1$ for $i=1,2,\dots,r$, then $n$ is an absolute pseudoprime.
	\end{statementblock}
	\begin{exercise}
		Use this theorem to show that $6601=7\cdot 23\cdot 41$ is an absolute pseudoprime.
	\end{exercise}
}
\slide{
	\begin{statementblock}{Wilson's Theorem (Theorem 5.4)}
		If $p$ is prime then $(p-1)!\equiv -1\pmod p$.
	\end{statementblock}
}
\slide{\vskip -.25in
	\begin{exercise}
		Rather than going through the entire proof of Wilson's Theorem, we will illustrate it for the case $p=11$.  For the full proof, see page 94 of the textbook.
		
		Theorem 4.7 says that for all $a=1,2,\dots, 10$, there is some $a'$ such that
		\[aa'\equiv 1\pmod{11}.\]
		Find this $a'$ for the $a$ values:
		$$\begin{array}{|c|c|}
			\hline
			\textbf{a}&\textbf{a'}\\\hline\hline
			1&{\color{white}answer}\\\hline
			2&\\\hline
			3&\\\hline
			4&\\ \hline
			5&\\\hline
		\end{array}\quad
		\begin{array}{|c|c|}
			\hline
			\textbf{a}&\textbf{a'}\\\hline\hline
			6&{\color{white}answer}\\\hline
			7&\\\hline
			8&\\\hline
			9&\\\hline
			10&\\\hline
		\end{array}$$
		Now compute $10!\pmod{11}$ by grouping up/canceling all possible pairs $a,a'$. 
	\end{exercise}
}
\slide{
	\begin{statementblock}{Theorem 5.5 (v1)}
		Let $p$ be a prime of the form $4k+1$ ($k\in\Z$).  Then 
		\[x=\left(\frac{p-1}{2}\right)!\]
		is a solution to the quadratic congruence equation
		\[x^2+1\equiv 0\pmod p.\]
	\end{statementblock}
	\begin{exercise}
		Verify this statement for $p=13$.\vskip .5in
	\end{exercise}
	\begin{exercise}
		Show that if $p=11$ then $x^2+1\equiv 0\pmod p$ has no solutions.
	\end{exercise}
}
\slide{
	\begin{exercise}
		As you might expect, the proof of this Theorem uses Wilson's Theorem.
		
		First note that for example 
		\begin{eqnarray}
			(13-1)!&=&1\cdot2\cdot3\cdot4\cdot5\cdot6\cdot7\cdot8\cdot9\cdot10\cdot11\cdot12\\
			&\equiv&1\cdot2\cdot3\cdot4\cdot5\cdot6\cdot(-6)\cdot(-5)\cdot(-4)\cdot\\
			&&(-3)\cdot(-2)\cdot(-1)\pmod{13}\\
			&=&(-1)^{(13-1)/2}(6!)^2\pmod{13}
		\end{eqnarray}
		Therefore by Wilson's Theorem $(6!)^2\equiv -1\pmod{13}$.
		
		\enumalph{
			\item Discuss why this proves the theorem for the case $p=13$.
			\item How does this generalize for other primes of the form $p=4k+1$.
			\item Why does this not work for primes of the form $p=4k+3$?
		}
	\end{exercise}
}
\slide{
	\begin{nb}
		If $n=ab$ is odd, then
		\[n=\left(\frac{a+b}{2}\right)^2-\left(\frac{a-b}{2}\right)^2.\]
		Conversely if $n=x^2-y^2$, then $n=(x-y)(x+y)$. 
		
		Therefore, one way to factor the integer $n$, would be to compute
		\[k^2-n,(k+1)^2-n,(k+2)^2-n,\dots\]
		starting with the smallest integer $k\geq \sqrt n$ until $x=\frac{n+1}{2}$, which would produce the trivial factorization $n=1\cdot n$.
	\end{nb}
}
\slide{
	\begin{exa}
		Let $n=119143$.  We start with $k=346$:
		\[
		\begin{array}{ll}
			346^2-119143=573&\text{not a square}\\
			347^2-119143=1266&\text{not a square}\\
			348^2-119143=1961&\text{not a square}\\
			349^2-119143=2658&\text{not a square}\\
			350^2-119143=3357&\text{not a square}\\
			351^2-119143=4058&\text{not a square}\\
			352^2-119143=4761=69^2&\text{a square!}\\
		\end{array}
		\]
		Therefore,
		\[119143 = 352^2-69^2=(352-69)(352+69)=421\cdot283\]
	\end{exa}
}
\slide{
	\begin{exercise}
		Use Fermat's method to factor 2279.
	\end{exercise}
}
\end{document}

