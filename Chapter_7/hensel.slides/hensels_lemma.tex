\documentclass[t]{beamer}

\subtitle{Hensel's Lemma}

\usepackage{listings}
\usepackage{caption, floatrow, makecell}%
\captionsetup{labelfont = sc}
\setcellgapes{3pt}

\definecolor{backcolour}{RGB}{237,236,230}
\definecolor{myblue}{RGB}{42,157,189}

\lstdefinestyle{mystyle}{
	language=Python,
	keywords=[2]{sage:},
	alsodigit={:,.,<,>},
	backgroundcolor=\color{backcolour},   
	commentstyle=\color{myblue},
	keywordstyle=\bfseries\color{Blu},
	keywordstyle=[2]\color{purple},
	numberstyle=\tiny\color{gray},
	stringstyle=\color{orange},
	basicstyle=\ttfamily\footnotesize,
	breakatwhitespace=false,         
	breaklines=true,                 
	captionpos=b,                    
	keepspaces=true,                   
	showspaces=false,                
	showstringspaces=false,
	showtabs=false,                  
	tabsize=2
}

\lstset{style=mystyle}
\usepackage{amsthm,amsmath,amsfonts,hyperref,graphicx,color,multicol,soul}
\usepackage{enumitem,tikz,tikz-cd,setspace,mathtools}

%%%%%%%%%%
%Beamer Template Customization
%%%%%%%%%%
\setbeamertemplate{navigation symbols}{}
\setbeamertemplate{theorems}[ams style]
\setbeamertemplate{blocks}[rounded]

\definecolor{Blu}{RGB}{43,62,133} % UWEC Blue
\setbeamercolor{structure}{fg=Blu} % Titles

%Unnumbered footnotes:
\newcommand{\blfootnote}[1]{%
	\begingroup
	\renewcommand\thefootnote{}\footnote{#1}%
	\addtocounter{footnote}{-1}%
	\endgroup
}

%%%%%%%%%%
%TikZ Stuff
%%%%%%%%%%
\usetikzlibrary{arrows}
\usetikzlibrary{shapes.geometric}
\tikzset{
	smaller/.style={
		draw,
		regular polygon,
		regular polygon sides=3,
		fill=white,
		node distance=2cm,
		minimum height=1in,
		line width = 2pt
	}
}
\tikzset{
	smsquare/.style={
		draw,
		regular polygon,
		regular polygon sides=4,
		fill=white,
		node distance=2cm,
		minimum height=1in,
		line width = 2pt
	}
}


%%%%%%%%%%
%Custom Commands
%%%%%%%%%%

\newcommand{\C}{\mathbb{C}}
\newcommand{\quats}{\mathbb{H}}
\newcommand{\N}{\mathbb{N}}
\newcommand{\Q}{\mathbb{Q}}
\newcommand{\R}{\mathbb{R}}
\newcommand{\Z}{\mathbb{Z}}

\newcommand{\ds}{\displaystyle}

\newcommand{\fn}{\insertframenumber}

\newcommand{\id}{\operatorname{id}}
\newcommand{\im}{\operatorname{im}}
\newcommand{\Aut}{\operatorname{Aut}}
\newcommand{\Inn}{\operatorname{Inn}}

\newcommand{\blank}[1]{\underline{\hspace*{#1}}}

\newcommand{\abar}{\overline{a}}
\newcommand{\bbar}{\overline{b}}
\newcommand{\cbar}{\overline{c}}

\newcommand{\nml}{\unlhd}

%%%%%%%%%%
%Custom Theorem Environments
%%%%%%%%%%
\theoremstyle{definition}
\newtheorem{exercise}{Exercise}
\newtheorem{question}[exercise]{Question}
\newtheorem{warmup}{Warm-Up}
\newtheorem*{defn}{Definition}
\newtheorem*{exa}{Example}
\newtheorem*{disc}{Group Discussion}
\newtheorem*{nb}{Note}
\newtheorem*{recall}{Recall}
\renewcommand{\emph}[1]{{\color{blue}\texttt{#1}}}

\definecolor{Gold}{RGB}{237, 172, 26}
%Statement block
\newenvironment{statementblock}[1]{%
	\setbeamercolor{block body}{bg=Gold!20}
	\setbeamercolor{block title}{bg=Gold}
	\begin{block}{\textbf{#1.}}}{\end{block}}
\newenvironment{thm}[1]{%
	\setbeamercolor{block body}{bg=Gold!20}
	\setbeamercolor{block title}{bg=Gold}
	\begin{block}{\textbf{Theorem #1.}}}{\end{block}}


%%%%%%%%%%
%Custom Environment Wrappers
%%%%%%%%%%
\newcommand{\enumarabic}[1]{
	\begin{enumerate}[label=\textbf{\arabic*.}]
		#1
	\end{enumerate}
}
\newcommand{\enumalph}[1]{
	\begin{enumerate}[label=(\alph*)]
		#1
	\end{enumerate}
}
\newcommand{\bulletize}[1]{
	\begin{itemize}[label=$\bullet$]
		#1
	\end{itemize}
}
\newcommand{\circtize}[1]{
	\begin{itemize}[label=$\circ$]
		#1
	\end{itemize}
}
\newcommand{\slide}[1]{
	\begin{frame}{\fn}
		#1
	\end{frame}
}
\newcommand{\slidec}[1]{
\begin{frame}[c]{\fn}
	#1
\end{frame}
}
\newcommand{\slidet}[2]{
	\begin{frame}{\fn\ - #1}
		#2
	\end{frame}
}


\newcommand{\startdoc}{
		\title{Math 425: Abstract Algebra 1}
		\author{Mckenzie West}
		\date{Last Updated: \today}
		\begin{frame}
			\maketitle
		\end{frame}
}

\newcommand{\topics}[2]{
	\begin{frame}{\insertframenumber}
		\begin{block}{\textbf{Last Section.}}
			\begin{itemize}[label=--]
				#1
			\end{itemize}
		\end{block}
		\begin{block}{\textbf{This Section.}}
			\begin{itemize}[label=--]
				#2
			\end{itemize}
		\end{block}
	\end{frame}
}

\begin{document}
	
\startdoc
\begin{frame}[c]{\insertframenumber}
	\begin{block}{\textbf{Previously.}}
	\begin{itemize}[label=--]
        \item Common Divisors and Common Multiples with FTA 
        \item Divisors of Factorials
        \item CRT and FTA 
        \item Prime power solution lifting
	\end{itemize}
	\end{block}
	\begin{block}{\textbf{Today.}}
		\begin{itemize}[label=--]
            \item Hensel's lemma
            \item Calculus in Number Theory
		\end{itemize}
	\end{block}
\end{frame}

\slide{
	\begin{statementblock}{Chinese Remainder Theorem (Rephrased)}
		Take distinct primes $p_1,p_2,\dots ,p_r$.
		
		Let $a,b,n\in\Z$ such that
		$\gcd(a,n)=1$, and $n$ has prime factorization $$n=p_1^{e_1}p_2^{e_2}\cdots p_r^{e_r}.$$ 
		Then for all $x\in\Z$ the congruence  $ax\equiv b\pmod n$ holds if and only if all of the following hold
		\begin{eqnarray*}
			ax&\equiv& b\pmod{p_1^{e_1}}\\
			ax&\equiv& b\pmod{p_2^{e_2}}\\
			&\vdots&\\
			ax&\equiv& b\pmod{p_r^{e_r}}.
		\end{eqnarray*}
		Moreover this $x$ is unique up to congruence modulo $n$.
	\end{statementblock}
}
\slide{
	\begin{nb}
		The ability to solve $ax\equiv b\pmod{p^r}$ will give us all the power we need. Bwahahahahaha!
	\end{nb}
	\exer{
		Solve $3x\equiv 2\pmod 5$. 
	}
}
\slide{
	\exer{
		We just found $x=4$ is a solution to $3x\equiv 2\pmod 5$. 
		\vskip 2em
		Any solution to $3x\equiv 2\pmod{25}$ reduces to a solution to $3x\equiv 2\pmod 5$.  Thus the solution to $3x\equiv 2\pmod{25}$ is of the form 
		$$x=4+5k$$
		for some $0\leq k<5$. For what $k$ is this true?
	}
}
\slide{
	\exer{
		We found that $x=9$ is a solution to $3x\equiv 2\pmod{25}$.  
		\vskip 2em
		Next we look for solutions to $3x\equiv 2\pmod{125}$. Similar to the last exercise, solutions will be of the form $$x=9+25k$$
		for some $0\leq k<5$. For what $k$ is this true?
	}
}

\slide{
	\begin{statementblock}{Hensel's Lemma}
		Let $f(x)$ be a polynomial with integer coefficients, $p$ a prime, and $e\geq 1$ an integer.

		Suppose $f(r)\equiv 0\pmod {p^e}$ and $f'(r)\neq 0\pmod p$, then 
			$$s=r+kp^{e}$$
		is a solution to $f(x)\equiv 0\pmod{p^{e+1}}$, where $k$ satisfies 
			$$\frac{f(r)}{p^e}+kf'(r)\equiv 0\pmod p.$$
	\end{statementblock}
}

\slide{
\begin{exercise}
	Apply Hensel's lemma to $f(x)=3x^2+2$ with $p=5$ to find solutions to $f(x)\equiv 0\pmod{5^2}$ and $f(x)\equiv 0\pmod{5^3}$.
\end{exercise}
}

\slide{
	\begin{exercise}
		Apply Hensel's lemma to $f(x)=x^2+1$ with $p=13$ to find solutions to $f(x)\equiv 0\pmod{13^2}$ and $f(x)\equiv 0\pmod{13^3}$.
	\end{exercise}
}
\slide{
	\begin{exercise}
		Explore Hensel's Lemma with respect to $f(x)=x^3+1$ modulo 3.
	\end{exercise}
}
\begin{frame}[fragile]{\fn}
	\begin{lstlisting}
sage: p = 3
sage: R.<x> = PolynomialRing(Integers(p))
sage: f = x^3+1
	
sage: for i in range(1,7):
sage: 	T.<z> = PolynomialRing(Integers(p^i))
sage: 	print(T(f).roots(multiplicities=False))	\end{lstlisting}

\begin{lstlisting}
[1]
[2, 5, 8]
[8, 17, 26]
[26, 53, 80]
[80, 161, 242]
[242, 485, 728]
\end{lstlisting}
\end{frame}
\slide{
	\begin{exercise}
		Explore Hensel's Lemma with respect to $f(x)=2x^2+x+2$ modulo 5.
	\end{exercise}
}
\begin{frame}[fragile]{\fn}
\begin{lstlisting}
sage: p = 3
sage: R.<x> = PolynomialRing(Integers(p))
sage: f = 2*x^2+x+2

sage: for i in [1,2]:
sage: 	T.<z> = PolynomialRing(Integers(p^i))
sage: 	print(T(f).roots(multiplicities=False))	\end{lstlisting}

\begin{lstlisting}
[1]
[]
\end{lstlisting}
\end{frame}
\slide{
	\begin{exercise}
		Explore Hensel's Lemma with respect to $f(x)=2x^4 + 2x^3 + x^2 + 2x + 2$ modulo $3$.
	\end{exercise}
}
\begin{frame}[fragile]{\fn}
\begin{lstlisting}
	sage: p = 3
	sage: R.<x> = PolynomialRing(Integers(p))
	sage: 2*x^4 + 2*x^3 + x^2 + 2*x + 2
	
	sage: for i in range(1,4):
	sage: 	T.<z> = PolynomialRing(Integers(p^i))
	sage: 	print(T(f).roots(multiplicities=False))	\end{lstlisting}

\begin{lstlisting}
	[1]
	[1, 4, 7]
	[]
\end{lstlisting}
\end{frame}
\slide{
\begin{exercise}
	Explore Hensel's Lemma with respect to $f(x)=x^2+2x+1$ with $p=3$.
\end{exercise}
}
\begin{frame}[fragile]{\fn}
\begin{lstlisting}
	sage: p = 3
	sage: R.<x> = PolynomialRing(Integers(p))
	sage: x^2+2*x+1
	
	sage: for i in range(1,4):
	sage: 	T.<z> = PolynomialRing(Integers(p^i))
	sage: 	print(T(f).roots(multiplicities=False))	\end{lstlisting}

\begin{lstlisting}
	[2]
	[2, 5, 8]
	[8, 17, 26]
	[8, 17, 26, 35, 44, 53, 62, 71, 80]
\end{lstlisting}
\end{frame}
\end{document}

%p = 3
%d = 4
%
%R.<x> = PolynomialRing(Integers(p))
%S.<t> = PolynomialRing(Integers(p^2))
%T.<z> = PolynomialRing(Integers(p^3))
%f = R.random_element(degree=d)
%while len(f.roots())<1 or len(S(f).roots(multiplicities=False))<1 or len(T(f).roots(multiplicities=False))>0:
%f = R.random_element(degree=d)
%print(f)
%print(f.roots())
%print(S(f).roots(multiplicities=False))
%print(T(f).roots(multiplicities=False))
%
