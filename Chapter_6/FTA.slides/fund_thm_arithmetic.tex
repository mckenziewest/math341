\documentclass[t]{beamer}

\subtitle{Fundamental Theorem of Arithmetic}

\usepackage{amsthm,amsmath,amsfonts,hyperref,graphicx,color,multicol,soul}
\usepackage{enumitem,tikz,tikz-cd,setspace,mathtools}

%%%%%%%%%%
%Beamer Template Customization
%%%%%%%%%%
\setbeamertemplate{navigation symbols}{}
\setbeamertemplate{theorems}[ams style]
\setbeamertemplate{blocks}[rounded]

\definecolor{Blu}{RGB}{43,62,133} % UWEC Blue
\setbeamercolor{structure}{fg=Blu} % Titles

%Unnumbered footnotes:
\newcommand{\blfootnote}[1]{%
	\begingroup
	\renewcommand\thefootnote{}\footnote{#1}%
	\addtocounter{footnote}{-1}%
	\endgroup
}

%%%%%%%%%%
%TikZ Stuff
%%%%%%%%%%
\usetikzlibrary{arrows}
\usetikzlibrary{shapes.geometric}
\tikzset{
	smaller/.style={
		draw,
		regular polygon,
		regular polygon sides=3,
		fill=white,
		node distance=2cm,
		minimum height=1in,
		line width = 2pt
	}
}
\tikzset{
	smsquare/.style={
		draw,
		regular polygon,
		regular polygon sides=4,
		fill=white,
		node distance=2cm,
		minimum height=1in,
		line width = 2pt
	}
}


%%%%%%%%%%
%Custom Commands
%%%%%%%%%%

\newcommand{\C}{\mathbb{C}}
\newcommand{\quats}{\mathbb{H}}
\newcommand{\N}{\mathbb{N}}
\newcommand{\Q}{\mathbb{Q}}
\newcommand{\R}{\mathbb{R}}
\newcommand{\Z}{\mathbb{Z}}

\newcommand{\ds}{\displaystyle}

\newcommand{\fn}{\insertframenumber}

\newcommand{\id}{\operatorname{id}}
\newcommand{\im}{\operatorname{im}}
\newcommand{\lcm}{\operatorname{lcm}}
\newcommand{\Aut}{\operatorname{Aut}}
\newcommand{\Inn}{\operatorname{Inn}}

\newcommand{\blank}[1]{\underline{\hspace*{#1}}}

\newcommand{\abar}{\overline{a}}
\newcommand{\bbar}{\overline{b}}
\newcommand{\cbar}{\overline{c}}

\newcommand{\nml}{\unlhd}

%%%%%%%%%%
%Custom Theorem Environments
%%%%%%%%%%
\theoremstyle{definition}
\newtheorem{exercise}{Exercise}
\newtheorem{question}[exercise]{Question}
\newtheorem{warmup}{Warm-Up}
\newtheorem*{exa}{Example}
\newtheorem*{disc}{Group Discussion}
\newtheorem*{recall}{Recall}
\renewcommand{\emph}[1]{{\color{blue}\texttt{#1}}}

\definecolor{Gold}{RGB}{237, 172, 26}
%Statement block
\newenvironment{statementblock}[1]{%
	\setbeamercolor{block body}{bg=Gold!20}
	\setbeamercolor{block title}{bg=Gold}
	\begin{block}{\textbf{#1.}}}{\end{block}}
\newenvironment{goldblock}{%
	\setbeamercolor{block body}{bg=Gold!20}
	\setbeamercolor{block title}{bg=Gold}
	\setbeamertemplate{blocks}[shadow=true]
	\begin{block}{}}{\end{block}}
\newenvironment{defn}{%
	\setbeamercolor{block body}{bg=gray!20}
	\setbeamercolor{block title}{bg=violet, fg=white}
	\setbeamertemplate{blocks}[shadow=true]
	\begin{block}{\textbf{Definition.}}}{\end{block}}
\newenvironment{nb}{%
	\setbeamercolor{block body}{bg=gray!20}
	\setbeamercolor{block title}{bg=teal, fg=white}
	\setbeamertemplate{blocks}[shadow=true]
	\begin{block}{\textbf{Note.}}}{\end{block}}
\newenvironment{blockexample}{%
	\setbeamercolor{block body}{bg=gray!20}
	\setbeamercolor{block title}{bg=Blu, fg=white}
	\setbeamertemplate{blocks}[shadow=true]
	\begin{block}{\textbf{Example.}}}{\end{block}}
\newenvironment{blocknonexample}{%
	\setbeamercolor{block body}{bg=gray!20}
	\setbeamercolor{block title}{bg=purple, fg=white}
	\setbeamertemplate{blocks}[shadow=true]
	\begin{block}{\textbf{Non-Example.}}}{\end{block}}
\newenvironment{thm}[1]{%
	\setbeamercolor{block body}{bg=Gold!20}
	\setbeamercolor{block title}{bg=Gold}
	\begin{block}{\textbf{Theorem #1.}}}{\end{block}}


%%%%%%%%%%
%Custom Environment Wrappers
%%%%%%%%%%
\newcommand{\exer}[1]{
	\begin{exercise}
		#1
	\end{exercise}
}
\newcommand{\exam}[1]{
\begin{blockexample}
	#1
\end{blockexample}
}
\newcommand{\nexam}[1]{
\begin{blocknonexample}
	#1
\end{blocknonexample}
}
\newcommand{\enumarabic}[1]{
	\begin{enumerate}[label=\textbf{\arabic*.}]
		#1
	\end{enumerate}
}
\newcommand{\enumalph}[1]{
	\begin{enumerate}[label=(\alph*)]
		#1
	\end{enumerate}
}
\newcommand{\bulletize}[1]{
	\begin{itemize}[label=$\bullet$]
		#1
	\end{itemize}
}
\newcommand{\circtize}[1]{
	\begin{itemize}[label=$\circ$]
		#1
	\end{itemize}
}
\newcommand{\slide}[1]{
	\begin{frame}{\fn}
		#1
	\end{frame}
}
\newcommand{\slidec}[1]{
\begin{frame}[c]{\fn}
	#1
\end{frame}
}
\newcommand{\slidet}[2]{
	\begin{frame}{\fn\ - #1}
		#2
	\end{frame}
}


\newcommand{\startdoc}{
		\title{Math 341: Classical Number Theory}
		\author{Mckenzie West}
		\date{Last Updated: \today}
		\begin{frame}
			\maketitle
		\end{frame}
}

\newcommand{\topics}[2]{
	\begin{frame}[c]{\insertframenumber}
		\begin{block}{\textbf{Last Section.}}
			\begin{itemize}[label=--]
				#1
			\end{itemize}
		\end{block}
		\begin{block}{\textbf{This Section.}}
			\begin{itemize}[label=--]
				#2
			\end{itemize}
		\end{block}
	\end{frame}
}

\begin{document}
	
\startdoc
\begin{frame}[c]{\insertframenumber}
	\begin{block}{\textbf{Previously.}}
	\begin{itemize}[label=--]
		\item Jupyter Exploration of Primes
	\end{itemize}
	\end{block}
	\begin{block}{\textbf{Today.}}
		\begin{itemize}[label=--]
			\item Prime factorization
			\item The Fundamental Theorem of Arithmetic
		\end{itemize}
	\end{block}
\end{frame}

\slide{
	\begin{defn}
		A \emph{prime} number is a positive integer $p>1$ whose only divisors are $1$ and $p$ itself.  
		
		Integers greater than 1 that are not prime are called \emph{composite}.
	\end{defn}
	\begin{exa}
		\enumarabic{
			\item Primes: 2, 3, 5, 7, 11, 13, 17, $\dots$
			\item Composites: 4, 6, 8, 9, 10, 12, 14, 15, 16, $\dots$
		}
	\end{exa}
}
\slide{
	\begin{statementblock}{Fundamental Theorem of Arithmetic}
		Ever positive integer $n>1$ is either a prime or a product of primes.  Moreover, this representation as a product of primes is unique up to the ordering.
	\end{statementblock}
	\begin{block}{\textbf{Goal.}}
		We're going to prove this.  But first a few other results that will help us.
	\end{block}
}
\slide{
	\begin{statementblock}{Theorem 3.1}
		If $p$ is prime and $p|ab$, then $p|a$ or $p|b$. (Here $a,b\in\Z$.)
	\end{statementblock}
}
\slide{
	\begin{statementblock}{Corollary 1}
		If $p$ is prime and $p|a_1a_2\cdots a_n$ for $a_1,a_2,\dots,a_n\in\Z$ then $p|a_k$ for some $k$.
	\end{statementblock}
}
\slide{
	\begin{statementblock}{Corollary 2}
		If $p,q_1,q_2,\dots,q_n$ are all primes and $p|q_1q_2\cdots q_n$, then $p=q_k$ for some $k$.
	\end{statementblock}
}
\slide{
	\begin{proof}[Proof of the Fundamental Theorem of Arithmetic]
		We use induction on $n$.\vskip 3in
	\end{proof}
}
\slide{
	\begin{defn}
		The \emph{canonical form} for an integer with respect to its prime factorization is
			\[n=p_1^{k_1}p_2^{k_2}\cdots p_r^{k_r}\]
		where $1<p_1<p_2<\cdots<p_r$ are the unique prime factors and each $k_i$ is a positive integer.
	\end{defn}
}
\slide{
	\begin{statementblock}{Theorem (Euclid)}
		There are infinitely many primes.
	\end{statementblock}
}
\slide{
	\begin{statementblock}{Corollary}
		If $p_n$ denotes the $n$th prime, then for all $n\geq 2$,
			\[p_n\leq p_1p_2\cdots p_{n-1}+1.\]
	\end{statementblock}
}
\slide{
	\begin{statementblock}{Bonse's Inequality}
		For all $n\geq 5$,
		\[p_n^2<p_1p_2\cdots p_{n-1}.\]
	\end{statementblock}
	\begin{statementblock}{Un-Named Inequality}
		For all $n\geq 3$,
		\[p_{2n}\leq p_2p_3\cdots p_n-2.\]
	\end{statementblock}
}
\slide{
	\begin{statementblock}{Theorem 3.5}
		If $p_n$ is the $n$th prime number then $\ds p_n\leq 2^{2^{n-1}}$.
	\end{statementblock}
	\begin{statementblock}{Corollary}
		For all $n\geq 1$ there are at least $n+1$ primes less than $2^{2^n}$.
	\end{statementblock}
}
\slide{
	\begin{question}
		So what numbers are prime??
	\end{question}
	\begin{exa}
		Here are the primes less than 100:
		\begin{center}
			\begin{tabular}{|c|c|c|c|c|}
				\hline
				2&3&5&7&11\\\hline
				13&17&19&23&29\\\hline
				31&37&41&43&47\\\hline
				53&59&61&67&71\\\hline
				73&79&83&89&97\\\hline
			\end{tabular}
		\end{center}
		Thus 25\% of all primes less than 100 are prime.
	\end{exa}
}
\slide{
	\begin{statementblock}{The Sieve of Eratosthenes}
		$$\begin{array}{cccccccccc}
			&2&3&4&5&6&7&8&9&10\\
			11&12&13&14&15&16&17&18&19&20\\
			21&22&23&24&25&26&27&28&29&30\\
			31&32&33&34&35&36&37&38&39&40\\
			41&42&43&44&45&46&47&48&49&50\\
			51&52&53&54&55&56&57&58&59&60\\
			61&62&63&64&65&66&67&68&69&70\\
			71&72&73&74&75&76&77&78&79&80\\
			81&82&83&84&85&86&87&88&89&90\\
			91&92&93&94&95&96&97&98&99&100
		\end{array}$$
	\end{statementblock}
}
\slide{
	\begin{defn}
		The \emph{prime counting function} is the function $\pi:\N\to\Z$ defined by $$\pi(n)=\#\{p\in\N\ |\ p\leq n\text{ and }p\text{ is prime}\}$$
	\end{defn}
}

\slide{
	\begin{statementblock}{Theorem 3.6}
		There are infinitely many primes of the form $p=4k+3$.
	\end{statementblock}
%	\begin{proof}[Proof Idea.]
%		Proof follows a very similar formula to the general infinitude of primes. 
%		\bulletize{
%			\item Let $3=p_1,p_2,p_3,\dots,p_n$ be primes of this form.
%			\item Consider $a=4p_1p_2\cdots p_n+3$.
%			\item There will be at least one prime of the form $4k+3$ that divides $a$ because primes of the form $4k+1$ can never multiply to get an integer of the form $4k+3$.
%		}
%	\end{proof}
%	\begin{nb}
%		This proof does not work show the result for primes of the form $4k+1$.
%	\end{nb}
}
\slide{
	\begin{statementblock}{Dirichlet's Theorem}
		If $a,b>1$ and $\gcd(a,b)=1$, then there are infinitely many primes of the form $p=a+bk$.  That is, there are infinitely many primes in the arithmetic progression
		\[a+b,a+2b,a+3b,a+4b,a+5b,\dots.\]
	\end{statementblock}
	\begin{exercise}
		Given $a=1$ and $b=10$, what does this mean for possible forms of prime numbers?  Similarly, what about $a=3,7,9$?
	\end{exercise}\vskip .5in
	\begin{exercise}
		What goes wrong if $\gcd(a,b)>1$?
	\end{exercise}\vskip 1in\mbox{}
}
\slide{
	\begin{question}
		How does Dirichlet's Theorem work when we use $a=p$ prime?
	\end{question}\vskip 1in
	\begin{exercise}
		Verify that if $a=47$ and $b=6$, the first 3 values in the progression are \textit{consecutive} primes.  
	\end{exercise}\vskip 1in\mbox{}
}
\slide{
	\begin{statementblock}{Conjecture (Hardy--Littlewood)}
		There are infinitely many values of $n$ for which $n^2+1$ is prime.
		
		See: \url{http://oeis.org/A002496}
	\end{statementblock}
	\begin{statementblock}{Theorem (Iwaniec, 1978)}
		There are infinitely many values of $n$ for which $n^2+1$ is either a prime or the product of two primes.
		
	\end{statementblock}
	\begin{nb}
		The paper proving this result was titled \textbf{Almost-primes represented by quadratic polynomials} and it was published in \textit{Inventiones mathematicae}, one of the most prestigious mathematics journals.
	\end{nb}
}
\end{document}

