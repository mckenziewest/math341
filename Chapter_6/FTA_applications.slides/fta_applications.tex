\documentclass[t]{beamer}

\subtitle{Applications of the Fundamental Theorem of Arithmetic}

\usepackage{amsthm,amsmath,amsfonts,hyperref,graphicx,color,multicol,soul}
\usepackage{enumitem,tikz,tikz-cd,setspace,mathtools}

%%%%%%%%%%
%Beamer Template Customization
%%%%%%%%%%
\setbeamertemplate{navigation symbols}{}
\setbeamertemplate{theorems}[ams style]
\setbeamertemplate{blocks}[rounded]

\definecolor{Blu}{RGB}{43,62,133} % UWEC Blue
\setbeamercolor{structure}{fg=Blu} % Titles

%Unnumbered footnotes:
\newcommand{\blfootnote}[1]{%
	\begingroup
	\renewcommand\thefootnote{}\footnote{#1}%
	\addtocounter{footnote}{-1}%
	\endgroup
}

%%%%%%%%%%
%TikZ Stuff
%%%%%%%%%%
\usetikzlibrary{arrows}
\usetikzlibrary{shapes.geometric}
\tikzset{
	smaller/.style={
		draw,
		regular polygon,
		regular polygon sides=3,
		fill=white,
		node distance=2cm,
		minimum height=1in,
		line width = 2pt
	}
}
\tikzset{
	smsquare/.style={
		draw,
		regular polygon,
		regular polygon sides=4,
		fill=white,
		node distance=2cm,
		minimum height=1in,
		line width = 2pt
	}
}


%%%%%%%%%%
%Custom Commands
%%%%%%%%%%

\newcommand{\C}{\mathbb{C}}
\newcommand{\quats}{\mathbb{H}}
\newcommand{\N}{\mathbb{N}}
\newcommand{\Q}{\mathbb{Q}}
\newcommand{\R}{\mathbb{R}}
\newcommand{\Z}{\mathbb{Z}}

\newcommand{\ds}{\displaystyle}

\newcommand{\fn}{\insertframenumber}

\newcommand{\id}{\operatorname{id}}
\newcommand{\im}{\operatorname{im}}
\newcommand{\lcm}{\operatorname{lcm}}
\newcommand{\Aut}{\operatorname{Aut}}
\newcommand{\Inn}{\operatorname{Inn}}

\newcommand{\blank}[1]{\underline{\hspace*{#1}}}

\newcommand{\abar}{\overline{a}}
\newcommand{\bbar}{\overline{b}}
\newcommand{\cbar}{\overline{c}}

\newcommand{\nml}{\unlhd}

%%%%%%%%%%
%Custom Theorem Environments
%%%%%%%%%%
\theoremstyle{definition}
\newtheorem{exercise}{Exercise}
\newtheorem{question}[exercise]{Question}
\newtheorem{warmup}{Warm-Up}
\newtheorem*{exa}{Example}
\newtheorem*{disc}{Group Discussion}
\newtheorem*{recall}{Recall}
\renewcommand{\emph}[1]{{\color{blue}\texttt{#1}}}

\definecolor{Gold}{RGB}{237, 172, 26}
%Statement block
\newenvironment{statementblock}[1]{%
	\setbeamercolor{block body}{bg=Gold!20}
	\setbeamercolor{block title}{bg=Gold}
	\begin{block}{\textbf{#1.}}}{\end{block}}
\newenvironment{goldblock}{%
	\setbeamercolor{block body}{bg=Gold!20}
	\setbeamercolor{block title}{bg=Gold}
	\setbeamertemplate{blocks}[shadow=true]
	\begin{block}{}}{\end{block}}
\newenvironment{defn}{%
	\setbeamercolor{block body}{bg=gray!20}
	\setbeamercolor{block title}{bg=violet, fg=white}
	\setbeamertemplate{blocks}[shadow=true]
	\begin{block}{\textbf{Definition.}}}{\end{block}}
\newenvironment{nb}{%
	\setbeamercolor{block body}{bg=gray!20}
	\setbeamercolor{block title}{bg=teal, fg=white}
	\setbeamertemplate{blocks}[shadow=true]
	\begin{block}{\textbf{Note.}}}{\end{block}}
\newenvironment{blockexample}{%
	\setbeamercolor{block body}{bg=gray!20}
	\setbeamercolor{block title}{bg=Blu, fg=white}
	\setbeamertemplate{blocks}[shadow=true]
	\begin{block}{\textbf{Example.}}}{\end{block}}
\newenvironment{blocknonexample}{%
	\setbeamercolor{block body}{bg=gray!20}
	\setbeamercolor{block title}{bg=purple, fg=white}
	\setbeamertemplate{blocks}[shadow=true]
	\begin{block}{\textbf{Non-Example.}}}{\end{block}}
\newenvironment{thm}[1]{%
	\setbeamercolor{block body}{bg=Gold!20}
	\setbeamercolor{block title}{bg=Gold}
	\begin{block}{\textbf{Theorem #1.}}}{\end{block}}


%%%%%%%%%%
%Custom Environment Wrappers
%%%%%%%%%%
\newcommand{\exer}[1]{
	\begin{exercise}
		#1
	\end{exercise}
}
\newcommand{\exam}[1]{
\begin{blockexample}
	#1
\end{blockexample}
}
\newcommand{\nexam}[1]{
\begin{blocknonexample}
	#1
\end{blocknonexample}
}
\newcommand{\enumarabic}[1]{
	\begin{enumerate}[label=\textbf{\arabic*.}]
		#1
	\end{enumerate}
}
\newcommand{\enumalph}[1]{
	\begin{enumerate}[label=(\alph*)]
		#1
	\end{enumerate}
}
\newcommand{\bulletize}[1]{
	\begin{itemize}[label=$\bullet$]
		#1
	\end{itemize}
}
\newcommand{\circtize}[1]{
	\begin{itemize}[label=$\circ$]
		#1
	\end{itemize}
}
\newcommand{\slide}[1]{
	\begin{frame}{\fn}
		#1
	\end{frame}
}
\newcommand{\slidec}[1]{
\begin{frame}[c]{\fn}
	#1
\end{frame}
}
\newcommand{\slidet}[2]{
	\begin{frame}{\fn\ - #1}
		#2
	\end{frame}
}


\newcommand{\startdoc}{
		\title{Math 341: Classical Number Theory}
		\author{Mckenzie West}
		\date{Last Updated: \today}
		\begin{frame}
			\maketitle
		\end{frame}
}

\newcommand{\topics}[2]{
	\begin{frame}[c]{\insertframenumber}
		\begin{block}{\textbf{Last Section.}}
			\begin{itemize}[label=--]
				#1
			\end{itemize}
		\end{block}
		\begin{block}{\textbf{This Section.}}
			\begin{itemize}[label=--]
				#2
			\end{itemize}
		\end{block}
	\end{frame}
}

\begin{document}
	
\startdoc
\begin{frame}[c]{\insertframenumber}
	\begin{block}{\textbf{Previously.}}
	\begin{itemize}[label=--]
        \item Prime factorization
        \item The Fundamental Theorem of Arithmetic
	\end{itemize}
	\end{block}
	\begin{block}{\textbf{Today.}}
		\begin{itemize}[label=--]
            \item Common Divisors and Common Multiples with FTA 
            \item Divisors of Factorials
            \item CRT and FTA 
            \item Prime power solution lifting
		\end{itemize}
	\end{block}
\end{frame}

\slide{
	\begin{statementblock}{Fundamental Theorem of Arithmetic}
		Ever positive integer $n>1$ is either a prime or a product of primes.  Moreover, this representation as a product of primes is unique up to the ordering.
	\end{statementblock}
	\begin{defn}
		The \emph{canonical form} for an integer with respect to its prime factorization is
			\[n=p_1^{k_1}p_2^{k_2}\cdots p_r^{k_r}\]
		where $1<p_1<p_2<\cdots<p_r$ are the unique prime factors and each $k_i$ is a positive integer.
	\end{defn}
}
\slide{
    \begin{exercise}
        Compute the prime factorizations of
    \end{exercise}
    (a) $a=1789348484 $\vfill
    (b) $b=798932750$\vfill 
    (c) $\gcd(a,b)$\vfill
    (d) $\lcm(a,b)$\vfill
}
\slide{
    \begin{thm}{}
        Let $a,b\in\Z$ with 
        $$a=(-1)^{e_0}p_1^{e_1}p_2^{e_2}\cdots p_r^{e_r}\quad\text{and}\quad 
        b=(-1)^{f_0}p_1^{f_1}p_2^{f_2}\cdots p_r^{f_r}$$
        are factorizations of $a$ and $b$ where $p_1,\dots,p_r$ are distinct primes and $e_i,f_j$ are non-negative integers.
        Then 
        $$\gcd(a,b)=p_1^{\min(e_1,f_1)}p_2^{\min(e_2,f_2)}\cdots p_r^{\min(e_r,f_r)},$$
        and
        $$\lcm(a,b)=p_1^{\max(e_1,f_1)}p_2^{\max(e_2,f_2)}\cdots p_r^{\max(e_r,f_r)}.$$
    \end{thm}
}
\slide{
    \begin{exercise}
        Let $d=\gcd(a,b)$. Let $1\leq i\leq r$ show that $p_i^{\min(e_i,f_i)}$ divides $d$.
    \end{exercise}
    \vfill
    \begin{exercise}
        Show that $p_i^{\min(e_i,f_i)+1}$ does not divide $d$.
    \end{exercise}
    \vfill
}
\slide{
    \begin{exercise}
        Show that if $q$ is prime and $q\neq p_i$ for all $i$, then $q\nmid d$.
    \end{exercise}
    \vfill
    \begin{exercise}
        Explain why $d=p_1^{\min(e_1,f_1)}p_2^{\min(e_2,f_2)}\cdots p_r^{\min(e_r,f_r)}$.
    \end{exercise}
    \vfill
}
\slide{
    \exer{
        Compute the prime factorizations:
        $$\begin{array}{|l|l|}
            \hline 
            n&\text{prime factorization of }n\\
            \hline 
            &\\
            4!&\\&\\
            5!&\\&\\
            6!&\\&\\
            7!&\\&\\
            8!&\\&\\
            \hline
        \end{array}$$
    }
}
\slide{
    \begin{defn}
        For a prime $p$ and an integer $n$ define $p^r\| n$ to mean $p^r$ divides $n$ but $p^{r+1}\nmid n$.

        Define the \emph{valuation} of $n$ at $p$ by $v_p(n)=r$ where $p^r\| n$.
    \end{defn}
    \exer{
        What is $v_2(3!)$, $v_2(4!)$, $v_2(5!)$?
    }
}
\slide{
    \exer{
        For a positive integer $n$, let $f_k(n)$ be the number of positive integers $m<n$ such that $k$ divides $m$.
        Find a formula for  $f_2(n)$, $f_4(n)$, $f_8(n)$. 
    }
    \vfill 
    \exer{
        Conjecture a formula for $v_2(n!)$ in terms of $f_2(n)$, $f_4(n)$, $f_8(n)$, etc.
    }
}
\slide{
    Recall from the CRT:

    Let $n_1,\dots, n_r$ be positive integers and $a_1,\dots,a_r$ be integers with $\gcd(a_i,n_i)=1$ for all $i$.
    Then for any $b_1,\dots,b_r\in \Z$, a solution to a system of congruences:
    $$a_1x\equiv b_1\!\pmod{n_1},\dots
    ,a_rx\equiv b_r\!\pmod{n_r}$$
    has a single solution modulo $n_1\cdots n_r$ if the $n_1,\dots,n_r$ are coprime.
    \vskip .25in
    Moreover, one thing we talked about was that if $N=mn$ with $\gcd(m,n)=1$, then
    a solution to $ax\equiv b\pmod N$ can be found using
    $$ax\equiv b\pmod m,\text{ and }ax\equiv b\pmod n.$$
}
\slide{
    \exer{
        Discuss the implications of this idea alongside the Fundamental Theorem of Arithmetic.
    }
}
\slide{
    \begin{nb}
        The ability to solve $ax\equiv b\pmod{p^r}$ will give us all the power we need. Bwahahahahaha!
    \end{nb}
    \exer{
        Solve $3x\equiv 2\pmod 5$. 
    }
}
\slide{
    \exer{
        We just found $x=4$ is a solution to $3x\equiv 2\pmod 5$. 
        \vskip 2em
        Any solution to $3x\equiv 2\pmod{25}$ reduces to a solution to $3x\equiv 2\pmod 5$.  Thus the solution to $3x\equiv 2\pmod{25}$ is of the form 
        $$x=4+5k$$
        for some $0\leq k<5$. For what $k$ is this true?
    }
}
\slide{
    \exer{
        We found that $x=9$ is a solution to $3x\equiv 2\pmod{25}$.  
        \vskip 2em
        Next we look for solutions to $3x\equiv 2\pmod{125}$. Similar to the last exercise, solutions will be of the form $$x=9+25k$$
        for some $0\leq k<5$. For what $k$ is this true?
    }
}

\end{document}

