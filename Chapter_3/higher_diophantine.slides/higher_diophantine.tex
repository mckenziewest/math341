\documentclass[t]{beamer}

\subtitle{Higher Order Diophantine Equations}

\usepackage{amsthm,amsmath,amsfonts,hyperref,graphicx,color,multicol,soul}
\usepackage{enumitem,tikz,tikz-cd,setspace,mathtools}

%%%%%%%%%%
%Beamer Template Customization
%%%%%%%%%%
\setbeamertemplate{navigation symbols}{}
\setbeamertemplate{theorems}[ams style]
\setbeamertemplate{blocks}[rounded]

\definecolor{Blu}{RGB}{43,62,133} % UWEC Blue
\setbeamercolor{structure}{fg=Blu} % Titles

%Unnumbered footnotes:
\newcommand{\blfootnote}[1]{%
	\begingroup
	\renewcommand\thefootnote{}\footnote{#1}%
	\addtocounter{footnote}{-1}%
	\endgroup
}

%%%%%%%%%%
%TikZ Stuff
%%%%%%%%%%
\usetikzlibrary{arrows}
\usetikzlibrary{shapes.geometric}
\tikzset{
	smaller/.style={
		draw,
		regular polygon,
		regular polygon sides=3,
		fill=white,
		node distance=2cm,
		minimum height=1in,
		line width = 2pt
	}
}
\tikzset{
	smsquare/.style={
		draw,
		regular polygon,
		regular polygon sides=4,
		fill=white,
		node distance=2cm,
		minimum height=1in,
		line width = 2pt
	}
}


%%%%%%%%%%
%Custom Commands
%%%%%%%%%%

\newcommand{\C}{\mathbb{C}}
\newcommand{\quats}{\mathbb{H}}
\newcommand{\N}{\mathbb{N}}
\newcommand{\Q}{\mathbb{Q}}
\newcommand{\R}{\mathbb{R}}
\newcommand{\Z}{\mathbb{Z}}

\newcommand{\ds}{\displaystyle}

\newcommand{\fn}{\insertframenumber}

\newcommand{\id}{\operatorname{id}}
\newcommand{\im}{\operatorname{im}}
\newcommand{\Aut}{\operatorname{Aut}}
\newcommand{\Inn}{\operatorname{Inn}}

\newcommand{\blank}[1]{\underline{\hspace*{#1}}}

\newcommand{\abar}{\overline{a}}
\newcommand{\bbar}{\overline{b}}
\newcommand{\cbar}{\overline{c}}

\newcommand{\nml}{\unlhd}

%%%%%%%%%%
%Custom Theorem Environments
%%%%%%%%%%
\theoremstyle{definition}
\newtheorem{exercise}{Exercise}
\newtheorem{question}[exercise]{Question}
\newtheorem{warmup}{Warm-Up}
\newtheorem*{defn}{Definition}
\newtheorem*{exa}{Example}
\newtheorem*{disc}{Group Discussion}
\newtheorem*{nb}{Note}
\newtheorem*{recall}{Recall}
\renewcommand{\emph}[1]{{\color{blue}\texttt{#1}}}

\definecolor{Gold}{RGB}{237, 172, 26}
%Statement block
\newenvironment{statementblock}[1]{%
	\setbeamercolor{block body}{bg=Gold!20}
	\setbeamercolor{block title}{bg=Gold}
	\begin{block}{\textbf{#1.}}}{\end{block}}
\newenvironment{thm}[1]{%
	\setbeamercolor{block body}{bg=Gold!20}
	\setbeamercolor{block title}{bg=Gold}
	\begin{block}{\textbf{Theorem #1.}}}{\end{block}}


%%%%%%%%%%
%Custom Environment Wrappers
%%%%%%%%%%
\newcommand{\enumarabic}[1]{
	\begin{enumerate}[label=\textbf{\arabic*.}]
		#1
	\end{enumerate}
}
\newcommand{\enumalph}[1]{
	\begin{enumerate}[label=(\alph*)]
		#1
	\end{enumerate}
}
\newcommand{\bulletize}[1]{
	\begin{itemize}[label=$\bullet$]
		#1
	\end{itemize}
}
\newcommand{\circtize}[1]{
	\begin{itemize}[label=$\circ$]
		#1
	\end{itemize}
}
\newcommand{\slide}[1]{
	\begin{frame}{\fn}
		#1
	\end{frame}
}
\newcommand{\slidec}[1]{
\begin{frame}[c]{\fn}
	#1
\end{frame}
}
\newcommand{\slidet}[2]{
	\begin{frame}{\fn\ - #1}
		#2
	\end{frame}
}


\newcommand{\startdoc}{
		\title{Math 425: Abstract Algebra 1}
		\author{Mckenzie West}
		\date{Last Updated: \today}
		\begin{frame}
			\maketitle
		\end{frame}
}

\newcommand{\topics}[2]{
	\begin{frame}{\insertframenumber}
		\begin{block}{\textbf{Last Section.}}
			\begin{itemize}[label=--]
				#1
			\end{itemize}
		\end{block}
		\begin{block}{\textbf{This Section.}}
			\begin{itemize}[label=--]
				#2
			\end{itemize}
		\end{block}
	\end{frame}
}
\usepackage{pgfplots}

\begin{document} 
	\startdoc
	
	\topics{
		% Last Time
		\item Diophantine Equations of the form $ax+by=c$
		\item Solving Diophantine Equations
	}
	{
		% This time
		\item Elliptic Curves
		\item Pythagorean Triples
		\item Fermat's Last Theorem
	}

\slide{
	\begin{defn}
		An \emph{elliptic curve} is a curve given by an equation of the form $$y^2=x^3+ax+b$$ with the requirement that $4a^3+27b^2\neq 0$.
	\end{defn}
	\begin{nb}
		The requirement of $4a^3+27b^2\neq 0$ results in:
		\bulletize{
			\item Geometrically: No cusps or self-intersections
			\item Algebraically: No repeated roots of $x^3+ax+b$
		}
	\end{nb}
}
\slide{
	\begin{defn}
		A \emph{Mordell equation} is an elliptic curve specifically of the form $y^2=x^3+k$ for $k\in\Z$.
	\end{defn}
	\begin{thm}{(Mordell)}
		There are finitely many pairs $x,y\in\Z$ that satisfy any given Mordell equation.
	\end{thm}
	\begin{exa}
		Some $k$ values with no solutions: $7,-5,11,-6,45,6,46,-24$
		
		Some $k$ values with known solutions:
		\begin{center}
			\begin{tabular}{l|l}
				$k$&Pairs\\\hline
				$16$&$(0,4),(0,-4)$\\
				$-1$&$(1,0)$\\
				$-4$&$(2,2),(2,-2),(5,11),(5,-11)$\\
				$-2$&$(3,5),(3,-5)$
			\end{tabular}
		\end{center}
	\end{exa}
}
\slidec{
	\begin{nb}
		All of these values were proven by means specific to $k$.  The question remains whether there's a quick way to provably find all integral solutions.
	\end{nb}
	\begin{exa}
		On this site you will find a table containing the number of \textit{known} solutions to the Mordell equation for the given $k$, with $x\leq 10^{10}$
		\url{https://hr.userweb.mwn.de/numb/mordell.html}
	\end{exa}
}
\slide{
	\begin{statementblock}{Catalan Conjecture, Mih\u{a}ilescu's Theorem }
		If $y^m=x^n+1$  for $x,y$ positive integers and $m,n>1$ then either $xy=0$ or the equation is of the form $3^2=2^3+1$.
		\vskip 2em
		That is the only integral solutions to the Mordell equation with $k=1$ are $(2,\pm 3), (0,\pm 1)$, $(-1,0)$.
	\end{statementblock}
}
\slide{
	\begin{defn}
		A \emph{Pythagorean triple} is a set of three integers $(x,y,z)$ such that $x^2+y^2=z^2$.  The triple is \emph{primitive} if $\gcd(x,y,z)=1$.
	\end{defn}
}
\slide{
	\begin{statementblock}{Lemma}
		If $(x,y,z)$ is a Pythagorean triple, then one of $x$ or $y$ is even while the other is odd.
	\end{statementblock}
}
\slide{
	\begin{thm}{12.1}
		All of the primitive Pythagorean triples $(x,y,z)$ such that
		\enumalph{
			\item $x$ is even
			\item $x,y,z>0$
		}
		can be described by
			$$x=2st\quad y=s^2-t^2\quad z=s^2+t^2,$$
		where $s>t>0$, $\gcd(s,t)=1$, and one of $s$ or $t$ is even.
	\end{thm}
}
\slide{
	\begin{statementblock}{Theorem}
		If $(x,y,z)$ is a primitive Pythagorean triple, then both $x+y$ and $x-y$ are 1 more or 1 less than a multiple of 8.
	\end{statementblock}
}
\slide{
	\begin{thm}{12.3}
		The Diophantine equation $x^4+y^4=z^2$ has no solution $(x,y,z)$ with $x,y,z>0$.
	\end{thm}
}
\slide{
	\begin{statementblock}{Corollary}
		The Diophantine equation $x^4+y^4=z^4$ has no solution $(x,y,z)$ with $x,y,z>0$.
	\end{statementblock}
}
\slide{
	\begin{statementblock}{Fermat's Last Theorem}
		For $n>\geq 3$, the Diophantine equation $x^n+y^n=z^n$ has no solution $(x,y,z)$ with $x,y,z>0$.
	\end{statementblock}
}
\end{document}

