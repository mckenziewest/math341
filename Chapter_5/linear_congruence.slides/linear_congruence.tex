\documentclass[t]{beamer}

\subtitle{Solving Linear Congruence and the Chinese Remainder Theorem}

\usepackage{amsthm,amsmath,amsfonts,hyperref,graphicx,color,multicol,soul}
\usepackage{enumitem,tikz,tikz-cd,setspace,mathtools}

%%%%%%%%%%
%Beamer Template Customization
%%%%%%%%%%
\setbeamertemplate{navigation symbols}{}
\setbeamertemplate{theorems}[ams style]
\setbeamertemplate{blocks}[rounded]

\definecolor{Blu}{RGB}{43,62,133} % UWEC Blue
\setbeamercolor{structure}{fg=Blu} % Titles

%Unnumbered footnotes:
\newcommand{\blfootnote}[1]{%
	\begingroup
	\renewcommand\thefootnote{}\footnote{#1}%
	\addtocounter{footnote}{-1}%
	\endgroup
}

%%%%%%%%%%
%TikZ Stuff
%%%%%%%%%%
\usetikzlibrary{arrows}
\usetikzlibrary{shapes.geometric}
\tikzset{
	smaller/.style={
		draw,
		regular polygon,
		regular polygon sides=3,
		fill=white,
		node distance=2cm,
		minimum height=1in,
		line width = 2pt
	}
}
\tikzset{
	smsquare/.style={
		draw,
		regular polygon,
		regular polygon sides=4,
		fill=white,
		node distance=2cm,
		minimum height=1in,
		line width = 2pt
	}
}


%%%%%%%%%%
%Custom Commands
%%%%%%%%%%

\newcommand{\C}{\mathbb{C}}
\newcommand{\quats}{\mathbb{H}}
\newcommand{\N}{\mathbb{N}}
\newcommand{\Q}{\mathbb{Q}}
\newcommand{\R}{\mathbb{R}}
\newcommand{\Z}{\mathbb{Z}}

\newcommand{\ds}{\displaystyle}

\newcommand{\fn}{\insertframenumber}

\newcommand{\id}{\operatorname{id}}
\newcommand{\im}{\operatorname{im}}
\newcommand{\Aut}{\operatorname{Aut}}
\newcommand{\Inn}{\operatorname{Inn}}

\newcommand{\blank}[1]{\underline{\hspace*{#1}}}

\newcommand{\abar}{\overline{a}}
\newcommand{\bbar}{\overline{b}}
\newcommand{\cbar}{\overline{c}}

\newcommand{\nml}{\unlhd}

%%%%%%%%%%
%Custom Theorem Environments
%%%%%%%%%%
\theoremstyle{definition}
\newtheorem{exercise}{Exercise}
\newtheorem{question}[exercise]{Question}
\newtheorem{warmup}{Warm-Up}
\newtheorem*{defn}{Definition}
\newtheorem*{exa}{Example}
\newtheorem*{disc}{Group Discussion}
\newtheorem*{nb}{Note}
\newtheorem*{recall}{Recall}
\renewcommand{\emph}[1]{{\color{blue}\texttt{#1}}}

\definecolor{Gold}{RGB}{237, 172, 26}
%Statement block
\newenvironment{statementblock}[1]{%
	\setbeamercolor{block body}{bg=Gold!20}
	\setbeamercolor{block title}{bg=Gold}
	\begin{block}{\textbf{#1.}}}{\end{block}}
\newenvironment{thm}[1]{%
	\setbeamercolor{block body}{bg=Gold!20}
	\setbeamercolor{block title}{bg=Gold}
	\begin{block}{\textbf{Theorem #1.}}}{\end{block}}


%%%%%%%%%%
%Custom Environment Wrappers
%%%%%%%%%%
\newcommand{\enumarabic}[1]{
	\begin{enumerate}[label=\textbf{\arabic*.}]
		#1
	\end{enumerate}
}
\newcommand{\enumalph}[1]{
	\begin{enumerate}[label=(\alph*)]
		#1
	\end{enumerate}
}
\newcommand{\bulletize}[1]{
	\begin{itemize}[label=$\bullet$]
		#1
	\end{itemize}
}
\newcommand{\circtize}[1]{
	\begin{itemize}[label=$\circ$]
		#1
	\end{itemize}
}
\newcommand{\slide}[1]{
	\begin{frame}{\fn}
		#1
	\end{frame}
}
\newcommand{\slidec}[1]{
\begin{frame}[c]{\fn}
	#1
\end{frame}
}
\newcommand{\slidet}[2]{
	\begin{frame}{\fn\ - #1}
		#2
	\end{frame}
}


\newcommand{\startdoc}{
		\title{Math 425: Abstract Algebra 1}
		\author{Mckenzie West}
		\date{Last Updated: \today}
		\begin{frame}
			\maketitle
		\end{frame}
}

\newcommand{\topics}[2]{
	\begin{frame}{\insertframenumber}
		\begin{block}{\textbf{Last Section.}}
			\begin{itemize}[label=--]
				#1
			\end{itemize}
		\end{block}
		\begin{block}{\textbf{This Section.}}
			\begin{itemize}[label=--]
				#2
			\end{itemize}
		\end{block}
	\end{frame}
}

\begin{document} 
	\startdoc
	
	\topics{
		% Last Time
		\item Zero Divisors
		\item Representations of Integers beyond decimal
		\item Fast exponentiation algorithm
		\item Sums of squares
		\item Sums of cubes
	}
	{
		% This time
		\item Solving Linear Congruence Equations
		\item The Chinese Remainder Theorem
	}



\slide{
	\begin{defn}
		A \emph{linear congruence} is an equation of the form
			\[ax\equiv b\pmod n,\]
		where $a,b,n\in\Z$ and $x$ is a variable.  
		A \emph{solution} to this linear congruence is an integer $x_0$ such that
			\[ax_0\equiv b\pmod n.\]
	\end{defn}
	\begin{exa} Find a solution.
		\enumalph{
			\item $3x\equiv 7\pmod{10}$\vskip .15in
			\item $2x\equiv 6\pmod{10}$\vskip .15in
			\item $2x\equiv 7\pmod{10}$
		}
	\end{exa}
}
\slide{
	\begin{defn}
		To \emph{solve} the linear congruence $ax\equiv b\pmod n$, we find all $0\leq x\leq n-1$ that satisfy the equation.
	\end{defn}
	\begin{exa} Solve.
		\enumalph{
			\item $3x\equiv 7\pmod{10}$\vskip .25in
			\item $2x\equiv 6\pmod{10}$\vskip .25in
			\item $2x\equiv 7\pmod{10}$
		}
	\end{exa}
}
\slide{
	\begin{nb}
		By definition of congruence, saying $ax\equiv b\pmod n$ is equivalent to saying there is some $y$ such that $ax-b=ny$.
	\end{nb}
}
\slide{
	\begin{statementblock}{Theorem 4.7}
		The linear congruence $ax\equiv b\pmod n$ has a solution if and only if $d|b$, where $d=\gcd(a,n)$.
		
		In that case there are exactly $d$ mutually incongruent solutions of the form
			\[x_0, x_0+\frac{n}{d}, x_0+2\frac{n}{d}+\cdots+x_0+(d-1)\frac{n}{d}.\]
	\end{statementblock}
}
\slide{
	\begin{exercise}
		Solve the linear congruence $9x\equiv 21\pmod{30}$.
	\end{exercise}
}
\slide{
	\begin{exercise}
		Solve the linear congruence $7x\equiv 3\pmod{24}$.
	\end{exercise}
}
\slide{
	\begin{statementblock}{Corollary}
		If $\gcd(a,n)=1$, then the linear congruence $ax\equiv b\pmod n$ has a unique solution modulo $n$.
	\end{statementblock}
	\begin{defn}
		Given $\gcd(a,n)=1$, the unique solution to $ax\equiv 1\pmod n$ is called the \emph{(multiplicative) inverse of $a$ modulo $n$}.
	\end{defn}
}
\slidet{Finding the inverse of $a$ modulo $n$}{
	\exer{
		Find the inverse of $13$ modulo $231$.
	}
}
\slide{
	\begin{exercise}
		Find an integer $x$ that satisfies
		\[3x\equiv7\pmod 8\quad\text{and}\quad 4x\equiv3\pmod{11}.\]
	\end{exercise}
}
\slide{
	\begin{exercise}
		Find an integer $x$ that satisfies
		\[6x\equiv4\pmod{10},\quad 3x\equiv7\pmod 8\quad\text{and}\quad 4x\equiv3\pmod{11}.\]
	\end{exercise}
}
\slide{
	\begin{statementblock}{Chinese Remainder Theorem (CRT)}
		Let $n_1,n_2,\dots,n_r$ be positive integers such that $\gcd(n_i,n_j)=1$ for all $i\neq j$.  Then the system
		\begin{eqnarray*}
		x&\equiv& a_1\pmod{n_1}\\
		x&\equiv& a_2\pmod{n_2}\\
		&\vdots\\
		x&\equiv& a_r\pmod{n_r}\\
		\end{eqnarray*}
		has a simultaneous solution which is unique modulo $N=n_1\cdot n_2\cdots n_r$.
	\end{statementblock}
}
\slide{
	\begin{proof}[``Proof''.]
		\bulletize{
			\item For all $k$, set $N_k= \frac{N}{n_k}$.
			\item For all $k$, set $x_k$ to be the solution of $N_k x\equiv 1\pmod{n_k}$.
			\item Then a solution is
				\[\bar x=a_1N_1x_1+a_2N_2x_2+\cdots+a_rN_rx_r.\]
		}
	\end{proof}
}
\slide{
	\begin{exercise}
		Solve the system
		\begin{multline*}
		x\equiv6\pmod{11},\quad x\equiv13\pmod{16},\\ x\equiv9\pmod{21},\quad x\equiv 19\pmod{25}
		\end{multline*}
	\end{exercise}
}
\slide{
	\begin{statementblock}{Lemma}
		If $m\mid n$ and $a\equiv b\pmod n$, then $a\equiv b\pmod m$ too.
	\end{statementblock}
}
\slide{
	\begin{statementblock}{Lemma}
		If \enumalph{\item$\gcd(m_1,m_2)=1$, \item$x\equiv y\pmod {m_1}$, and \item$x\equiv y \pmod{m_2}$,} then $$x\equiv y\pmod {m_1m_2}.$$
	\end{statementblock}
}
\slide{
	\begin{exercise}
		Solve the linear congruence
			\[x\equiv 9\pmod {276}\]
		by using $276=2^2\cdot 3\cdot 23$ and the CRT.
	\end{exercise}
}
\slide{
	\begin{exercise}
		Solve the linear congruences
		\[x^2\equiv 1\pmod{16}\quad\text{and}\quad x^2\equiv1\pmod 9.\]
	\end{exercise}
}
\slide{
	\begin{nb}
		We can also solve systems if the moduli are not relatively prime.
	\end{nb}
	\begin{exercise}
		Solve 
		\begin{equation*}
			x\equiv6\pmod{12},\quad x\equiv2\pmod{8}.
		\end{equation*}
	\end{exercise}
}
\slide{
	\begin{exercise}
		Solve 
		\begin{equation*}
			x\equiv6\pmod{12},\quad x\equiv5\pmod{8}.
		\end{equation*}
	\end{exercise}
}
\slide{
	\begin{nb}
		We can also solve systems if there are coefficients.
	\end{nb}
	\begin{exercise}
		Solve 
		\begin{equation*}
			5x\equiv6\pmod{12},\quad 3x\equiv2\pmod{7}.
		\end{equation*}
	\end{exercise}
}
\slide{
	\begin{statementblock}{Theorem}
		The system of linear congruences
		\[ax+by\equiv r\pmod n\]
		\[cx+dy\equiv s\pmod n\]
		has a unique solution modulo $n$ whenever $\gcd(ad-bc,n)=1$.
		
		Specifically if $t$ is an integer such that $t(ad-bc)\equiv 1\pmod n$, then the solution satisfies
			\[x\equiv t(dr-bs)\pmod n\]
			\[y\equiv t(as-cr)\pmod n.\]
	\end{statementblock}
}
\slide{\vskip-.25in
	\begin{proof}[``Proof''.]
		\bulletize{
			\item Multiply the first equation by $d$ and the second equation by $b$ and subtract to get
				\[(ad-bc)x\equiv (rd-sb)\pmod n.\]
			\item Multiply the left and right of this equation by $t$ to get
				\[x\equiv t(rd-sb)\pmod n.\]
			\item Similarly, for $y$ multiply the first equation by $c$ and second by $a$ then subtract the first from the second to get
				\[(ad-bc)y\equiv (as-cr)\pmod n.\]
			\item Multiply the left and right of this equation by $t$ to get
				\[y\equiv t(as-cr)\pmod n.\]
		}
	\end{proof}
}
\slide{
	\begin{exercise}
		Solve the system of linear congruences.
			\begin{eqnarray*}
			x-y&\equiv& 14\pmod {15}\\
			4x+3y&\equiv&10\pmod{15}
			\end{eqnarray*}
	\end{exercise}
	}
\end{document}

