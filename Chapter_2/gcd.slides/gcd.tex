\documentclass[t]{beamer}

\subtitle{Greatest Common Divisors and Relatively Prime Integers}

\usepackage{amsthm,amsmath,amsfonts,hyperref,graphicx,color,multicol,soul}
\usepackage{enumitem,tikz,tikz-cd,setspace,mathtools}

%%%%%%%%%%
%Beamer Template Customization
%%%%%%%%%%
\setbeamertemplate{navigation symbols}{}
\setbeamertemplate{theorems}[ams style]
\setbeamertemplate{blocks}[rounded]

\definecolor{Blu}{RGB}{43,62,133} % UWEC Blue
\setbeamercolor{structure}{fg=Blu} % Titles

%Unnumbered footnotes:
\newcommand{\blfootnote}[1]{%
	\begingroup
	\renewcommand\thefootnote{}\footnote{#1}%
	\addtocounter{footnote}{-1}%
	\endgroup
}

%%%%%%%%%%
%TikZ Stuff
%%%%%%%%%%
\usetikzlibrary{arrows}
\usetikzlibrary{shapes.geometric}
\tikzset{
	smaller/.style={
		draw,
		regular polygon,
		regular polygon sides=3,
		fill=white,
		node distance=2cm,
		minimum height=1in,
		line width = 2pt
	}
}
\tikzset{
	smsquare/.style={
		draw,
		regular polygon,
		regular polygon sides=4,
		fill=white,
		node distance=2cm,
		minimum height=1in,
		line width = 2pt
	}
}


%%%%%%%%%%
%Custom Commands
%%%%%%%%%%

\newcommand{\C}{\mathbb{C}}
\newcommand{\quats}{\mathbb{H}}
\newcommand{\N}{\mathbb{N}}
\newcommand{\Q}{\mathbb{Q}}
\newcommand{\R}{\mathbb{R}}
\newcommand{\Z}{\mathbb{Z}}

\newcommand{\ds}{\displaystyle}

\newcommand{\fn}{\insertframenumber}

\newcommand{\id}{\operatorname{id}}
\newcommand{\im}{\operatorname{im}}
\newcommand{\lcm}{\operatorname{lcm}}
\newcommand{\Aut}{\operatorname{Aut}}
\newcommand{\Inn}{\operatorname{Inn}}

\newcommand{\blank}[1]{\underline{\hspace*{#1}}}

\newcommand{\abar}{\overline{a}}
\newcommand{\bbar}{\overline{b}}
\newcommand{\cbar}{\overline{c}}

\newcommand{\nml}{\unlhd}

%%%%%%%%%%
%Custom Theorem Environments
%%%%%%%%%%
\theoremstyle{definition}
\newtheorem{exercise}{Exercise}
\newtheorem{question}[exercise]{Question}
\newtheorem{warmup}{Warm-Up}
\newtheorem*{exa}{Example}
\newtheorem*{disc}{Group Discussion}
\newtheorem*{recall}{Recall}
\renewcommand{\emph}[1]{{\color{blue}\texttt{#1}}}

\definecolor{Gold}{RGB}{237, 172, 26}
%Statement block
\newenvironment{statementblock}[1]{%
	\setbeamercolor{block body}{bg=Gold!20}
	\setbeamercolor{block title}{bg=Gold}
	\begin{block}{\textbf{#1.}}}{\end{block}}
\newenvironment{goldblock}{%
	\setbeamercolor{block body}{bg=Gold!20}
	\setbeamercolor{block title}{bg=Gold}
	\setbeamertemplate{blocks}[shadow=true]
	\begin{block}{}}{\end{block}}
\newenvironment{defn}{%
	\setbeamercolor{block body}{bg=gray!20}
	\setbeamercolor{block title}{bg=violet, fg=white}
	\setbeamertemplate{blocks}[shadow=true]
	\begin{block}{\textbf{Definition.}}}{\end{block}}
\newenvironment{nb}{%
	\setbeamercolor{block body}{bg=gray!20}
	\setbeamercolor{block title}{bg=teal, fg=white}
	\setbeamertemplate{blocks}[shadow=true]
	\begin{block}{\textbf{Note.}}}{\end{block}}
\newenvironment{blockexample}{%
	\setbeamercolor{block body}{bg=gray!20}
	\setbeamercolor{block title}{bg=Blu, fg=white}
	\setbeamertemplate{blocks}[shadow=true]
	\begin{block}{\textbf{Example.}}}{\end{block}}
\newenvironment{blocknonexample}{%
	\setbeamercolor{block body}{bg=gray!20}
	\setbeamercolor{block title}{bg=purple, fg=white}
	\setbeamertemplate{blocks}[shadow=true]
	\begin{block}{\textbf{Non-Example.}}}{\end{block}}
\newenvironment{thm}[1]{%
	\setbeamercolor{block body}{bg=Gold!20}
	\setbeamercolor{block title}{bg=Gold}
	\begin{block}{\textbf{Theorem #1.}}}{\end{block}}


%%%%%%%%%%
%Custom Environment Wrappers
%%%%%%%%%%
\newcommand{\exer}[1]{
	\begin{exercise}
		#1
	\end{exercise}
}
\newcommand{\exam}[1]{
\begin{blockexample}
	#1
\end{blockexample}
}
\newcommand{\nexam}[1]{
\begin{blocknonexample}
	#1
\end{blocknonexample}
}
\newcommand{\enumarabic}[1]{
	\begin{enumerate}[label=\textbf{\arabic*.}]
		#1
	\end{enumerate}
}
\newcommand{\enumalph}[1]{
	\begin{enumerate}[label=(\alph*)]
		#1
	\end{enumerate}
}
\newcommand{\bulletize}[1]{
	\begin{itemize}[label=$\bullet$]
		#1
	\end{itemize}
}
\newcommand{\circtize}[1]{
	\begin{itemize}[label=$\circ$]
		#1
	\end{itemize}
}
\newcommand{\slide}[1]{
	\begin{frame}{\fn}
		#1
	\end{frame}
}
\newcommand{\slidec}[1]{
\begin{frame}[c]{\fn}
	#1
\end{frame}
}
\newcommand{\slidet}[2]{
	\begin{frame}{\fn\ - #1}
		#2
	\end{frame}
}


\newcommand{\startdoc}{
		\title{Math 341: Classical Number Theory}
		\author{Mckenzie West}
		\date{Last Updated: \today}
		\begin{frame}
			\maketitle
		\end{frame}
}

\newcommand{\topics}[2]{
	\begin{frame}[c]{\insertframenumber}
		\begin{block}{\textbf{Last Section.}}
			\begin{itemize}[label=--]
				#1
			\end{itemize}
		\end{block}
		\begin{block}{\textbf{This Section.}}
			\begin{itemize}[label=--]
				#2
			\end{itemize}
		\end{block}
	\end{frame}
}

\begin{document} 
	\startdoc
	
	\topics{
		% Last Time
		\item The Division Algorithm
		\item Consequences of the Division Algorithm
	}
	{
		% This time
		\item Greatest Common Divisor
		\item Bezout's Identity
	}

\slide{
	\begin{defn}
		Let $a$ and $b$ be given integers, with at least one of them different from zero. The \emph{greatest common divisor} of $a$ and $b$, denoted by $\gcd(a,b)$, is the positive integer $d$ satisfy the following:
		\enumalph{
			\item $d|a$ and $d|b$.
			\item If $c|a$ and $c|b$, then $c\leq d$.
		}
	\end{defn}
	\begin{exercise}
		\enumalph{\setlength{\itemsep}{1.5em}
			\item $\gcd(-12,30)$
			\item $\gcd(-5,5)$
			\item $\gcd(8,17)$
			\item $\gcd(-8,-36)$
		}
	\end{exercise}
}
\slide{
	\begin{exercise}
		Find the $\gcd$ of 81, 231, and 3465.
		
		Let's create a factor tree for each:
	\end{exercise}
}
\slide{
	\begin{defn}
		Let $a$ and $b$ be given integers. The \emph{least common multiple} of $a$ and $b$, denoted by $\lcm(a,b)$, is the positive integer $m$ satisfy the following:
		\enumalph{
			\item $a|m$ and $b|m$.
			\item If $a|c$ and $b|c$, then $m\leq c$.
		}
	\end{defn}
}
\slide{
\begin{exercise}
	Find the $\lcm$ of 81, 231, and 3465.
	
	Recall:
		\begin{quote}
			$\begin{array}{rcl}
		81&=&3\cdot3\cdot 3\cdot 3\\
		231&=&3\cdot7\cdot11\\
		3465&=&3\cdot3\cdot5\cdot7\cdot11
		\end{array}$
		\end{quote}
\end{exercise}
}
\slide{
	\begin{exercise}
		Show that if $d=\gcd(a,a+n)$, then $d|n$.
	\end{exercise}
}
\slide{
	\exer{ 
		What is $\gcd(0,0)$ according to this definition?
	}
}
\slide{
	\begin{statementblock}{GCD as linear combination}
		Given integers $a$ and $b$, not both zero, there exist integers $x$ and $y$ such that
			\[\gcd(a,b)=ax+by.\]
	\end{statementblock}
}
\slide{
	\begin{defn}
		We call the two integers $a$ and $b$, not both zero, \emph{relatively prime} if $\gcd(a,b)=1$.
	\end{defn}
	\begin{statementblock}{Relative Prime Test}
		Let $a$ and $b$  be integers, not both zero. If there exist integers $x$ and $y$ such that $1=ax+by$, then $a$ and $b$ are relatively prime.
	\end{statementblock}
}
\slide{
	\begin{exercise}
		Show that for all $a\in\Z$,
			\[\gcd(2a+1,9a+4)=1.\]
	\end{exercise}
}
\slide{
	\begin{statementblock}{Corollary 1}
		If $\gcd(a,b)=d$, then $\gcd(a/d,b/d)=1$.
	\end{statementblock}\vskip 1in
	\begin{statementblock}{Corollary 2}
		If $a|c$ and $b|c$, with $\gcd(a,b)=1$, then $ab|c$.
	\end{statementblock}
}
\slide{
	\begin{statementblock}{Euclid's Lemma}
		If $a|bc$, with $\gcd(a,b)=1$, then $a|c$.
	\end{statementblock}
	\begin{proof}
		Let $a,b,c\in \Z$ such that $a|bc$ and $\gcd(a,b)=1$.
	\end{proof}
}
\slide{
	\begin{statementblock}{Equivalent Definition}
		Let $a,b\in\Z$ not both zero and $d\in\N$.  We have
		\[d=\gcd(a,b)\quad\Leftrightarrow\quad \begin{array}{ll}
		\text{(a)}&d|a\text{ and }d|b.\\
		\text{(b)}&\text{Whenever }c|a\text{ and }c|b,\text{ then }c|d.
		\end{array}\]
	\end{statementblock}
}
\slide{
	\begin{exercise}
		Show that if $\gcd(a,b)=1$ and $\gcd(a,c)=1$, then $\gcd(a,bc)=1$.
	\end{exercise}
}
\end{document}

