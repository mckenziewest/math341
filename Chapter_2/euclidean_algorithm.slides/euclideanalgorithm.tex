\documentclass[t]{beamer}

\subtitle{Euclidean Algorithm}

\usepackage{amsthm,amsmath,amsfonts,hyperref,graphicx,color,multicol,soul}
\usepackage{enumitem,tikz,tikz-cd,setspace,mathtools}

%%%%%%%%%%
%Beamer Template Customization
%%%%%%%%%%
\setbeamertemplate{navigation symbols}{}
\setbeamertemplate{theorems}[ams style]
\setbeamertemplate{blocks}[rounded]

\definecolor{Blu}{RGB}{43,62,133} % UWEC Blue
\setbeamercolor{structure}{fg=Blu} % Titles

%Unnumbered footnotes:
\newcommand{\blfootnote}[1]{%
	\begingroup
	\renewcommand\thefootnote{}\footnote{#1}%
	\addtocounter{footnote}{-1}%
	\endgroup
}

%%%%%%%%%%
%TikZ Stuff
%%%%%%%%%%
\usetikzlibrary{arrows}
\usetikzlibrary{shapes.geometric}
\tikzset{
	smaller/.style={
		draw,
		regular polygon,
		regular polygon sides=3,
		fill=white,
		node distance=2cm,
		minimum height=1in,
		line width = 2pt
	}
}
\tikzset{
	smsquare/.style={
		draw,
		regular polygon,
		regular polygon sides=4,
		fill=white,
		node distance=2cm,
		minimum height=1in,
		line width = 2pt
	}
}


%%%%%%%%%%
%Custom Commands
%%%%%%%%%%

\newcommand{\C}{\mathbb{C}}
\newcommand{\quats}{\mathbb{H}}
\newcommand{\N}{\mathbb{N}}
\newcommand{\Q}{\mathbb{Q}}
\newcommand{\R}{\mathbb{R}}
\newcommand{\Z}{\mathbb{Z}}

\newcommand{\ds}{\displaystyle}

\newcommand{\fn}{\insertframenumber}

\newcommand{\id}{\operatorname{id}}
\newcommand{\im}{\operatorname{im}}
\newcommand{\Aut}{\operatorname{Aut}}
\newcommand{\Inn}{\operatorname{Inn}}

\newcommand{\blank}[1]{\underline{\hspace*{#1}}}

\newcommand{\abar}{\overline{a}}
\newcommand{\bbar}{\overline{b}}
\newcommand{\cbar}{\overline{c}}

\newcommand{\nml}{\unlhd}

%%%%%%%%%%
%Custom Theorem Environments
%%%%%%%%%%
\theoremstyle{definition}
\newtheorem{exercise}{Exercise}
\newtheorem{question}[exercise]{Question}
\newtheorem{warmup}{Warm-Up}
\newtheorem*{defn}{Definition}
\newtheorem*{exa}{Example}
\newtheorem*{disc}{Group Discussion}
\newtheorem*{nb}{Note}
\newtheorem*{recall}{Recall}
\renewcommand{\emph}[1]{{\color{blue}\texttt{#1}}}

\definecolor{Gold}{RGB}{237, 172, 26}
%Statement block
\newenvironment{statementblock}[1]{%
	\setbeamercolor{block body}{bg=Gold!20}
	\setbeamercolor{block title}{bg=Gold}
	\begin{block}{\textbf{#1.}}}{\end{block}}
\newenvironment{thm}[1]{%
	\setbeamercolor{block body}{bg=Gold!20}
	\setbeamercolor{block title}{bg=Gold}
	\begin{block}{\textbf{Theorem #1.}}}{\end{block}}


%%%%%%%%%%
%Custom Environment Wrappers
%%%%%%%%%%
\newcommand{\enumarabic}[1]{
	\begin{enumerate}[label=\textbf{\arabic*.}]
		#1
	\end{enumerate}
}
\newcommand{\enumalph}[1]{
	\begin{enumerate}[label=(\alph*)]
		#1
	\end{enumerate}
}
\newcommand{\bulletize}[1]{
	\begin{itemize}[label=$\bullet$]
		#1
	\end{itemize}
}
\newcommand{\circtize}[1]{
	\begin{itemize}[label=$\circ$]
		#1
	\end{itemize}
}
\newcommand{\slide}[1]{
	\begin{frame}{\fn}
		#1
	\end{frame}
}
\newcommand{\slidec}[1]{
\begin{frame}[c]{\fn}
	#1
\end{frame}
}
\newcommand{\slidet}[2]{
	\begin{frame}{\fn\ - #1}
		#2
	\end{frame}
}


\newcommand{\startdoc}{
		\title{Math 425: Abstract Algebra 1}
		\author{Mckenzie West}
		\date{Last Updated: \today}
		\begin{frame}
			\maketitle
		\end{frame}
}

\newcommand{\topics}[2]{
	\begin{frame}{\insertframenumber}
		\begin{block}{\textbf{Last Section.}}
			\begin{itemize}[label=--]
				#1
			\end{itemize}
		\end{block}
		\begin{block}{\textbf{This Section.}}
			\begin{itemize}[label=--]
				#2
			\end{itemize}
		\end{block}
	\end{frame}
}

\begin{document} 
	\startdoc
	
	\topics{
		% Last Time
		\item GCD and LCM
		\item properties of gcd
	}
	{
		% This time
		\item Euclidean Algorithm
		\item Finding $x,y$ with $ax+by=\gcd(a,b)$
	}
\slide{
	\begin{statementblock}{Linear Combination Theorem}
		Given integers $a$ and $b$, not both zero, there exist integers $x$ and $y$ such that
			\[\gcd(a,b)=ax+by.\]
	\end{statementblock}
}
\slide{ 
	\begin{statementblock}{GCD and Remainders}
		For $a,b\in\Z$ not both zero.  If $a=bq+r$, then 
			$$\gcd(a,b)=\gcd(b,r).$$
	\end{statementblock}
}
\slide{
	\begin{defn}
		The \emph{Euclidean algorithm} is a method from computing $\gcd(a,b)$. Assume for simplicity that $a\geq b>0$ since $\gcd(|a|,|b|)=\gcd(a,b)$ and $\gcd(a,0)=a$ if $a>0$. By the division algorithm, we can write
		
			\[
			\begin{array}{rcll}
			a&=&q_0b+r_1 & 0<r_1<b\\
			b&=&q_1r_1+r_2 & 0<r_2<r_1\\
			r_1&=&q_2r_2+r_3 & 0<r_3<r_2\\
			r_2&=&q_3r_3+r_4 & 0<r_4<r_3\\
			&\vdots\\
			r_{n-2}&=&q_{n-1}r_{n-1}+r_n & 0<r_n<r_{n-1}\\
			r_{n-1}&=&q_nr_n+0.
			\end{array}
			\]
		Then \fbox{$r_n=\gcd(a,b)$}.
	\end{defn}
}
\slide{
	\begin{nb}
		\bulletize{
			\item The Euclidean algorithm terminates because $$b>r_1>r_2>r_3>\cdots>r_n>0.$$ Thus the remainders, which form a set of positive numbers, is decreasing and thus has a smallest element which must appear within a finite number of steps.
		
			\item Moreover, 
				\begin{multline*}
				\gcd(a,b)=\gcd(b,r_1)=\gcd(r_1,r_2)=\cdots\\
				=\gcd(r_{n-1},r_n)=\gcd(r_n,0)=r_n.
				\end{multline*}
			}
	\end{nb}
}
\slide{
	\begin{exercise}
		Use the Euclidean algorithm to compute $\gcd(1767,11571)$:
		\begin{eqnarray*}
		11571&=&\blank{.25in}\cdot {\color{red}1767} + {\color{orange}969}\\\\
		{\color{red}1767}&=&\blank{.25in}\cdot {\color{orange}969}+ {\color{yellow}798}\\\\
		{\color{orange}969}&=&\blank{.25in}\cdot {\color{yellow}798}+ {\color{green}171}\\\\
		{\color{yellow}798}&=&\blank{.25in}\cdot {\color{green}171}+ {\color{blue}114}\\\\
		{\color{green}171}&=&\blank{.25in}\cdot {\color{blue}114}+ {\color{purple}57}\\\\
		{\color{blue}114}&=&\blank{.25in}\cdot {\color{purple}57}+ 0
		\end{eqnarray*}
	\end{exercise}
}
\slide{
	\begin{exercise}
		Compute $\gcd(15,49)$ using the Euclidean algorithm.
		
		\begin{eqnarray*}
			49&=&\blank{.25in}\cdot 15 + \blank{.25in}\\\\
			15&=&\blank{.25in}\cdot\blank{.25in}+\blank{.25in}\\\\
			\blank{.25in}&=&\blank{.25in}\cdot\blank{.25in}+\blank{.25in}\\\\
			\blank{.25in}&=&\blank{.25in}\cdot\blank{.25in}+\blank{.25in}
		\end{eqnarray*}
		
		Thus $\gcd(15,49)=1$.
	\end{exercise}
}
\slide{
	\begin{question}
		This algorithm is annoying, yes, but good for a computer.  It also helps us find the $x$ and $y$ from B\'ezout's Identity.
	\end{question}
	\begin{exa}
		If we want to find $x$ and $y$ such that $1=15x+49y$, we work backwards up the Euclidean algorithm:
		
		\begin{center}
		\fbox{\begin{minipage}{.3\textwidth}
			\begin{eqnarray*}
			{\color{green}49}&=&3\cdot {\color{red}15}+{\color{orange}4}\\
			{\color{red}15}&=&3\cdot {\color{orange}4} + {\color{blue}3}\\
			{\color{orange}4}&=&1\cdot {\color{blue}3} + \textbf{1}\\
			{\color{blue}3}&=&3\cdot \textbf{1} + 0\\
			\end{eqnarray*}
		\end{minipage}}
		\quad
		\begin{minipage}{.4\textwidth}
			\begin{eqnarray*}
				\textbf{1}&=&{\color{orange}4}-1\cdot {\color{blue}3}\\
				&=&{\color{orange}4}-1\cdot ({\color{red}15}-3\cdot {\color{orange}4})\\
				&=&4\cdot{\color{orange}4}-1\cdot{\color{red}15}\\
				&=&4\cdot({\color{green}49}-3\cdot {\color{red}15})-1\cdot{\color{red}15}\\
				&=&4\cdot {\color{green}49} - 13\cdot{\color{red}15}\\
			\end{eqnarray*}
		\end{minipage}
		\end{center}
	\end{exa}
}
\slide{
	\begin{exercise}
		Use your Euclidean algorithm results to find $x$ and $y$ such that $57=11571x+1767y$.
		\vskip .25in
		$\begin{array}{rcl}
			11571&=&6\cdot {\color{red}1767} + {\color{orange}969}\\
			{\color{red}1767}&=&1\cdot {\color{orange}969}+ {\color{yellow}798}\\
			{\color{orange}969}&=&1\cdot {\color{yellow}798}+ {\color{green}171}\\
			{\color{yellow}798}&=&4\cdot {\color{green}171}+ {\color{blue}114}\\
			{\color{green}171}&=&1\cdot {\color{blue}114}+ {\color{purple}57}\\
			{\color{blue}114}&=&2\cdot {\color{purple}57}+ 0
		\end{array}$
	\end{exercise}
}
\slide{
	\begin{exercise}
		Find integers $x$ and $y$ such that $\gcd(172,20)=172x+20y$.  (First find the gcd.)
	\end{exercise}
}
\slide{
	\begin{exercise}
		Find a different pair of integers $x$ and $y$ such that $\gcd(172,20)=172x+20y$.
	\end{exercise}
}
\slide{
	\begin{exercise}
		Find integers $x$ and $y$ such that $336=172x+20y$. 
	\end{exercise}
}
\slide{
\begin{thm}{2.6}
	Let $a,b\in\Z$ not both zero and $d\in\N$.  We have
	\[d=\gcd(a,b)\quad\Leftrightarrow\quad \begin{array}{ll}
	\text{(a)}&d|a\text{ and }d|b.\\
	\text{(b)}&\text{Whenever }c|a\text{ and }c|b,\text{ then }c|d.
	\end{array}\]
\end{thm}
	\begin{nb}
		This gives us an alternate definition of $\gcd$.  Rather than being the biggest, it can be seen as the one divisible by all other divisors.
	\end{nb}
}
\slide{
	\begin{thm}{}
		If $a,b,k$ are integers then
		$$\gcd(ka,kb)=k\gcd(a,b).$$
	\end{thm}
}

\end{document}

