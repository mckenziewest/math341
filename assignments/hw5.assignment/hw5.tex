\documentclass[12pt]{article}
\usepackage{amsmath,amsthm,amssymb,hyperref,color}
\usepackage[margin=1in]{geometry}


\newcommand{\R}{\mathbb{R}}
\newcommand{\Z}{\mathbb{Z}}
\newcommand{\lcm}{\operatorname{lcm}}

\begin{document}
	Math 341 Homework 5
	\hfill
	Name: \underline{\hspace*{2in}}
	
\begin{enumerate}
	\item  (10 pts) Prove that if $a\equiv b\pmod n$ and the integers $a,b,n$ are all divisible by $d>0$, then $a/d\equiv b/d\pmod{n/d}$.
	\vfill
	\item (10 pts) If $\gcd(a,b)=n$, show that $a\equiv b\pmod n$.
	\vfill 
	\newpage
	\item (10 pts) Suppose $N=a_ma_{m-1}\dots a_1a_0$ are the digits of $N$, that is 
		$$N= a_m10^m+a_{m-1}10^{m-1}+\cdots+a_110+a_0$$ with $0\leq a_i<10$ for all $i$. Show that $6\mid n$ if and only if $6\mid (a_0+4a_1+\cdots + 4a_{m-1}+4a_m)$.
	\vfill
	\item (5 pts) Verify that $53^{103}+103^{53}$ is divisible by 39.
	\vfill
	\newpage
	\item (10 pts) Show that for all $n\geq 1$, 
			$$27\mid 2^{5n+1}+5^{n+2}.$$
	\vfill
	\item (5 pts) Come up with your own divisibility rule like the one in the last problem. (This feels difficult to me, but I also think it will be rewarding to think about how these infinite divisibility rules come about.)
	\vfill
	\newpage
	\item (10 pts) Use induction on $n$ to show the following statement. If $a$ is an odd integer then for all $n\geq 1$, 
			$$a^{2^n}\equiv 1\pmod{2^{n+2}}.$$
\end{enumerate}
		
\end{document}