\documentclass[12pt]{article}
\usepackage{amsmath,amsthm,amssymb,hyperref,color}
\usepackage[margin=1in]{geometry}

\newcommand{\Z}{\mathbb{Z}}

\begin{document}
	Math 341 Homework 8
	\hfill
	Name: \underline{\hspace*{2in}}
	
	\textbf{Instructions:} This assignment is on material from portions of Chapters 6 and 7: the Fundamental Theorem of Arithmetic, Hensel's Lemma, Fermat's Little Theorem, and Wilson's Theorem.
\begin{enumerate}
	\item 
		\begin{enumerate}
			\item (10 pts) Show that if $a$ and $b$ are positive integers such that $a^3|b^2$, then $a|b$.\vfill
			\item (5 pts) Show that it is possible for $a^2|b^3$ but $a\nmid b$ by finding possible $a$ and $b$ values.\vskip 2in
		\end{enumerate}
	\newpage
	\item  For parts of this problem, 			
			it may or may not be helpful to use the formula from class, that for any prime $p$, the valuation at $p$ of $n!$ is $$v_p(n!)=\left\lfloor\frac{n}{p}\right\rfloor+\left\lfloor\frac{n}{p^2}\right\rfloor+\left\lfloor\frac{n}{p^3}\right\rfloor+\cdots.$$
		  Note that in all parts of this problem, the number of zeros at the end of the decimal representation equals the largest power of $10$ that divides the number.
		\begin{enumerate}
			\item (5 pts) What integers $n$ satisfy the property that $n!$ has exactly $6$ zeros at the end of its decimal representation?
			\vskip 1in
			\item (5 pts) How many zeros are there at the end of the decimal representation of $5^k!$?
			\vskip 1in
			\item (10 pts) 
			Show that the number of zeros at the end of the decimal representation of $n!$ is less than the number of zeros at the end of the decimal representation of $(n+5)!$.

			\vfill
			
		\end{enumerate}
	\newpage
	\item \begin{enumerate}
		\item(5 pts) Use Hensel's lemma to show that $x^2+2x+5\equiv 0\pmod p^k$ has a solution for $p=5,13,17$ and all positive integers $k$.\vfill
		\item(10 pts) Use the Chinese Remainder Theorem to show that $x^2+2x+5\equiv 0\pmod n$ has a solution if the prime factorization of $n$ is $5^a13^b17^c$.\vfill
	\end{enumerate}
	\newpage
	\item (10 pts) Let $p$ be a prime and $\gcd(a,p)=1$. Use Fermat's theorem to verify that $x=a^{p-2}b$ is a solution to the congruence equation $ax\equiv b\pmod p$.\vfill
	\item (10 pts) If $p$ is prime, show that $M_p=2^p-1$ is either prime or a pseudoprime.
	\vfill
	\newpage
	\item (10 pts) Prove that $n>1$ is prime if and only if $(n-2)!\equiv 1\pmod n$.
\end{enumerate}
\end{document}