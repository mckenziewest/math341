\documentclass[12pt]{article}
\usepackage{amsmath,amsthm,amssymb,hyperref}
\usepackage[margin=1in]{geometry}


\newcommand{\N}{\mathbb{N}}
\newcommand{\Q}{\mathbb{Q}}
\newcommand{\R}{\mathbb{R}}
\newcommand{\Z}{\mathbb{Z}}

\newcommand{\lcm}{\operatorname{lcm}}

\begin{document}
Math 341 Midterm Exam
\hfill
Name: \underline{\hspace*{2in}}
\begin{enumerate}
	\item (20 pts) Watch \url{https://www.youtube.com/watch?v=KwFyO-ubuSM}, ProfOmarMath's video on Pell's Equation then answer the following questions.
		\begin{enumerate}
			\item (5 pts) In at least 3 sentences, describe the method that Professor Omar used to find infinitely many solutions to $x^2-2y^2=1$.
				\vfill
			\item Consider the equation $x^2-7y^2=1$.  
				\begin{enumerate}
					\item (2 pts) Find a single integer solution to this equation with $1\leq x,y\leq 10$. (Remember to give a small explanation, verifying your solution works for this and all computation problems.)
						\vskip 1in
					\item (5 pts) Find a recurrence relation that will produce infinitely many solutions the equation.
						\vfill
						\newpage
					\item (3 pts) Using your recurrence relation, write down at least 4 more solutions to the equation.
						\vfill
					\item (5 pts) Find an ratio of integers that approximates $\sqrt{7}$ to at least 4 decimal places. 
					\vfill
				\end{enumerate}
		\end{enumerate}
	\newpage
	\item (10 pts) If you can hop 289 units left and right, and skip 323 units left and right, where can you travel (beginning at 0)?  Note: 5 of these points are designated for you showing the entire Euclidean Algorithm steps.
		\vfill
		\newpage
	\item (10 pts) Prove that if $x$ is an integer for which $x|x^2+1$, then $x=\pm 1$.
		\vfill
		\newpage
	\item (20 pts) 
		\begin{enumerate}
			\item Let $r$ and $s$ be integers. Use proof by induction to show that if $n\geq 1$ is odd, then $r+s$ divides $r^n+s^n$.
			
			Note: You must use proof by induction.  Keep in mind that you can assume the inductive hypothesis for an odd integer $k$ then consider the case $n=k+2$, this is valid induction.
				\vfill
				\newpage
			\item Use the result of part (a) to show that 
				\[1^{47}+2^{47}+3^{47}+4^{47}+5^{47}+6^{47}\]
				is divisible by 7.
				
				Note: You must use part (a).  Little credit will be given for any other solution. You can though, do this part without proving part (a).
				\vfill
		\end{enumerate} 
	\newpage
	\item \begin{enumerate}
		\item (5 pts) If $x\in\Z$ what are the possible 1's digits of $x^2$? (Find and prove.)\vfill
		\item (5 pts) If $x\in\Z$ what are the possible 1's digits of $x^3$? (Find and prove.)\vfill
		\item (5 pts) If $x\in\Z$ what are the possible remainders when dividing $x^4$ by 7? (Find and prove.)\vfill
	\end{enumerate}
	\newpage
	\item (5 pts) For what values of $1\leq n\leq 20$ does
		$$x^2+x\equiv 1\pmod n$$
		have a solution?
		
		For those that have a solution, find all $0\leq x<n$ that satisfy the equivalence.
		
		If you use Sage or another programming language to solve this exercise, you must submit your code in some readable format. And really, you should be using code for this.
	\end{enumerate}
\end{document}