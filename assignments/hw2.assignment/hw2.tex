\documentclass[12pt]{article}
\usepackage{amsmath,amsthm,amssymb,hyperref,color}
\usepackage[margin=1in]{geometry}


\newcommand{\R}{\mathbb{R}}
\newcommand{\Z}{\mathbb{Z}}


\begin{document}
	Math 341 Homework 2
	\hfill
	Name: \underline{\hspace*{2in}}
	\begin{enumerate}
		\item (10 pts) Let $a$ be an integer. Prove each of the following statements.
			\begin{enumerate}
				\item $a|0$\vfill 
				\item $1|a$\vfill 
				\item $a|a$\vfill 
			\end{enumerate}
		\newpage
	\item (10 pts) Given integers $a,b,c,d$, prove using the definition of divisibility that if $a|b$ and $c|d$, then $ac|bd$.
	\vfill 
	\item (5 pts) Find a counterexample to show that $a|b$ and $c|d$, it is not necessarily true that $a+c|b+d$.\vfill
	\newpage 
	\item (10 pts) Given integers $a$ and $b$, prove using the definition of divisibility to show $a|b$ and $b|a$ if and only if $|a|=|b|$.
	
	Note: You may also use the fact that $x^2=1$ if and only if $x=\pm 1$.
	
	\textbf{Important:} This is a biconditional statement (if and only if) which means that in order to prove it, you must prove two conditional statements as given below.
		\begin{enumerate}
			\item If $a|b$ and $b|a$, then $|a|=|b|$.\vfill
			\item If $|a|=|b|$ then $a|b$ and $b|a$.\vfill
		\end{enumerate}
	\end{enumerate}
\end{document}