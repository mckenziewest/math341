\documentclass[12pt]{article}
\usepackage{amsmath,amsthm,amssymb,hyperref,color}
\usepackage[margin=1in]{geometry}


\newcommand{\R}{\mathbb{R}}
\newcommand{\Z}{\mathbb{Z}}


\begin{document}
	Math 341 Homework 2
	\hfill
	Name: \underline{\hspace*{2in}}
	\begin{enumerate}
		\item (10 pts) Let $a$ be an integer. Prove each of the following statements.
			\begin{enumerate}
				\item $a|0$\vfill 
				\item $1|a$\vfill 
				\item $a|a$\vfill 
			\end{enumerate}
		\newpage
	\item (10 pts) Given integers $a,b,c,d$, prove using the definition of divisibility that if $a|b$ and $c|d$, then $ac|bd$.
	\vfill 
	\item (5 pts) Find a counterexample to show that $a|b$ and $c|d$, it is not necessarily true that $a+c|b+d$.\vfill
	\newpage 
	\item (10 pts) Given integers $a$ and $b$, prove using the definition of divisibility to show $a|b$ and $b|a$ if and only if $|a|=|b|$.
	
	Note: You may also use the fact that $x^2=1$ if and only if $x=\pm 1$.
	
	\textbf{Important:} This is a biconditional statement (if and only if) which means that in order to prove it, you must prove two conditional statements as given below.
		\begin{enumerate}
			\item If $a|b$ and $b|a$, then $|a|=|b|$.\vfill
			\item If $|a|=|b|$ then $a|b$ and $b|a$.\vfill
		\end{enumerate}
	\newpage
	\item (10 pts) 
		\begin{enumerate}
			\item For several $n\geq 1$, compute $4^n+3\cdot 7^{n+2}+5$. Conjecture a common divisor. That is, find the largest integer $D$ such that the following is true.
			\begin{center}
				For all $n\geq 1$, the integer $4^n+3\cdot 7^{n+2}+5$ is divisible by $D$.
			\end{center}
			\vskip 2in
			\item Use a proof by induction to prove your conjecture for your value of $D$.
		\end{enumerate}
	\newpage 
	\item (5 pts) Consider the situation where the positive integer $a$ is divided by the positive integer $b$ using the division algorithm, yielding 
$$a=652\cdot b+8634.$$
		Now, for any integer $k$, we can write a new equation 
			$$a+k=652\cdot (b+k)+r,$$
		for some integer $r$. For what values of $k$ will this equation satisfy the division algorithm when dividing $a+k$ by $b+k$?
	\newpage
	\item (5 pts) Follow this link to a UWEC library listing for the book \textit{Fibonacci and Lucas numbers with applications}.
	
	\url{https://wisconsin-uwec.primo.exlibrisgroup.com/permalink/01UWI_EC/1rrp16b/cdi_askewsholts_vlebooks_9781118742181}
	
	We can access this book online through the library, select the first of the "View Online" options after reading the Login instructions.
	
	This book is all about sequences like the Fibonacci sequence. I'm going to just point you to a small part. 
	
	But here's some background first: Define $F_0=0$, $F_1=1$, and for all $n\geq 2$ $F_n=F_{n-1}+F_{n-2}$.
	
	Now that you know what $F_n$ is and you are logged in to the book, follow the link here to get to Theorem 10.1.
	
	\url{https://learning.oreilly.com/library/view/fibonacci-and-lucas/9781118742129/c10.xhtml#:-:text=Theorem%2010.1}
	
	Read the proof of this Theorem, note that it takes the property $F_{r+s}=F_{r-1}F_s+F_rF_{s+1}$ for granted. 
	
	\begin{enumerate}
		\item What kind of proof is this and why? \vfill 
		\item What's special about this proof? What's similar/different about this one compared to the one's we've been doing of this type?\vfill
	\end{enumerate}
	
	\end{enumerate}
		
\end{document}