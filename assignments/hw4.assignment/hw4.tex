\documentclass[12pt]{article}
\usepackage{amsmath,amsthm,amssymb,hyperref,color}
\usepackage[margin=1in]{geometry}


\newcommand{\R}{\mathbb{R}}
\newcommand{\Z}{\mathbb{Z}}
\newcommand{\lcm}{\operatorname{lcm}}

\begin{document}
	Math 341 Homework 4
	\hfill
	Name: \underline{\hspace*{2in}}
	
	\textbf{Instructions:} This assignment will work differently than the others that we have had so far. There are choices for you to make about what exercises you complete.
	
	Below there are 5 options, of which you must complete three, each of which are worth 20 total points.
\begin{enumerate}
	\item Complete part 1 of the Python worksheet we worked on in class on Monday, February 20, proving that if $a^n\mid b^n$, then $a\mid b$.
	\item Complete part 2 of the Python worksheet we worked on in class on Monday, February 20, proving that if $\gcd(a,b)=1$, then $\gcd(*,*)=*$.
	\item Write a function, preferably in Python/Sage, though other languages will do, with the following specifications:
	
		\fbox{\begin{minipage}{.9\textwidth}
			\textbf{Description:} Solves the Diophantine equation $aX+bY=c$.
			
			\textbf{Input:} Three integers $a,b,c$.
			
			\textbf{Output:} Integers $x,y,m,n$ such that all solutions to the equation $aX+bY=c$ can be written as $(x+mt,y+nt)$.
			
			\textbf{Errors:} Error if \begin{itemize}
				\item both $a$ and $b$ are 0, or
				\item $\gcd(a,b)$ does not divide $c$.
			\end{itemize}
		\end{minipage}}
	
	\item Write a function, preferably in Python/Sage, though other languages will do, with the following specifications:
	
	\fbox{\begin{minipage}{.9\textwidth}
			\textbf{Description:} Finds positive integer solutions to $aX+bY=c$.
			
			\textbf{Input:} Three integers $a,b,c$.
			
			\textbf{Output:} A list of all solutions to the Diophine equation $aX+bY=c$ with positive coordinates.
			
			\textbf{Errors:} Error if 
				\begin{itemize}
					\item any of $a$, $b$, or $c$ is not positive, or
					\item $\gcd(a,b)$ does not divide $c$.
				\end{itemize} 
	\end{minipage}}
\item In section 3.3.2 of the online texbook \begin{center}
	\url{https://math.gordon.edu/ntic/ntic/section-positive-lattice.html#subsection-toward-full-positive-linear}
\end{center} the author discusses 3 cases for number of positive integer solutions to $2x+3y=c$. Do the same for $6x+9y=c$. That is, compute the number of positive integer solutions for several $c$ values and find a pattern.
	
	
\end{enumerate}
		
\end{document}