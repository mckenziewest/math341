\documentclass[12pt]{article}
\usepackage{amsmath,amsthm,amssymb,hyperref,color}
\usepackage[margin=1in]{geometry}

\newcommand{\Z}{\mathbb{Z}}

\begin{document}
	Math 341 Homework 6
	\hfill
	Name: \underline{\hspace*{2in}}
	
	\textbf{Instructions:} For this assignment you may complete any number of items, the maximum number of points you will receive is 50. A partially incomplete question will not be considered.
\begin{enumerate}
	\item (20 pts) Complete Day 13 of Advent of Code 2020 here:
			 \begin{center}
			 	\url{https://adventofcode.com/2020/day/13}
			 \end{center}
		Advent of Code is an annual event that provides a new coding challenge every day December 1-25. The challenges involve some text processing and always have two parts.
		Many were unable to complete the second part of this particular challenge because it was unobtainable by brute force, only through the Chinese Remainder Theorem.
		
		You will need to sign up/create an account to complete this exercise. Please submit any code used to complete this challenge to Canvas.
		
	\item (5 pts) Solve the linear congruence $17x\equiv 3\pmod 2\cdot3\cdot 5\cdot 7$ by solving the system
		\begin{eqnarray*}
			17x&\equiv& 3\pmod 2\\
			17x&\equiv& 3\pmod 3\\
			17x&\equiv& 3\pmod 5\\
			17x&\equiv& 3\pmod 7
		\end{eqnarray*}
		Show your complete work. You must use the Chinese Remainder Theorem to earn credit. If you use Sage or a calculator, make note where.
		\vfill
		\newpage
	\item (10 pts) If $x\equiv a\pmod n$, prove that either $x\equiv a \pmod{2n}$ or $x\equiv a+n\pmod{2n}$.
		\vfill
	\item (10 pts) Obtain the two incongruent solutions modulo 210 of the system
	\begin{eqnarray*}
		2x\equiv 3\pmod 5\\4x\equiv 2\pmod 6\\3x\equiv 2\pmod 7
	\end{eqnarray*}
		Show your complete work. You must use the Chinese Remainder Theorem to earn credit. If you use Sage or a calculator, make note where.
		\vfill
		\newpage
	\item (10 pts) When eggs in a basket are removed two, three, four, five, or six at a time, there remain, respectively, one, two, three, four, or five eggs. When they are taken out seven at a time, none are left over. Find the smallest number of eggs that could have been contained in the basket. (Brahmagupta, 7th century AD – and many other variations in other cultures)
	
		Show your complete work. You must use the Chinese Remainder Theorem to earn credit. If you use Sage or a calculator, make note where.
		\vfill
	\item (5 pts) Find three consecutive integers each of which is divisible by a square.
		(Hint: Find an integer $a$ such that $2^2\mid a$, $3^2\mid a+1$, and $5^2\mid a+2$.)
		\vfill
		\newpage
	\item (5 pts) Find three consecutive integers such that the first is divisible by a square, the second by a cube, and the third by a fourth power.
		\vfill
	\item (5 pts) Find the solution to the following system of congruences:
			\begin{eqnarray*}11x+5y&\equiv& 7\pmod{20}\\6x+3y&\equiv& 8\pmod{20}\end{eqnarray*}
		
		Show your complete work. You must reference a Theorem to earn credit. If you use Sage or a calculator, make note where.
		\vfill
\end{enumerate}
		
\end{document}