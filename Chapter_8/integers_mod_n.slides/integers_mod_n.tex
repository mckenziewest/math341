\documentclass[t]{beamer}

\subtitle{Integers mod $n$}

\usepackage{amsthm,amsmath,amsfonts,hyperref,graphicx,color,multicol,soul}
\usepackage{enumitem,tikz,tikz-cd,setspace,mathtools}

%%%%%%%%%%
%Beamer Template Customization
%%%%%%%%%%
\setbeamertemplate{navigation symbols}{}
\setbeamertemplate{theorems}[ams style]
\setbeamertemplate{blocks}[rounded]

\definecolor{Blu}{RGB}{43,62,133} % UWEC Blue
\setbeamercolor{structure}{fg=Blu} % Titles

%Unnumbered footnotes:
\newcommand{\blfootnote}[1]{%
	\begingroup
	\renewcommand\thefootnote{}\footnote{#1}%
	\addtocounter{footnote}{-1}%
	\endgroup
}

%%%%%%%%%%
%TikZ Stuff
%%%%%%%%%%
\usetikzlibrary{arrows}
\usetikzlibrary{shapes.geometric}
\tikzset{
	smaller/.style={
		draw,
		regular polygon,
		regular polygon sides=3,
		fill=white,
		node distance=2cm,
		minimum height=1in,
		line width = 2pt
	}
}
\tikzset{
	smsquare/.style={
		draw,
		regular polygon,
		regular polygon sides=4,
		fill=white,
		node distance=2cm,
		minimum height=1in,
		line width = 2pt
	}
}


%%%%%%%%%%
%Custom Commands
%%%%%%%%%%

\newcommand{\C}{\mathbb{C}}
\newcommand{\quats}{\mathbb{H}}
\newcommand{\N}{\mathbb{N}}
\newcommand{\Q}{\mathbb{Q}}
\newcommand{\R}{\mathbb{R}}
\newcommand{\Z}{\mathbb{Z}}

\newcommand{\ds}{\displaystyle}

\newcommand{\fn}{\insertframenumber}

\newcommand{\id}{\operatorname{id}}
\newcommand{\im}{\operatorname{im}}
\newcommand{\Aut}{\operatorname{Aut}}
\newcommand{\Inn}{\operatorname{Inn}}

\newcommand{\blank}[1]{\underline{\hspace*{#1}}}

\newcommand{\abar}{\overline{a}}
\newcommand{\bbar}{\overline{b}}
\newcommand{\cbar}{\overline{c}}

\newcommand{\nml}{\unlhd}

%%%%%%%%%%
%Custom Theorem Environments
%%%%%%%%%%
\theoremstyle{definition}
\newtheorem{exercise}{Exercise}
\newtheorem{question}[exercise]{Question}
\newtheorem{warmup}{Warm-Up}
\newtheorem*{defn}{Definition}
\newtheorem*{exa}{Example}
\newtheorem*{disc}{Group Discussion}
\newtheorem*{nb}{Note}
\newtheorem*{recall}{Recall}
\renewcommand{\emph}[1]{{\color{blue}\texttt{#1}}}

\definecolor{Gold}{RGB}{237, 172, 26}
%Statement block
\newenvironment{statementblock}[1]{%
	\setbeamercolor{block body}{bg=Gold!20}
	\setbeamercolor{block title}{bg=Gold}
	\begin{block}{\textbf{#1.}}}{\end{block}}
\newenvironment{thm}[1]{%
	\setbeamercolor{block body}{bg=Gold!20}
	\setbeamercolor{block title}{bg=Gold}
	\begin{block}{\textbf{Theorem #1.}}}{\end{block}}


%%%%%%%%%%
%Custom Environment Wrappers
%%%%%%%%%%
\newcommand{\enumarabic}[1]{
	\begin{enumerate}[label=\textbf{\arabic*.}]
		#1
	\end{enumerate}
}
\newcommand{\enumalph}[1]{
	\begin{enumerate}[label=(\alph*)]
		#1
	\end{enumerate}
}
\newcommand{\bulletize}[1]{
	\begin{itemize}[label=$\bullet$]
		#1
	\end{itemize}
}
\newcommand{\circtize}[1]{
	\begin{itemize}[label=$\circ$]
		#1
	\end{itemize}
}
\newcommand{\slide}[1]{
	\begin{frame}{\fn}
		#1
	\end{frame}
}
\newcommand{\slidec}[1]{
\begin{frame}[c]{\fn}
	#1
\end{frame}
}
\newcommand{\slidet}[2]{
	\begin{frame}{\fn\ - #1}
		#2
	\end{frame}
}


\newcommand{\startdoc}{
		\title{Math 425: Abstract Algebra 1}
		\author{Mckenzie West}
		\date{Last Updated: \today}
		\begin{frame}
			\maketitle
		\end{frame}
}

\newcommand{\topics}[2]{
	\begin{frame}{\insertframenumber}
		\begin{block}{\textbf{Last Section.}}
			\begin{itemize}[label=--]
				#1
			\end{itemize}
		\end{block}
		\begin{block}{\textbf{This Section.}}
			\begin{itemize}[label=--]
				#2
			\end{itemize}
		\end{block}
	\end{frame}
}

\begin{document} 
	\startdoc
	\topics{
		\item Fermat's Theorem
		\item Pseudoprimes
		\item Wilson's Theorem
	}{
		\item Integers and Arithmetic modulo $n$
		\item Inverses Modulo $n$
	}

\slide{
	\begin{statementblock}{Theorem 1.3.1}
		Congruence modulo $n$ is an equivalence relation on $\Z$.
	\end{statementblock}
}

\slide{
	\begin{defn}
		If $a\in\Z$, then its equivalence class, $[a]$, with respect to congruence modulo $n$ is called its \emph{residue class modulo $n$}.
		
			\[[a]=\{x\in\Z : x\equiv a\pmod n\}.\]
	\end{defn}
}
\slide{
	\begin{defn}
		The \emph{set of integers modulo $n$} is denoted $\Z_n$ and is given by
			\[\Z_n=\{[0],[1],[2],\dots,[n-1]\}.\]
	\end{defn}
	\begin{exa}
		$\Z_7=$
	\end{exa}
	\begin{exercise}
		Which of $[0],\dots,[6]$ equals $[47]$ in $\Z_7$? What about $[-16]$?
	\end{exercise}
}
\slide{
	\begin{statementblock}{Theorem}
		If $r$ is the remainder when dividing $a$ by $n$, then in $\Z_n$, we have $$[a]=[r].$$
	\end{statementblock}
}
\slide{
	\begin{statementblock}{Claim}
		Addition and multiplication in $\Z_n$, as defined below, are well-defined:
			\begin{itemize}
				\item[(1)] $[a]+[b]=[a+b]$
				\item[(2)] $[a][b]=[ab]$
			\end{itemize}
	\end{statementblock}
	\begin{nb}
		The important point here is that any well-defined arithmetic operation on $\Z_n$ should NOT depend on the choice of residue class representative.
	\end{nb}
	\begin{exa}
		In $\Z_7$, $[48]=[6]$ and $[3]=[10]$, so $[48]+[3]=[6]+[10]$.
	\end{exa}
}
\slide{
	\begin{exercise}
		Fill out the addition and multiplication tables for $\Z_4$.
		\Large
		$$
		\begin{array}{c|c|c|c|c}
		+_4&[0]&[1]&[2]&[3]\\\hline
		[0]&&&&\\\hline
		[1]&&&&\\\hline
		[2]&&&&\\\hline
		[3]&&&&
		\end{array}
		\hskip 3em
		\begin{array}{c|c|c|c|c}
		\times_4&[0]&[1]&[2]&[3]\\\hline
		[0]&&&&\\\hline
		[1]&&&&\\\hline
		[2]&&&&\\\hline
		[3]&&&&
		\end{array}$$
	\end{exercise}
}
\slide{
	\begin{thm}{(Algebraic Properties of $\Z_n$)}
		Let $n\geq 2$ be a fixed modulus and let $a,b$ and $c$ denote arbitrary integers. Then the following hold in $\Z_n$.
		\enumarabic{\item $[a]+[b]=[b]+[a]$ and $[a][b]=[b][a]$.
			\item $[a]+([b]+[c])=([a]+[b])+[c]$ and $[a]([b][c])=([a][b])[c]$.
			\item $[a]+[0]=[a]$ and $[a][1]=[a]$.
			\item $[a]+[-a]=[0]$.
			\item $[a]([b]+[c])=[a][b]+[a][c]$.
		}
	\end{thm}
}
\slide{
	\begin{defn}
		We call a class $[a]\in\Z_n$ \emph{invertible} if there is some $[b]\in Z_n$ such that $[a][b]=[1]$.
	\end{defn}
	\begin{exa}
		Consider $\Z_4$.
	\end{exa}
}
\slide{
	\begin{recall}
		For all $a\in\Z$, the equation $ax\equiv b\pmod n$ has a solution if and only if $\gcd(a,n)\mid b$.
	\end{recall}
	\begin{exercise}
		Show $[6]\in\Z_8$ has no multiplicative inverse.
	\end{exercise}
}
\slide{
	\begin{thm}{(Inverses mod primes)}
		If $p$ is prime, then all of $[1],[2],\dots,[p-1]$ are invertible in $\Z_p$.
	\end{thm}
}
\slide{
	\begin{thm}{(Inverses mod $n$)}
		For integers $n\geq 2$, the class $[a]$ is inveritble in $\Z_n$ if and only if $\gcd(a,n)=1$.
	\end{thm}
}
\slide{
	\begin{question}
		What do you notice about the relationship between $n$ and the values in $\Z_n$ that have inverses?
		
		This slide and the next have multiplication tables for $\Z_7$, $\Z_8$, $\Z_9$, and $\Z_{10}$.  Identify the rows that have a 1 in them - these are the classes with inverses.
	\end{question}
	\vskip 1em
	\begin{minipage}{.45\textwidth}\small
		Multiplication in $\Z_7$
		
		$\begin{array}{r|rrrrrrr}
			\times & 0 & 1 & 2 & 3 & 4 & 5 & 6 \\\hline
			0 & 0 & 0 & 0 & 0 & 0 & 0 & 0 \\
			1 & 0 & 1 & 2 & 3 & 4 & 5 & 6 \\
			2 & 0 & 2 & 4 & 6 & 1 & 3 & 5 \\
			3 & 0 & 3 & 6 & 2 & 5 & 1 & 4 \\
			4 & 0 & 4 & 1 & 5 & 2 & 6 & 3 \\
			5 & 0 & 5 & 3 & 1 & 6 & 4 & 2 \\
			6 & 0 & 6 & 5 & 4 & 3 & 2 & 1
		\end{array}$
	\end{minipage}
	\hskip 2em
	\begin{minipage}{.45\textwidth}\small
		Multiplication in $\Z_8$
		
		$
		\begin{array}{r|rrrrrrrr}
			\times & 0 & 1 & 2 & 3 & 4 & 5 & 6 & 7 \\\hline
			0 & 0 & 0 & 0 & 0 & 0 & 0 & 0 & 0 \\
			1 & 0 & 1 & 2 & 3 & 4 & 5 & 6 & 7 \\
			2 & 0 & 2 & 4 & 6 & 0 & 2 & 4 & 6 \\
			3 & 0 & 3 & 6 & 1 & 4 & 7 & 2 & 5 \\
			4 & 0 & 4 & 0 & 4 & 0 & 4 & 0 & 4 \\
			5 & 0 & 5 & 2 & 7 & 4 & 1 & 6 & 3 \\
			6 & 0 & 6 & 4 & 2 & 0 & 6 & 4 & 2 \\
			7 & 0 & 7 & 6 & 5 & 4 & 3 & 2 & 1
		\end{array}
		$
	\end{minipage}
	
	{\small * Brackets omitted for the sake of visual appearance.}
}
\slide{\hskip -1em
	\begin{minipage}{.45\textwidth}
		\small
		Multiplication in $\Z_9$
		
		$
		\begin{array}{r|rrrrrrrrr}
			\times & 0 & 1 & 2 & 3 & 4 & 5 & 6 & 7 & 8 \\\hline
			0 & 0 & 0 & 0 & 0 & 0 & 0 & 0 & 0 & 0 \\
			1 & 0 & 1 & 2 & 3 & 4 & 5 & 6 & 7 & 8 \\
			2 & 0 & 2 & 4 & 6 & 8 & 1 & 3 & 5 & 7 \\
			3 & 0 & 3 & 6 & 0 & 3 & 6 & 0 & 3 & 6 \\
			4 & 0 & 4 & 8 & 3 & 7 & 2 & 6 & 1 & 5 \\
			5 & 0 & 5 & 1 & 6 & 2 & 7 & 3 & 8 & 4 \\
			6 & 0 & 6 & 3 & 0 & 6 & 3 & 0 & 6 & 3 \\
			7 & 0 & 7 & 5 & 3 & 1 & 8 & 6 & 4 & 2 \\
			8 & 0 & 8 & 7 & 6 & 5 & 4 & 3 & 2 & 1
		\end{array}
		$
	\end{minipage}
	\hskip 2em
	\begin{minipage}{.45\textwidth}\small
		Multiplication in $\Z_{10}$
		
		$\begin{array}{r|rrrrrrrrrr}
			0 & 0 & 1 & 2 & 3 & 4 & 5 & 6 & 7 & 8 & 9 \\\hline
			0 & 0 & 0 & 0 & 0 & 0 & 0 & 0 & 0 & 0 & 0 \\
			1 & 0 & 1 & 2 & 3 & 4 & 5 & 6 & 7 & 8 & 9 \\
			2 & 0 & 2 & 4 & 6 & 8 & 0 & 2 & 4 & 6 & 8 \\
			3 & 0 & 3 & 6 & 9 & 2 & 5 & 8 & 1 & 4 & 7 \\
			4 & 0 & 4 & 8 & 2 & 6 & 0 & 4 & 8 & 2 & 6 \\
			5 & 0 & 5 & 0 & 5 & 0 & 5 & 0 & 5 & 0 & 5 \\
			6 & 0 & 6 & 2 & 8 & 4 & 0 & 6 & 2 & 8 & 4 \\
			7 & 0 & 7 & 4 & 1 & 8 & 5 & 2 & 9 & 6 & 3 \\
			8 & 0 & 8 & 6 & 4 & 2 & 0 & 8 & 6 & 4 & 2 \\
			9 & 0 & 9 & 8 & 7 & 6 & 5 & 4 & 3 & 2 & 1
		\end{array}$
	\end{minipage}
	* Brackets omitted for the sake of visual appearance.
}
\slide{
	\begin{nb}
		Recall this technique for finding the inverse.  
	\end{nb}
	\begin{exa}
		Find the inverse of $[16]$ in $\Z_{35}$.
		
		\fbox{\begin{minipage}{2in}
				Euclidean Algorithm:
				
				$\begin{array}{rcl}
					35&=&2(16)+3\\
					16&=&5(3)+1\\
					3&=&3(1)+0
				\end{array}$
		\end{minipage}}
		\fbox{\begin{minipage}{2in}
				
				B\'ezout:
				
				$\begin{array}{rcl}
					1&=&16-5(3)\\
					&=&16-5(35-2(16))\\
					&=&11(16)-5(35)
				\end{array}$
		\end{minipage}}
		\vskip 1em
		The equation $1=11(16)-5(35)$ modulo 35 gives:
		\[1\equiv 11\cdot 16\pmod{35}.\]
		Therefore, the multiplicative inverse of $[16]$ in $\Z_{35}$ is $[11]$.
	\end{exa}
}
\slide{
	\begin{exercise}
		Solve the equation $[16]x=[9],$ in $\Z_{35}$.
	\end{exercise}
}
\slide{
	\begin{exercise}
		Solve the system of equations in $\Z_{13}$
		\[\begin{cases}
			[5]x+[2]y=[1]\\
			[2]x+[10]y=[2].
		\end{cases}\]
	\end{exercise}
}
\slide{
	\begin{statementblock}{(Zero divisors)}
		The following are equivalent for any integer $n\geq 2$.
		\enumarabic{\item Every element $[a]\neq[0]$ in $\Z_n$ has a multiplicative inverse.
			\item If $[a][b]=[0]$ in $\Z_n$, then either $[a]=[0]$ or $[b]=[0]$.
			\item The integer $n$ is prime.
		}
	\end{statementblock}
}
\end{document}

