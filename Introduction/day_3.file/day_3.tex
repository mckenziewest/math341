\documentclass[12pt]{article}
\usepackage{amsmath,amsthm,amssymb,hyperref,fancyhdr,color}
\usepackage[margin=.75in]{geometry}

\newcommand{\R}{\mathbb{R}}
\newcommand{\Z}{\mathbb{Z}}


\begin{document}
	\noindent Math 341: Classical Number Theory\hfill Introduction to Number Theory\\
	\mbox{}\hfill Day 3
	
	
	\begin{enumerate}
		\item You will explore the following statement: If $a,b$ are integers such that both $a$ and $b$ have the same remainder when divided by some integer $d$, then $d$ divides $a-b$ evenly.
		
		\begin{enumerate}
			\item Consider $a=51$, $b=16$, and $d=7$.  
				\begin{enumerate}
					\item What is the remainder when dividing 51 or 16 by 7?
					\item Using the remainder from the last part in the second box,
						$$51=\fbox{\color{white}ABC}\cdot7+\fbox{\color{white}ABC}$$
						$$16=\fbox{\color{white}ABC}\cdot7+\fbox{\color{white}ABC}$$
					\item Using these last two lines, write
						$$51-16=(\fbox{\color{white}ABC}-\fbox{\color{white}ABC})\cdot 7.$$
				\end{enumerate}
			\item Now we'll do the same in general.
				
				Suppose $a$ and $b$ have the same remainder when divided by $d$.  This means we can write
						$$a=qd+r\quad\text{and}\quad b=sd+r,$$ 
					for some integers $q$ and $s$.
				
				\begin{enumerate}
					\item Simplify $a-b$.\vskip 1in
					\item Why does this mean $d$ divides $a-b$?\vskip 1in
				\end{enumerate}
		\end{enumerate}
		\item The next statement we need is that of the Pigeon Hole Principle.
		
			Let $m$ and $n$ be positive integers with $m<n$. If we place $n$ objects into $n$ containers then one of those containers must contain more than one object.
			
			\begin{enumerate}
				\item Use the Pigeon Hole Principle to explain why if you have 1000 people in the same room then at least two of them will have the same birthday.\vfill
				\item As written, does the PHP say anything else about duplicate birthdays?\vfill
			\end{enumerate}
		\newpage
		\item We will use the remainder and PHP results to verify the following:
			\begin{quote}
				Consider the following sequence of integers with repeating digits
					$$1,11,111,1111,11111,\dots.$$
				There is at least one term in this sequence divisible by 2023.
			\end{quote}
			For simplicity, we're going to write
				$$y_1=1, y_2=11,y_3=111,y_4=1111,y_5=11111,\dots.$$
			We will also use $r_1,r_2,r_3,r_4,r_5,\dots$ to be the remainder when we divide the corresponding $y$-value by 2023.
			\begin{enumerate}
				\item To verify you are following, compute some $r$-values:
				
					$$\begin{array}{lllllll}
						r_1=1&\hspace*{3em}&r_2=11&\hspace*{3em}&r_3=111&\hspace*{3em}&r_4=1111\\ \\
						r_5=996&&r_6=&&r_7=&&r_8=
					\end{array}$$
				
					Hint: One quick way to compute a remainder is to use Google.  To compute $r_5$, I typed \texttt{11111\%2023} into the search bar.  The \texttt{\%} is the \emph{modulo} operator.  It computes the remainder when dividing the number before by the number after.
					
					Do a sanity check: what is $(11111-996)/2023$?
				\item Think for a moment, what are the possible values for remainders?  What's the smallest a remainder can be? What's the biggest a remainder can be?
						\vskip 1in
				\item What does it mean if the remainder is 0?
					\vskip 1in
				\item How many possible remainders are there?
					\vskip 1in
				\item Use the PHP to argue that there must be at least two equal $r$-values.
					\vskip 1in
				\item We can now assume that we know $j$ and $k$ such that $r_j=r_k$.
					That is we know $y_j$ and $y_k$ have the same remainder when divided by 2023.
					What does the divisibility property we started with say about $y_j-y_k$?
					\vskip 1in
				\item Now let's explore what $y_j-y_k$ could be.  Compute the following and look for a pattern: \\\\
					$\begin{array}{l}
						y_5-y_3=\\\\
						y_7-y_2=\\\\
						y_{10}-y_{6}=\\
					\end{array}$
				\item Explain why (asuming $j>k$)
					$$y_j-y_k=10^j y_{k-j}$$
					\vskip 1in
				\item The divisibility property should tell us that $y_j-y_k$ is divisible by 2023. So now use the equality in the last part to explain why $y_{j-k}$ is divisible by $2023$.
			\end{enumerate}
			
			
\end{enumerate}
\end{document}