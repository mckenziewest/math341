\documentclass[12pt]{article}
\usepackage{amsmath,amsthm,amssymb,hyperref,fancyhdr,pgffor}
\usepackage[margin=.75in]{geometry}

\def\version{1}
\pagestyle{fancy} 
\fancyhf{}\renewcommand{\headrulewidth}{0pt}
\cfoot{\version-\thepage}
\newcommand{\R}{\mathbb{R}}
\newcommand{\Z}{\mathbb{Z}}

\newcommand{\conjone}{		
	\item Goldbach's Conjecture: Every even integer greater than 2 can be written as the sum of two primes.
	\begin{enumerate}
		\item Verify this for $n=4,6,8,10,\dots$, as far as you would like. This might involve 
		using a computer if you're familiar with programming. If not, see what you can do, find a pattern for how you are doing that work, and write down some general instructions for how you're doing it.\vfill
		\item Go online and find the current state of the art for the conjecture, what do we know?\vfill
	\end{enumerate}
}


\newcommand{\conjtwo}{		
	\item Legendre's Conjecture: There is always at least one prime between consecutive perfect squares.
	\begin{enumerate}
		\item Verify that there is a prime number between each of $1,2^2,3^2,4^2,5^2,6^2,\dots$ as far as you wish. This might involve 
		using a computer if you're familiar with programming. If not, see what you can do, find a pattern for how you are doing that work, and write down some general instructions for how you're doing it.\vfill
		\item Go online and find the current state of the art for the conjecture, what do we know?\vfill
	\end{enumerate}
}

\newcommand{\conjthree}{
	\item Landau's Conjecture: There are infinitely many primes $p$ such that $p-1$ is a perfect square.
	\begin{enumerate}
		\item Find some primes that satisfy this conjecture. This might involve 
		using a computer if you're familiar with programming. If not, see what you can do, find a pattern for how you are doing that work, and write down some general instructions for how you're doing it.\vfill
		\item Go online and find the current state of the art for the conjecture, what do we know?\vfill
	\end{enumerate}
}


\begin{document}
	\foreach\version in {1,2,3}{
		\setcounter{page}{1}
	\noindent Math 341: Classical Number Theory\hfill Introduction to Number Theory\\
	\mbox{}\hfill Day 2
	
	
	\begin{enumerate}
		\item Sums of Two Squares: Some numbers can be written as the sum of two squares. For example, $5=2^2+1^2$ and $4=2^2+0^2$.
		\begin{enumerate}
			\item Here is the full list of positive integers less than 100 that are sums of two squares. Verify a couple of these.  
			\begin{center}
				\begin{tabular}{ccccccccccc}
					1&2&4&5&8&9&10&13&16&17&18\\
					20&25&26&29&32&34&36&37&40&41&45\\
					49&50&52&53&58&61&64&65&68&72&73\\
					74&80&81&82&85&89&90&97&98
				\end{tabular}
			\end{center}
			\item Can you find any that can be written as the sum of two squares in more than one way?\vfill
			\item What is special about the odd prime numbers in this list?\vfill
		\end{enumerate}
	\newpage
\if\version1
	\conjone
\else
\if\version2
	\conjtwo
\else
	\conjthree
\fi\fi
		\item Erd\H os--Straus Conjecture: For all $n\geq 2$, the rational number $\frac{4}{n}$ can be expressed as the sum of three positive unit fractions (fractions of the form $\frac{1}{k}$). For example, when $n=5$, $\frac{4}{5}=\frac{1}{2}+\frac{1}{5}+\frac{1}{10}$.
		\begin{enumerate}
			\item Try to verify this conjecture for a few different values of $n$. (I think this one might be difficult.) This might involve 
			using a computer if you're familiar with programming. If not, see what you can do, find a pattern for how you are doing that work, and write down some general instructions for how you're doing it.\vfill
			\item Go online and find the current state of the art for the conjecture, what do we know?\vfill
		\end{enumerate}
\newpage
\if\version1
\conjtwo\conjthree
\else
\if\version2
\conjone\conjthree
\else
\conjone\conjtwo
\fi\fi
\newpage
	\item Brocard's Conjecture: There are at least four prime numbers between $(p_n)^2$ and $(p_{n+1})^2$, for all $n\geq 2$, where $p_n$ is the $n$th prime number.
	\begin{enumerate}
		\item Verify this conjecture for $n=2,3,\dots$.
		\vfill 
		\item Does anything surprise you about this conjecture?\vfill 
		\item What can you find online about this conjecture?\vfill
	\end{enumerate}
\end{enumerate}
\newpage
}
\end{document}