\documentclass[t]{beamer}

\subtitle{Euler's Phi Function and Multiplicative Functions}

\usepackage{amsthm,amsmath,amsfonts,hyperref,graphicx,color,multicol,soul}
\usepackage{enumitem,tikz,tikz-cd,setspace,mathtools}

%%%%%%%%%%
%Beamer Template Customization
%%%%%%%%%%
\setbeamertemplate{navigation symbols}{}
\setbeamertemplate{theorems}[ams style]
\setbeamertemplate{blocks}[rounded]

\definecolor{Blu}{RGB}{43,62,133} % UWEC Blue
\setbeamercolor{structure}{fg=Blu} % Titles

%Unnumbered footnotes:
\newcommand{\blfootnote}[1]{%
	\begingroup
	\renewcommand\thefootnote{}\footnote{#1}%
	\addtocounter{footnote}{-1}%
	\endgroup
}

%%%%%%%%%%
%TikZ Stuff
%%%%%%%%%%
\usetikzlibrary{arrows}
\usetikzlibrary{shapes.geometric}
\tikzset{
	smaller/.style={
		draw,
		regular polygon,
		regular polygon sides=3,
		fill=white,
		node distance=2cm,
		minimum height=1in,
		line width = 2pt
	}
}
\tikzset{
	smsquare/.style={
		draw,
		regular polygon,
		regular polygon sides=4,
		fill=white,
		node distance=2cm,
		minimum height=1in,
		line width = 2pt
	}
}


%%%%%%%%%%
%Custom Commands
%%%%%%%%%%

\newcommand{\C}{\mathbb{C}}
\newcommand{\quats}{\mathbb{H}}
\newcommand{\N}{\mathbb{N}}
\newcommand{\Q}{\mathbb{Q}}
\newcommand{\R}{\mathbb{R}}
\newcommand{\Z}{\mathbb{Z}}

\newcommand{\ds}{\displaystyle}

\newcommand{\fn}{\insertframenumber}

\newcommand{\id}{\operatorname{id}}
\newcommand{\im}{\operatorname{im}}
\newcommand{\Aut}{\operatorname{Aut}}
\newcommand{\Inn}{\operatorname{Inn}}

\newcommand{\blank}[1]{\underline{\hspace*{#1}}}

\newcommand{\abar}{\overline{a}}
\newcommand{\bbar}{\overline{b}}
\newcommand{\cbar}{\overline{c}}

\newcommand{\nml}{\unlhd}

%%%%%%%%%%
%Custom Theorem Environments
%%%%%%%%%%
\theoremstyle{definition}
\newtheorem{exercise}{Exercise}
\newtheorem{question}[exercise]{Question}
\newtheorem{warmup}{Warm-Up}
\newtheorem*{defn}{Definition}
\newtheorem*{exa}{Example}
\newtheorem*{disc}{Group Discussion}
\newtheorem*{nb}{Note}
\newtheorem*{recall}{Recall}
\renewcommand{\emph}[1]{{\color{blue}\texttt{#1}}}

\definecolor{Gold}{RGB}{237, 172, 26}
%Statement block
\newenvironment{statementblock}[1]{%
	\setbeamercolor{block body}{bg=Gold!20}
	\setbeamercolor{block title}{bg=Gold}
	\begin{block}{\textbf{#1.}}}{\end{block}}
\newenvironment{thm}[1]{%
	\setbeamercolor{block body}{bg=Gold!20}
	\setbeamercolor{block title}{bg=Gold}
	\begin{block}{\textbf{Theorem #1.}}}{\end{block}}


%%%%%%%%%%
%Custom Environment Wrappers
%%%%%%%%%%
\newcommand{\enumarabic}[1]{
	\begin{enumerate}[label=\textbf{\arabic*.}]
		#1
	\end{enumerate}
}
\newcommand{\enumalph}[1]{
	\begin{enumerate}[label=(\alph*)]
		#1
	\end{enumerate}
}
\newcommand{\bulletize}[1]{
	\begin{itemize}[label=$\bullet$]
		#1
	\end{itemize}
}
\newcommand{\circtize}[1]{
	\begin{itemize}[label=$\circ$]
		#1
	\end{itemize}
}
\newcommand{\slide}[1]{
	\begin{frame}{\fn}
		#1
	\end{frame}
}
\newcommand{\slidec}[1]{
\begin{frame}[c]{\fn}
	#1
\end{frame}
}
\newcommand{\slidet}[2]{
	\begin{frame}{\fn\ - #1}
		#2
	\end{frame}
}


\newcommand{\startdoc}{
		\title{Math 425: Abstract Algebra 1}
		\author{Mckenzie West}
		\date{Last Updated: \today}
		\begin{frame}
			\maketitle
		\end{frame}
}

\newcommand{\topics}[2]{
	\begin{frame}{\insertframenumber}
		\begin{block}{\textbf{Last Section.}}
			\begin{itemize}[label=--]
				#1
			\end{itemize}
		\end{block}
		\begin{block}{\textbf{This Section.}}
			\begin{itemize}[label=--]
				#2
			\end{itemize}
		\end{block}
	\end{frame}
}

\begin{document} 
	\startdoc
	\topics{
		\item Integers and Arithmetic modulo $n$
		\item Inverses Modulo $n$
	}{
		\item Units mod $n$
		\item Euler's phi function
	}

\slide{
	\begin{defn}
		We call a class $[a]\in\Z_n$ \emph{invertible} or a \emph{unit} if there is some $[b]\in\Z_n$ such that $[a][b]=[1]$.
	\end{defn}
	\begin{recall}
		A class $[a]\in\Z_n$ is invertible if and only if $\gcd(a,n)=1$.
	\end{recall}
}
\slide{
	\begin{defn}
		The \emph{unit group modulo $n$} is the set of classes
			$$U_n=\{[a]\in\Z_n \ :\ \gcd(a,n)=1\}.$$
	\end{defn}
	\begin{exercise}
		What are $U_4$, $U_5$, $U_6$, and $U_7$?
	\end{exercise}
}
\slide{
	\begin{defn}
		The \emph{Euler Phi-Function} is defined by	
			\[\varphi(n)=|\{a\in\N\ |\ \gcd(a,n)=1\text{ and }a\leq n\}|.\]
		That is $\varphi(n)$ is the number of positive integers less than or equal to $n$ that are relatively prime to $n$.
	\end{defn}
	\begin{statementblock}{Number of Units}
		For positive integers $n$, we have $|U_n|=\varphi(n)$.
	\end{statementblock}
	\begin{exercise}
		Compute $\varphi(n)$ for $n=1,2,3,4,5,6,7$.
	\end{exercise}
}
\slide{
	\begin{exercise}
		If $p$ is prime, what is $\varphi(p)$?\vskip .5in
	\end{exercise}
}
\slide{
	\begin{exercise}
		Consider $n=125=5^3$, how might you compute $\varphi(5^3)$?  
		
		(Hint: It will be easier to count all integers $a$ with $\gcd(5^3,a)>1$ and subtract that value from 125.)
	\end{exercise}
}
\slide{
	\begin{exercise}
		If $p$ is prime and $k\geq1$, what is $\varphi(p^k)$?
	\end{exercise}
}
\slide{
	\begin{statementblock}{Theorem 7.1}
		If $p$ is prime and $k>0$, then	
			\[\varphi(p^k)=p^k-p^{k-1}=p^k\left(1-\frac{1}{p}\right).\]
	\end{statementblock}
	\begin{exercise}
		Compute $\varphi(n)$ for $n=27, 49, 243, 343, 625$.
	\end{exercise}
}
\slide{
	\begin{statementblock}{Theorem 7.2}
		The function $\varphi$ is multiplicative. That is, for all positive integers $m,n$ with $\gcd(m,n)=1$, we have $\varphi(mn)=\varphi(m)\varphi(n)$.
	\end{statementblock}
	\begin{exercise}
		Use this Theorem and the previous one to compute $\varphi(n)$ for $n=24,36,100000$. 
	\end{exercise}
}
\slide{
	\begin{thm}{7.3}
		If $n=p_1^{k_1}p_2^{k_2}\cdots p_r^{k_r}$ is the prime factorization of $n>1$, then 
		$\varphi(n)=n\left(1-\frac{1}{p_1}\right)\left(1-\frac{1}{p_2}\right)\cdots\left(1-\frac{1}{p_r}\right)$.
	\end{thm}
}
\slide{
	\begin{thm}{7.4}
		For $n>2$, we have $\varphi(n)$ is an even integer.
	\end{thm}
}
\slide{
	\begin{defn}
		Any function who's domain is the set of positive integers is called a \emph{number theoretic function}
	\end{defn}
	\exam{
		\bulletize{
			\item $\varphi(n)$
			\item $\tau(n)=$ number of positive divisors of $n$
			\item $\sigma(n)=$ sum of positive divisors of $n$
			\item $\mu(n)=\begin{cases}
				1&\text{if }n=1\\
				0&\text{if }p^2|n\text{ for some prime } p\\
				(-1)^r&\text{if }n=p_1p_2\cdots p_r\text{ where the }p_i\text{ are distinct primes}
			\end{cases}$
		}
	}
}
\slide{
	\begin{thm}{6.2}
		If $n=p_1^{k_1}p_2^{k_2}\cdots p_r^{k_r}$ is the prime factorization of $n>1$, then 
		\vskip .25in
		\enumalph{
			\item $\tau(n)=(k_1+1)(k_2+1)\cdots(k_r+1)$\vskip .25in
			\item $\sigma(n)=\frac{p_1^{k_1+1}-1}{p_1-1}\frac{p_2^{k_2+1}-1}{p_2-1}\cdots \frac{p_r^{k_r+1}-1}{p_r-1}$
		}
	\end{thm}
	\begin{thm}{6.3}
		The functions $\sigma$ and $\tau$ are multiplicative.
	\end{thm}
	\begin{thm}{6.5}
		The function $\mu$ is multiplicative.
	\end{thm}
}
\slide{
	\begin{exercise}
		Liouville's lambda function $\lambda(n)$ is defined by factoring $n$ into products of primes $n=p_1^{e_1}p_2^{e_2}\cdots p_r^{e_r}$ and setting
			\[\lambda(n)=(-1)^{e_1+e_2+\cdots +e_r}.\]
		(Let $\lambda(1)=1$.)
		\enumalph{
			\item Compute $\lambda(n)$ for $n=30,504,60750$.\vskip .5in
			\item Verify that if $\gcd(m,n)=1$, then $\lambda(mn)=\lambda(m)\lambda(n)$.
		}
	\end{exercise}
}
\slide{
	\begin{exercise}
		\enumalph{
			\item[(c)] Define
				\[G(n)=\sum_{d|n}\lambda(d).\]
				Compute the values of $G(n)$ for $n=1,2,3\dots,18$.\vskip 1in
			\item[(d)] Make a conjecture about the value of $G(n)$.
		}
	\end{exercise}
}
\end{document}

