\documentclass[t]{beamer}

\subtitle{Euler's Theorem and Properties of $\varphi(n)$}

\usepackage{amsthm,amsmath,amsfonts,hyperref,graphicx,color,multicol,soul}
\usepackage{enumitem,tikz,tikz-cd,setspace,mathtools}

%%%%%%%%%%
%Beamer Template Customization
%%%%%%%%%%
\setbeamertemplate{navigation symbols}{}
\setbeamertemplate{theorems}[ams style]
\setbeamertemplate{blocks}[rounded]

\definecolor{Blu}{RGB}{43,62,133} % UWEC Blue
\setbeamercolor{structure}{fg=Blu} % Titles

%Unnumbered footnotes:
\newcommand{\blfootnote}[1]{%
	\begingroup
	\renewcommand\thefootnote{}\footnote{#1}%
	\addtocounter{footnote}{-1}%
	\endgroup
}

%%%%%%%%%%
%TikZ Stuff
%%%%%%%%%%
\usetikzlibrary{arrows}
\usetikzlibrary{shapes.geometric}
\tikzset{
	smaller/.style={
		draw,
		regular polygon,
		regular polygon sides=3,
		fill=white,
		node distance=2cm,
		minimum height=1in,
		line width = 2pt
	}
}
\tikzset{
	smsquare/.style={
		draw,
		regular polygon,
		regular polygon sides=4,
		fill=white,
		node distance=2cm,
		minimum height=1in,
		line width = 2pt
	}
}


%%%%%%%%%%
%Custom Commands
%%%%%%%%%%

\newcommand{\C}{\mathbb{C}}
\newcommand{\quats}{\mathbb{H}}
\newcommand{\N}{\mathbb{N}}
\newcommand{\Q}{\mathbb{Q}}
\newcommand{\R}{\mathbb{R}}
\newcommand{\Z}{\mathbb{Z}}

\newcommand{\ds}{\displaystyle}

\newcommand{\fn}{\insertframenumber}

\newcommand{\id}{\operatorname{id}}
\newcommand{\im}{\operatorname{im}}
\newcommand{\lcm}{\operatorname{lcm}}
\newcommand{\Aut}{\operatorname{Aut}}
\newcommand{\Inn}{\operatorname{Inn}}

\newcommand{\blank}[1]{\underline{\hspace*{#1}}}

\newcommand{\abar}{\overline{a}}
\newcommand{\bbar}{\overline{b}}
\newcommand{\cbar}{\overline{c}}

\newcommand{\nml}{\unlhd}

%%%%%%%%%%
%Custom Theorem Environments
%%%%%%%%%%
\theoremstyle{definition}
\newtheorem{exercise}{Exercise}
\newtheorem{question}[exercise]{Question}
\newtheorem{warmup}{Warm-Up}
\newtheorem*{exa}{Example}
\newtheorem*{disc}{Group Discussion}
\newtheorem*{recall}{Recall}
\renewcommand{\emph}[1]{{\color{blue}\texttt{#1}}}

\definecolor{Gold}{RGB}{237, 172, 26}
%Statement block
\newenvironment{statementblock}[1]{%
	\setbeamercolor{block body}{bg=Gold!20}
	\setbeamercolor{block title}{bg=Gold}
	\begin{block}{\textbf{#1.}}}{\end{block}}
\newenvironment{goldblock}{%
	\setbeamercolor{block body}{bg=Gold!20}
	\setbeamercolor{block title}{bg=Gold}
	\setbeamertemplate{blocks}[shadow=true]
	\begin{block}{}}{\end{block}}
\newenvironment{defn}{%
	\setbeamercolor{block body}{bg=gray!20}
	\setbeamercolor{block title}{bg=violet, fg=white}
	\setbeamertemplate{blocks}[shadow=true]
	\begin{block}{\textbf{Definition.}}}{\end{block}}
\newenvironment{nb}{%
	\setbeamercolor{block body}{bg=gray!20}
	\setbeamercolor{block title}{bg=teal, fg=white}
	\setbeamertemplate{blocks}[shadow=true]
	\begin{block}{\textbf{Note.}}}{\end{block}}
\newenvironment{blockexample}{%
	\setbeamercolor{block body}{bg=gray!20}
	\setbeamercolor{block title}{bg=Blu, fg=white}
	\setbeamertemplate{blocks}[shadow=true]
	\begin{block}{\textbf{Example.}}}{\end{block}}
\newenvironment{blocknonexample}{%
	\setbeamercolor{block body}{bg=gray!20}
	\setbeamercolor{block title}{bg=purple, fg=white}
	\setbeamertemplate{blocks}[shadow=true]
	\begin{block}{\textbf{Non-Example.}}}{\end{block}}
\newenvironment{thm}[1]{%
	\setbeamercolor{block body}{bg=Gold!20}
	\setbeamercolor{block title}{bg=Gold}
	\begin{block}{\textbf{Theorem #1.}}}{\end{block}}


%%%%%%%%%%
%Custom Environment Wrappers
%%%%%%%%%%
\newcommand{\exer}[1]{
	\begin{exercise}
		#1
	\end{exercise}
}
\newcommand{\exam}[1]{
\begin{blockexample}
	#1
\end{blockexample}
}
\newcommand{\nexam}[1]{
\begin{blocknonexample}
	#1
\end{blocknonexample}
}
\newcommand{\enumarabic}[1]{
	\begin{enumerate}[label=\textbf{\arabic*.}]
		#1
	\end{enumerate}
}
\newcommand{\enumalph}[1]{
	\begin{enumerate}[label=(\alph*)]
		#1
	\end{enumerate}
}
\newcommand{\bulletize}[1]{
	\begin{itemize}[label=$\bullet$]
		#1
	\end{itemize}
}
\newcommand{\circtize}[1]{
	\begin{itemize}[label=$\circ$]
		#1
	\end{itemize}
}
\newcommand{\slide}[1]{
	\begin{frame}{\fn}
		#1
	\end{frame}
}
\newcommand{\slidec}[1]{
\begin{frame}[c]{\fn}
	#1
\end{frame}
}
\newcommand{\slidet}[2]{
	\begin{frame}{\fn\ - #1}
		#2
	\end{frame}
}


\newcommand{\startdoc}{
		\title{Math 341: Classical Number Theory}
		\author{Mckenzie West}
		\date{Last Updated: \today}
		\begin{frame}
			\maketitle
		\end{frame}
}

\newcommand{\topics}[2]{
	\begin{frame}[c]{\insertframenumber}
		\begin{block}{\textbf{Last Section.}}
			\begin{itemize}[label=--]
				#1
			\end{itemize}
		\end{block}
		\begin{block}{\textbf{This Section.}}
			\begin{itemize}[label=--]
				#2
			\end{itemize}
		\end{block}
	\end{frame}
}

\begin{document} 
	\startdoc
	\topics{
		\item Euler's phi function
		\item Multiplicative functions
	}{
		\item Euler's theorem
		\item Properties of Euler's phi function
	}

\slide{
	\begin{recall}
	\begin{defn}
		The \emph{Euler Phi-Function} is defined by	
			\[\varphi(n)=|\{a\in\N\ |\ \gcd(a,n)=1,\text{ and }a\leq n\}|.\]
	\end{defn}
	\begin{thm}{7.3}
		If the integer $n>1$ has the prime factorization $n=p_1^{e_1}p_2^{e_2}\cdots p_r^{e_r}$, then
			\begin{eqnarray*}
				\varphi(n)&=&(p_1^{e_1}-p_1^{e_1-1})(p_2^{e_2}-p_2^{e_2-1})\cdots(p_r^{e_r}-p_r^{e_r-1})\\
				&=&n\left(1-\frac{1}{p_1}\right)\left(1-\frac{1}{p_2}\right)\cdots\left(1-\frac{1}{p_r}\right).
			\end{eqnarray*}
	\end{thm}
	\end{recall}
}
\slide{
	\begin{recall}
		\textbf{Fermat's Little Theorem:} If $p$ is prime and $p\nmid a$, then $$a^{p-1}\equiv 1\pmod p.$$
	\end{recall}
	\begin{thm}{7.5 (Euler's Theorem)}
		If $n\geq 1$ and $\gcd(a,n)=1$, then $a^{\varphi(n)}\equiv 1\pmod n$. 
	\end{thm}
}
\slide{
	\begin{exercise}
		Use Euler's Theorem to find the last two digits of $3^{3243}$.  
	\end{exercise}
}
\slide{
	\begin{exercise}
		Show that if $n$ is an odd integer not divisible by 5, then $n$ divides some number for which every digit is 1.
	\end{exercise}
}
\slide{
\begin{statementblock}{Theorem 7.6}
		For every positive integer $n\geq 1$,
		\[n=\sum_{d|n}\varphi(d)\]
	\end{statementblock}
	\begin{exercise}
		Consider the case $n=12$. 
	\end{exercise}
}
\slide{
	\begin{statementblock}{Theorem 7.7}
		For $n>1$, the sum of the positive integers less than $n$ and relatively prime to $n$ is $\frac{1}{2}n\varphi(n)$.
	\end{statementblock}
	\begin{exercise}
		Consider $n=15$.  The $\varphi(15)=8$ numbers in particular are
		\[1,2,4,7,8,11,13,14.\]
		Pair these values up as $a,n-a$.  How many pairs are there?
	\end{exercise}
}
\slide{
	\begin{statementblock}{Theorem 7.8}
		For any positive integer $n$,
		\[\varphi(n)=n\sum_{d|n}\frac{\mu(d)}{d},\]
		where $\mu$ is the M\"obius $\mu$-function:
		$$
		\mu(d)=
		\begin{cases}
			1&\text{if }d=1\\
			0&\text{if }p^2\mid d \text{ for some prime }p\\
			(-1)^r&\text{if }d=p_1p_2\cdots p_r\text{ for distinct primes }p_1,\dots,p_r
		\end{cases}
		$$
	\end{statementblock}
}
\end{document}

