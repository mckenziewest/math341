\documentclass[t]{beamer}

\subtitle{Order of Integers and Primitive Roots}

\usepackage{amsthm,amsmath,amsfonts,hyperref,graphicx,color,multicol,soul}
\usepackage{enumitem,tikz,tikz-cd,setspace,mathtools}

%%%%%%%%%%
%Beamer Template Customization
%%%%%%%%%%
\setbeamertemplate{navigation symbols}{}
\setbeamertemplate{theorems}[ams style]
\setbeamertemplate{blocks}[rounded]

\definecolor{Blu}{RGB}{43,62,133} % UWEC Blue
\setbeamercolor{structure}{fg=Blu} % Titles

%Unnumbered footnotes:
\newcommand{\blfootnote}[1]{%
	\begingroup
	\renewcommand\thefootnote{}\footnote{#1}%
	\addtocounter{footnote}{-1}%
	\endgroup
}

%%%%%%%%%%
%TikZ Stuff
%%%%%%%%%%
\usetikzlibrary{arrows}
\usetikzlibrary{shapes.geometric}
\tikzset{
	smaller/.style={
		draw,
		regular polygon,
		regular polygon sides=3,
		fill=white,
		node distance=2cm,
		minimum height=1in,
		line width = 2pt
	}
}
\tikzset{
	smsquare/.style={
		draw,
		regular polygon,
		regular polygon sides=4,
		fill=white,
		node distance=2cm,
		minimum height=1in,
		line width = 2pt
	}
}


%%%%%%%%%%
%Custom Commands
%%%%%%%%%%

\newcommand{\C}{\mathbb{C}}
\newcommand{\quats}{\mathbb{H}}
\newcommand{\N}{\mathbb{N}}
\newcommand{\Q}{\mathbb{Q}}
\newcommand{\R}{\mathbb{R}}
\newcommand{\Z}{\mathbb{Z}}

\newcommand{\ds}{\displaystyle}

\newcommand{\fn}{\insertframenumber}

\newcommand{\id}{\operatorname{id}}
\newcommand{\im}{\operatorname{im}}
\newcommand{\lcm}{\operatorname{lcm}}
\newcommand{\Aut}{\operatorname{Aut}}
\newcommand{\Inn}{\operatorname{Inn}}

\newcommand{\blank}[1]{\underline{\hspace*{#1}}}

\newcommand{\abar}{\overline{a}}
\newcommand{\bbar}{\overline{b}}
\newcommand{\cbar}{\overline{c}}

\newcommand{\nml}{\unlhd}

%%%%%%%%%%
%Custom Theorem Environments
%%%%%%%%%%
\theoremstyle{definition}
\newtheorem{exercise}{Exercise}
\newtheorem{question}[exercise]{Question}
\newtheorem{warmup}{Warm-Up}
\newtheorem*{exa}{Example}
\newtheorem*{disc}{Group Discussion}
\newtheorem*{recall}{Recall}
\renewcommand{\emph}[1]{{\color{blue}\texttt{#1}}}

\definecolor{Gold}{RGB}{237, 172, 26}
%Statement block
\newenvironment{statementblock}[1]{%
	\setbeamercolor{block body}{bg=Gold!20}
	\setbeamercolor{block title}{bg=Gold}
	\begin{block}{\textbf{#1.}}}{\end{block}}
\newenvironment{goldblock}{%
	\setbeamercolor{block body}{bg=Gold!20}
	\setbeamercolor{block title}{bg=Gold}
	\setbeamertemplate{blocks}[shadow=true]
	\begin{block}{}}{\end{block}}
\newenvironment{defn}{%
	\setbeamercolor{block body}{bg=gray!20}
	\setbeamercolor{block title}{bg=violet, fg=white}
	\setbeamertemplate{blocks}[shadow=true]
	\begin{block}{\textbf{Definition.}}}{\end{block}}
\newenvironment{nb}{%
	\setbeamercolor{block body}{bg=gray!20}
	\setbeamercolor{block title}{bg=teal, fg=white}
	\setbeamertemplate{blocks}[shadow=true]
	\begin{block}{\textbf{Note.}}}{\end{block}}
\newenvironment{blockexample}{%
	\setbeamercolor{block body}{bg=gray!20}
	\setbeamercolor{block title}{bg=Blu, fg=white}
	\setbeamertemplate{blocks}[shadow=true]
	\begin{block}{\textbf{Example.}}}{\end{block}}
\newenvironment{blocknonexample}{%
	\setbeamercolor{block body}{bg=gray!20}
	\setbeamercolor{block title}{bg=purple, fg=white}
	\setbeamertemplate{blocks}[shadow=true]
	\begin{block}{\textbf{Non-Example.}}}{\end{block}}
\newenvironment{thm}[1]{%
	\setbeamercolor{block body}{bg=Gold!20}
	\setbeamercolor{block title}{bg=Gold}
	\begin{block}{\textbf{Theorem #1.}}}{\end{block}}


%%%%%%%%%%
%Custom Environment Wrappers
%%%%%%%%%%
\newcommand{\exer}[1]{
	\begin{exercise}
		#1
	\end{exercise}
}
\newcommand{\exam}[1]{
\begin{blockexample}
	#1
\end{blockexample}
}
\newcommand{\nexam}[1]{
\begin{blocknonexample}
	#1
\end{blocknonexample}
}
\newcommand{\enumarabic}[1]{
	\begin{enumerate}[label=\textbf{\arabic*.}]
		#1
	\end{enumerate}
}
\newcommand{\enumalph}[1]{
	\begin{enumerate}[label=(\alph*)]
		#1
	\end{enumerate}
}
\newcommand{\bulletize}[1]{
	\begin{itemize}[label=$\bullet$]
		#1
	\end{itemize}
}
\newcommand{\circtize}[1]{
	\begin{itemize}[label=$\circ$]
		#1
	\end{itemize}
}
\newcommand{\slide}[1]{
	\begin{frame}{\fn}
		#1
	\end{frame}
}
\newcommand{\slidec}[1]{
\begin{frame}[c]{\fn}
	#1
\end{frame}
}
\newcommand{\slidet}[2]{
	\begin{frame}{\fn\ - #1}
		#2
	\end{frame}
}


\newcommand{\startdoc}{
		\title{Math 341: Classical Number Theory}
		\author{Mckenzie West}
		\date{Last Updated: \today}
		\begin{frame}
			\maketitle
		\end{frame}
}

\newcommand{\topics}[2]{
	\begin{frame}[c]{\insertframenumber}
		\begin{block}{\textbf{Last Section.}}
			\begin{itemize}[label=--]
				#1
			\end{itemize}
		\end{block}
		\begin{block}{\textbf{This Section.}}
			\begin{itemize}[label=--]
				#2
			\end{itemize}
		\end{block}
	\end{frame}
}

\begin{document} 
	\startdoc
	\topics{
		\item Order of an integer mod $n$
		\item Primitive Roots
		\item Primitive Roots mod $p$
	}{
		\item Quadratic residues
		\item The Legendre Symbol
		\item Quadratic Reciprocity
	}


\begin{frame}[fragile]
	\frametitle{\fn}
	\begin{defn}
		Let $p$ be an odd prime and $\gcd(a,p)=1$.  If the quadratic congruence $x^2\equiv a\pmod p$ has a solution then $a$ is said to be a \emph{quadratic residue of $p$}.  Otherwise $a$ is called a \emph{quadratic nonresidue of $p$}.
	\end{defn}
	\begin{recall}
		We can use Sage to find quadratic residues:
		\begin{lstlisting}
sage: quadratic_residues(11)
[0, 1, 3, 4, 5, 9]\end{lstlisting}
	\end{recall}
\end{frame}
\slide{
	\begin{exercise}
		How do quadratic residues fit into the discussion of solving congruence equations of the form\[ax^2+bx+c\equiv 0\pmod p?\]\vskip 1in
	\end{exercise}
}
\slide{
	\begin{statementblock}{Euler's Criterion (Theorem 9.1)}
		Let $p$ be an odd prime and $\gcd(a,p)=1$.  Then $a$ is a quadratic residue of $p$ if and only if 	
		\[a^{(p-1)/2}\equiv 1\pmod p.\]
	\end{statementblock}
	\begin{exercise}
		Show that 3 is a quadratic residue of 23 but a nonresidue mod 31.
	\end{exercise}
}
%\slide{
%	\begin{statementblock}{Euler's Criterion (Theorem 9.1)}
%		Let $p$ be an odd prime and $\gcd(a,b)=1$.  Then $a$ is a quadratic residue of $p$ if and only if 	
%		\[a^{(p-1)/2}\equiv 1\pmod p.\]
%	\end{statementblock}
%	\begin{exercise}
%		Let's talk about the proof
%		
%		($\Rightarrow$) Let $a$ be a quadratic residue of $p$.  Let $x_1$ be an integer satisfying $x_1^2\equiv a \pmod p$. Use Fermat's Little Theorem to compute
%		\[a^{(p-1)/2}\equiv\hspace*{2in}\equiv 1\pmod p.\]
%		
%		\vskip .5in
%		Continued on next slide
%	\end{exercise}
%}
%\slide{
%	\begin{exercise}		
%		($\Leftarrow$) Assume $a$ satisfies $a^{(p-1)/2}\equiv 1\pmod p$.  Let $r$ be a primitive root of $p$.  Thus we can write $a\equiv r^k\pmod p$ for some integer $1\leq k\leq p-1$. \fbox{WHY}  Then
%		\[r^{k(p-1)/2}\equiv\hspace*{2in}\equiv 1\pmod p.\]
%		Recall that $\ord_p(r)=\blank{.25in}$, and by Theorem 8.1, $\ord_p(r)$ must then divide $k(p-1)/2$. \fbox{WHY}
%		
%		Thus $k$ must be even \fbox{WHY}.  Write $k=2j$.  Then 
%		\[(\blank{.25in})^2\equiv a\pmod p.\]
%	\end{exercise}
%}
\slide{
	\begin{statementblock}{Corollary}
		Let $p$ be an odd prime and $\gcd(a,p)=1$.  Then $a$ is a quadratic residue or nonresidue of $p$ according to whether
		\[a^{(p-1)/2}\equiv 1\pmod p\quad\text{or}\quad a^{(p-1)/2}\equiv -1\pmod p.\]
	\end{statementblock}
	\begin{exercise}
		Is $2$ a quadratic residue of 13?\vskip .5in
		Is $3$ a quadratic residue of 13?
	\end{exercise}
}
\slide{
	\begin{exercise}
		Prove the following:
		
		\begin{statementblock}{Proposition}
			 If $a$ is a quadratic residue of the odd prime $p$, then $a$ is not a primitive root of $p$.
		\end{statementblock}
	\end{exercise}
}
\slide{
	\begin{exercise}
		Prove that if $p=2^k+1$ is prime then every quadratic nonresidue of $p$ is a primitive root of $p$.
	\end{exercise}
}

\slide{
	\vskip -.25in
	\begin{defn}
		Let $p$ be an odd prime and let $\gcd(a,p)=1$. The \emph{Legendre symbol $(a/p)$} is defined by
		\[\left(a/p\right)=\begin{cases}
			1&\text{if $a$ is a quadratic residue of $p$}\\
			-1&\text{if $a$ is a quadratic nonresidue of $p$}\\
		\end{cases}\]
	\end{defn}
	\begin{nb}
		You may also see this in the following forms
		\[(a/p)=\left(\frac{a}{p}\right)=(a|p).\]
	\end{nb}
	\begin{exercise}
		Compute
		\enumalph{
			\item $(1/5)$
			\item $(9/23)$
			\item $(-23/59)$
		}
	\end{exercise}
}
\slide{
	\begin{exercise}
		What is $(1/p)$?
	\end{exercise}
}
\slide{
	\begin{exercise}
		Consider $(-1/p)$ for $p=3,5,7,11,13$. Conjecture a rule
		
		$(-1/p)=1$ if and only if \underline{\hspace*{1in}}.
		
		$$\begin{array}{r|r}
			p&\left(\frac{-1}{p}\right)\\\hline
			3&-1\ \mbox{}\\
			5&1\ \mbox{}\\ 
			7&-1\ \mbox{}\\
			11&-1\ \mbox{}\\
			13&1\ \mbox{}
		\end{array}$$
	\end{exercise}
}
\slide{
	\begin{exercise}
		If $a\equiv b\pmod p$, how do $(a/p)$ and $(b/p)$ compare?
	\end{exercise}
}
\slide{
	\begin{exercise}
		Let $a$ be an integer with $p\nmid a$.  What is $(a^2/p)$?
	\end{exercise}
}
\slide{
	\begin{exercise}
		Go back to the Corollary to Euler's Criterion to answer this question.
		
		How do $(a/p)$ and $a^{(p-1)/2}$ compare?
	\end{exercise}
}
\slide{
	\begin{exercise}
		How do you propose $(a/b)$, $(b/p)$ and $(ab/p)$ compare?
		
		Here are some values from which to draw conclusions.
		\enumalph{
			\item $(2/7)=1$, $(3/7)=-1$, $(6/7)=-1$
			\item $(2/7)=1$, $(4/7)=1$, $(8/7)=1$
			\item $(3/7)=-1$, $(6/7)=-1$, $(18/7)=1$
		}
	\end{exercise}
}
\slide{
	\begin{exercise}
		Use the fact that
		\[(a/p) \equiv (a)^{(p-1)/2}\pmod p\]
		to show that 
		\[(ab/p) = (a/p)(b/p)\]
	\end{exercise}
}
\slide{
	\begin{statementblock}{Theorem 9.2}
		Let $p$ be an odd prime and let $a$ and $b$ be integers that are relatively prime to $p$. Then the Legendre symbol has the following properties
		\enumalph{
			\item If $a\equiv b\pmod p$, then $(a/p)=(b/p)$.
			\item $(a^2/p)=1$.
			\item $(a/p)\equiv a^{(p-1)/2}\pmod p$.
			\item $(ab/p)=(a/p)(b/p)$.
			\item $(1/p)=1$ and $(-1/p)=(-1)^{(p-1)/2}$.
		}
	\end{statementblock}
	\begin{statementblock}{Corollary}
		If $p$ is an odd prime, then 	
		\[(-1/p)=\begin{cases}
			1&\text{if }p\equiv 1\pmod 4\\
			-1&\text{if }p\equiv 3\pmod 4\\
		\end{cases}\]
	\end{statementblock}
}
\slide{
	\begin{exa}
		This theorem is actually quite useful when computing $(a/p)$.
		
		Let's compute $(-112/23)$.
		
		\bulletize{
			\item By part (d): $(-112/23)=(-1/23)(112/23)$
			\item By part (e): $(-112/23)=-(112/23)$
			\item By part (a): $(-112/23)=-(20/23)$
			\item By part (d): $(-112/23)=-(4/23)(5/23)$
			\item By part (b): $(-112/23)=-1\cdot(5/23)$
		}
	\end{exa}
	\begin{exercise}
		What is $(5/23)$?  What is $(-112/23)$?
	\end{exercise}
}
\slide{
	\begin{statementblock}{Theorem 9.4}
		If $p$ is an odd prime, then
		\[\sum_{a=1}^{p-1} (a/p)=0.\]
		Hence, there are precisely $(p-1)/2$ quadratic residues and $(p-1)/2$ quadratic nonresidues, $1\leq a\leq p-1$, of $p$.
	\end{statementblock}
	\begin{exercise}
		Discuss this Theorem. Why does the first sentence imply the second?
		
		No, it's not as obvious as the theorem makes it sound.  But think for a bit what the summation is and what the terms are.
	\end{exercise}
}
\slide{
	\begin{statementblock}{Corollary}
		The quadratic residues of an odd prime $p$ are congruent modulo $p$ to the even powers of a primitive root $r$ of $p$; the quadratic nonresidues are congruent to the odd powers of $r$.
	\end{statementblock}
	\begin{exercise}
		Recall that 2 is a primitive root of 11.  Use this Corollary to classify all of the integers $1\leq a\leq 10$ as quadratic (non)residues of 11.
	\end{exercise}
}
\slide{
	\begin{statementblock}{Gauss's Lemma (Theorem 9.5)}
		Let $p$ be an odd prime and $\gcd(a,p)=1$.  Let
		\[S=\left\{a,2a,3a,\dots,\frac{p-1}{2}\cdot a\right\}.\]
		Let $n$ denote the number of integers in $S$ whose remainder when divided by $p$ is larger than $p/2$.  Then
		\[(a/p)=(-1)^n.\]
	\end{statementblock}
}
\slide{
	\begin{exercise}
		I find this Theorem to be very hard to understand as is.  So let's do an example. Let $p=13$ and $a=5$.
		\enumalph{
			\item What is $S$?\vskip .5in
			\item What are the remainders when dividing the integers in $S$ by $p$?
			\vskip .5in
			\item How many of these remainders are below $(p-1)/2$?
			(Find $n$.)\vskip .25in
			\item What is $(-1)^n$?\vskip .25in
			\item What is $(5/13)$?
		}
	\end{exercise}
}

\slide{
	Below is the table of $(p/q)$ values for $3\leq p,q\leq 31$ prime.
	$$\begin{array}{|c|c|c|c|c|c|c|c|c|c|c|c|}
		\hline
		p&3&5&7&11&13&17&19&23&29&31\\\hline
		3
		& \cellcolor{white!25}0 
		& \cellcolor{red!25}-1 
		& \cellcolor{red!25}-1 
		& \cellcolor{blue!25}1 
		& \cellcolor{blue!25}1 
		& \cellcolor{red!25}-1 
		& \cellcolor{red!25}-1 
		& \cellcolor{blue!25}1 
		& \cellcolor{red!25}-1 
		& \cellcolor{red!25}-1 
		\\\hline
		5
		& \cellcolor{red!25}-1 
		& \cellcolor{white!25}0 
		& \cellcolor{red!25}-1 
		& \cellcolor{blue!25}1 
		& \cellcolor{red!25}-1 
		& \cellcolor{red!25}-1 
		& \cellcolor{blue!25}1 
		& \cellcolor{red!25}-1 
		& \cellcolor{blue!25}1 
		& \cellcolor{blue!25}1 
		\\\hline
		7
		& \cellcolor{blue!25}1 
		& \cellcolor{red!25}-1 
		& \cellcolor{white!25}0 
		& \cellcolor{red!25}-1 
		& \cellcolor{red!25}-1 
		& \cellcolor{red!25}-1 
		& \cellcolor{blue!25}1 
		& \cellcolor{red!25}-1 
		& \cellcolor{blue!25}1 
		& \cellcolor{blue!25}1 
		\\\hline
		11
		& \cellcolor{red!25}-1 
		& \cellcolor{blue!25}1 
		& \cellcolor{blue!25}1 
		& \cellcolor{white!25}0 
		& \cellcolor{red!25}-1 
		& \cellcolor{red!25}-1 
		& \cellcolor{blue!25}1 
		& \cellcolor{red!25}-1 
		& \cellcolor{red!25}-1 
		& \cellcolor{red!25}-1 
		\\\hline
		13
		& \cellcolor{blue!25}1 
		& \cellcolor{red!25}-1 
		& \cellcolor{red!25}-1 
		& \cellcolor{red!25}-1 
		& \cellcolor{white!25}0 
		& \cellcolor{blue!25}1 
		& \cellcolor{red!25}-1 
		& \cellcolor{blue!25}1 
		& \cellcolor{blue!25}1 
		& \cellcolor{red!25}-1 
		\\\hline
		17
		& \cellcolor{red!25}-1 
		& \cellcolor{red!25}-1 
		& \cellcolor{red!25}-1 
		& \cellcolor{red!25}-1 
		& \cellcolor{blue!25}1 
		& \cellcolor{white!25}0 
		& \cellcolor{blue!25}1 
		& \cellcolor{red!25}-1 
		& \cellcolor{red!25}-1 
		& \cellcolor{red!25}-1 
		\\\hline
		19
		& \cellcolor{blue!25}1 
		& \cellcolor{blue!25}1 
		& \cellcolor{red!25}-1 
		& \cellcolor{red!25}-1 
		& \cellcolor{red!25}-1 
		& \cellcolor{blue!25}1 
		& \cellcolor{white!25}0 
		& \cellcolor{red!25}-1 
		& \cellcolor{red!25}-1 
		& \cellcolor{blue!25}1 
		\\\hline
		23
		& \cellcolor{red!25}-1 
		& \cellcolor{red!25}-1 
		& \cellcolor{blue!25}1 
		& \cellcolor{blue!25}1 
		& \cellcolor{blue!25}1 
		& \cellcolor{red!25}-1 
		& \cellcolor{blue!25}1 
		& \cellcolor{white!25}0 
		& \cellcolor{blue!25}1 
		& \cellcolor{red!25}-1 
		\\\hline
		29
		& \cellcolor{red!25}-1 
		& \cellcolor{blue!25}1 
		& \cellcolor{blue!25}1 
		& \cellcolor{red!25}-1 
		& \cellcolor{blue!25}1 
		& \cellcolor{red!25}-1 
		& \cellcolor{red!25}-1 
		& \cellcolor{blue!25}1 
		& \cellcolor{white!25}0 
		& \cellcolor{red!25}-1 
		\\\hline
		31
		& \cellcolor{blue!25}1 
		& \cellcolor{blue!25}1 
		& \cellcolor{red!25}-1 
		& \cellcolor{blue!25}1 
		& \cellcolor{red!25}-1 
		& \cellcolor{red!25}-1 
		& \cellcolor{red!25}-1 
		& \cellcolor{blue!25}1 
		& \cellcolor{red!25}-1 
		& \cellcolor{white!25}0 
		\\\hline
	\end{array}$$
}
%def get_color(p,q):
%    if kronecker(p,q)==1:
%        return 'blue'
%    elif kronecker(p,q)==-1:
%        return 'red'
%    else:
%        return 'white'
%print("&".join(["p"]+[str(x) for x in range(3,32) if is_prime(x)]+["\\\\\\hline"]))
%for p in [x for x in range(3,32) if is_prime(x)]:
%    print(p)
%    for q in [x for x in range(3,32) if is_prime(x)]:
%        print(f"& \\cellcolor{{{get_color(p,q)}!25}}{kronecker(p,q)} ")
%    print("\\\\\\hline")

\slide{
	\begin{statementblock}{Quadratic Reciprocity}
		If $p$ and $q$ are distinct odd primes, then
		\[(p/q)(q/p) = (-1)^{\frac{p-1}{2}\frac{q-1}{2}}\]
	\end{statementblock}
	\begin{exercise}
		Re-write this Theorem using $p,q\pmod 4$ to classify $(p/q)(q/p)$.
	\end{exercise}
}


\slide{\vskip-.35in
	\begin{block}{\textbf{Why is this true?}}
		\begin{statementblock}{Gauss's Lemma (Theorem 9.5)}
			Let $p$ be an odd prime and $\gcd(a,p)=1$.  Let
			\[S=\left\{a,2a,3a,\dots,\frac{p-1}{2}\cdot a\right\}.\]
			Let $n$ denote the number of integers in $S$ whose remainder when divided by $p$ is larger than $p/2$.  Then
			\[(a/p)=(-1)^n.\]
		\end{statementblock}
		\begin{center}
			%\includegraphics[width=1.25in]{images/quad_rec_proof}
		\end{center}
	\end{block}
}

\slide{
	\begin{statementblock}{Lemma}
		If $p$ and $q$ are distinct odd primes, then
		\[(q/p) = (-1)^{\sum_{k=1}^{(p-1)/2}\lfloor ka/p\rfloor}.\]
	\end{statementblock}
}
\slide{
	\begin{statementblock}{Theorem 9.6}
		If $p$ is an odd prime, then
		\[(2/p)-\begin{cases}
			1&\text{if }p\equiv 1\pmod 8\text{ or }p\equiv7\pmod8\\
			-1&\text{if }p\equiv 3\pmod 8\text{ or }p\equiv5\pmod8\\
		\end{cases}\]
	\end{statementblock}
%	\begin{proof}
%		Here we notice
%		\[S = \{1\cdot 2,2\cdot2,3\cdot2,\dots,\frac{p-1}{2}\cdot 2\}.\]
%		Then
%		\[n=\frac{p-1}{2}-\left\lfloor\frac{p}{4}\right\rfloor.\]
%		\begin{exercise}
%			Compute $\frac{p-1}{2}-\left\lfloor \frac{p}{4}\right\rfloor$ for $p=8k+j$ for $j$ in $\{1,3,5,7\}$.
%		\end{exercise}
%	\end{proof}
}
\slide{
	\begin{statementblock}{Theorem 9.7}
		If $p$ and $2p+1$ are both odd primes, then the integer $(-1)^{(p-1)/2}\cdot 2$ is a primitive root of $2p+1$.
	\end{statementblock}
}
\slide{
	\begin{exercise}
		Compute $(53/59)$.\vskip .75in\mbox{}
	\end{exercise}
	\begin{exercise}
		Compute $(-23/127)$.
	\end{exercise}
}
\slide{
	\begin{statementblock}{Theorem 9.10}
		If $p\neq 3$ is an odd prime, then
		\[(3/p)=\begin{cases}
			1&\text{if }p\equiv \pm 1\pmod{12}\\
			-1&\text{if }p\equiv \pm 5\pmod{12}
		\end{cases}\]
	\end{statementblock}
}
%\slide{
%	\begin{exercise}
%		Solve the quadratic congruence
%		\[x^2\equiv 196\pmod{1357}\]
%	\end{exercise}
%}
\end{document}

