\documentclass[t]{beamer}

\subtitle{Order of Integers and Primitive Roots}

\usepackage{amsthm,amsmath,amsfonts,hyperref,graphicx,color,multicol,soul}
\usepackage{enumitem,tikz,tikz-cd,setspace,mathtools}

%%%%%%%%%%
%Beamer Template Customization
%%%%%%%%%%
\setbeamertemplate{navigation symbols}{}
\setbeamertemplate{theorems}[ams style]
\setbeamertemplate{blocks}[rounded]

\definecolor{Blu}{RGB}{43,62,133} % UWEC Blue
\setbeamercolor{structure}{fg=Blu} % Titles

%Unnumbered footnotes:
\newcommand{\blfootnote}[1]{%
	\begingroup
	\renewcommand\thefootnote{}\footnote{#1}%
	\addtocounter{footnote}{-1}%
	\endgroup
}

%%%%%%%%%%
%TikZ Stuff
%%%%%%%%%%
\usetikzlibrary{arrows}
\usetikzlibrary{shapes.geometric}
\tikzset{
	smaller/.style={
		draw,
		regular polygon,
		regular polygon sides=3,
		fill=white,
		node distance=2cm,
		minimum height=1in,
		line width = 2pt
	}
}
\tikzset{
	smsquare/.style={
		draw,
		regular polygon,
		regular polygon sides=4,
		fill=white,
		node distance=2cm,
		minimum height=1in,
		line width = 2pt
	}
}


%%%%%%%%%%
%Custom Commands
%%%%%%%%%%

\newcommand{\C}{\mathbb{C}}
\newcommand{\quats}{\mathbb{H}}
\newcommand{\N}{\mathbb{N}}
\newcommand{\Q}{\mathbb{Q}}
\newcommand{\R}{\mathbb{R}}
\newcommand{\Z}{\mathbb{Z}}

\newcommand{\ds}{\displaystyle}

\newcommand{\fn}{\insertframenumber}

\newcommand{\id}{\operatorname{id}}
\newcommand{\im}{\operatorname{im}}
\newcommand{\Aut}{\operatorname{Aut}}
\newcommand{\Inn}{\operatorname{Inn}}

\newcommand{\blank}[1]{\underline{\hspace*{#1}}}

\newcommand{\abar}{\overline{a}}
\newcommand{\bbar}{\overline{b}}
\newcommand{\cbar}{\overline{c}}

\newcommand{\nml}{\unlhd}

%%%%%%%%%%
%Custom Theorem Environments
%%%%%%%%%%
\theoremstyle{definition}
\newtheorem{exercise}{Exercise}
\newtheorem{question}[exercise]{Question}
\newtheorem{warmup}{Warm-Up}
\newtheorem*{defn}{Definition}
\newtheorem*{exa}{Example}
\newtheorem*{disc}{Group Discussion}
\newtheorem*{nb}{Note}
\newtheorem*{recall}{Recall}
\renewcommand{\emph}[1]{{\color{blue}\texttt{#1}}}

\definecolor{Gold}{RGB}{237, 172, 26}
%Statement block
\newenvironment{statementblock}[1]{%
	\setbeamercolor{block body}{bg=Gold!20}
	\setbeamercolor{block title}{bg=Gold}
	\begin{block}{\textbf{#1.}}}{\end{block}}
\newenvironment{thm}[1]{%
	\setbeamercolor{block body}{bg=Gold!20}
	\setbeamercolor{block title}{bg=Gold}
	\begin{block}{\textbf{Theorem #1.}}}{\end{block}}


%%%%%%%%%%
%Custom Environment Wrappers
%%%%%%%%%%
\newcommand{\enumarabic}[1]{
	\begin{enumerate}[label=\textbf{\arabic*.}]
		#1
	\end{enumerate}
}
\newcommand{\enumalph}[1]{
	\begin{enumerate}[label=(\alph*)]
		#1
	\end{enumerate}
}
\newcommand{\bulletize}[1]{
	\begin{itemize}[label=$\bullet$]
		#1
	\end{itemize}
}
\newcommand{\circtize}[1]{
	\begin{itemize}[label=$\circ$]
		#1
	\end{itemize}
}
\newcommand{\slide}[1]{
	\begin{frame}{\fn}
		#1
	\end{frame}
}
\newcommand{\slidec}[1]{
\begin{frame}[c]{\fn}
	#1
\end{frame}
}
\newcommand{\slidet}[2]{
	\begin{frame}{\fn\ - #1}
		#2
	\end{frame}
}


\newcommand{\startdoc}{
		\title{Math 425: Abstract Algebra 1}
		\author{Mckenzie West}
		\date{Last Updated: \today}
		\begin{frame}
			\maketitle
		\end{frame}
}

\newcommand{\topics}[2]{
	\begin{frame}{\insertframenumber}
		\begin{block}{\textbf{Last Section.}}
			\begin{itemize}[label=--]
				#1
			\end{itemize}
		\end{block}
		\begin{block}{\textbf{This Section.}}
			\begin{itemize}[label=--]
				#2
			\end{itemize}
		\end{block}
	\end{frame}
}

\begin{document} 
	\startdoc
	\topics{
		\item Euler's theorem
		\item Properties of Euler's phi function
	}{
		\item Order of an integer mod $n$
		\item Primitive Roots
		\item Primitive Roots mod $p$
	}


\slide{
	\begin{recall}
		\textbf{Euler's Theorem} If $\gcd(a,n)=1$, then $a^{\phi(n)}\equiv 1\pmod n$.
		
		\begin{exa}
			This does not have to be the smallest such power. 
			For example consider the powers of 3 modulo 11:
			\[3^1\equiv 3,\quad 3^2\equiv 9,\quad3^3\equiv 5,\quad3^4\equiv 4,\quad3^5\equiv 1\pmod{11}\]
			\[3^6\equiv 3,\quad 3^7\equiv 9,\quad3^8\equiv 5,\quad3^9\equiv 4,\quad3^{10}\equiv 1\pmod{11}\]
			Similarly 
			\[10^1\equiv 10,\quad 10^2\equiv 1,\quad10^3\equiv 10,\quad 10^4\equiv 1,\quad10^5\equiv 10\pmod{11}\]
		\end{exa}
	\end{recall}
}
\slide{
	\begin{defn}
		Let $n>1$ and $\gcd(a,n)=1$.  The \emph{order of $a$ modulo $n$} is the smallest positive integer $k$ for which $a^k\equiv 1\pmod n$.
	
		Denote this value by $\ord_n(a)$.
	\end{defn}
	\begin{exa}
		$$\ord_{11}(3)=5\quad\text{and}\quad\ord_{11}(10)=2$$
	\end{exa}
}
\slide{
	\begin{exercise}
		For all other remainders modulo $11$, $a=1,2,4,5,6,7,8,9$, compute $\ord_{11}(a)$.
		$$\begin{array}{|c|c|}
		\hline
		\mathbf{a}&\mathbf{\ord_{11}(a)}\\\hline
		1&\\\hline
		2&\\\hline
		3&5\\\hline
		4&\\\hline
		5&\\\hline
		\end{array}
		\quad\begin{array}{|c|c|}
		\hline
		\mathbf{a}&\mathbf{\ord_{11}(a)}\\\hline
		6&\\\hline
		7&\\\hline
		8&\\\hline
		9&\\\hline
		10&2\\\hline
		\end{array}$$
	\end{exercise}
	\begin{question}
		Do you notice any patterns? Which of these do you think are specific to $n=11$ and what might be true for more values of $n$?
	\end{question}
}
\slide{
	\begin{exercise}
		Below are the powers of $2$ modulo $11$.
		$$\begin{array}{|c|c|c|}
		\hline
		\mathbf{k}&\mathbf{2^k\ (\text{mod}\ 11)}&\mathbf{\ord_{11}(2^k)}\\\hline
		1&2&10\\\hline
		2&4&5\\\hline
		3&8&10\\\hline
		4&5&5\\\hline
		5&10&2\\\hline
		6&9&5\\\hline
		7&7&10\\\hline
		8&3&5\\\hline
		9&6&10\\\hline
		10&1&1\\\hline
		\end{array}$$
		
		Do you notice any patterns here? Which of these do you think are specific to $n=11$ and what might be true for more values of $n$?
	\end{exercise}
}
\slide{
	\begin{statementblock}{Theorem 8.1}
		Let $a$ and $n$ be positive integers with $\gcd(a,n)=1$. Let $k=\ord_n(a)$ be the order of $a$ modulo $n$.  Then $a^j\equiv 1\pmod n$ if and only if $k|j$.  In particular $k|\phi(n)$.
	\end{statementblock}
}
\slide{
	\begin{exercise}
		\enumalph{
			\item According to Theorem 8.1, what are possible values for $\ord_{17}(a)$?\vskip .5in
			\item How can you use this information to more quickly determine $\ord_{17}(a)$?\vskip .5in
			\item Determine $\ord_{17}(a)$ for $a=2,4,10$. (You can use a calculator but also use the above answers to do it quicker.)
		}
	\end{exercise}
}
\slide{
	\begin{statementblock}{Theorem 8.2}
		If the integer $\ord_n(a)=k$, then $a^i\equiv a^j\pmod n$ if and only if $i\equiv j\pmod k$.
	\end{statementblock}
	\begin{exercise}
		Let's explore this theorem using  $a=3$ and $n=11$.
		\[3^1\equiv 3,\quad 3^2\equiv 9,\quad3^3\equiv 5,\quad3^4\equiv 4,\quad3^5\equiv 1\pmod{11}\]
		\[3^6\equiv 3,\quad 3^7\equiv 9,\quad3^8\equiv 5,\quad3^9\equiv 4,\quad3^{10}\equiv 1\pmod{11}\]
	\end{exercise}
}
\slide{
	\begin{statementblock}{Corollary}
		If $a$ has order $k$ modulo $n$, then the integers $a,a^2,\dots,a^k$ are incongruent modulo $n$.
	\end{statementblock}
}
\slide{
	\begin{statementblock}{Theorem 8.3}
		If $\ord_n(a)=k$, then
			\[\ord_n(a^j)=\frac{k}{\gcd(j,k)}.\]
	\end{statementblock}
	\begin{exercise}
		If $\ord_{43}(2)=14$, compute
			\enumalph{\setlength{\itemsep}{1.75em}
				\item $\ord_{43}(4)$
				\item $\ord_{43}(8)$
				\item $\ord_{43}(16)$
				\item $\ord_{43}(128)$
			}
	\end{exercise}
}
\slide{
	\begin{statementblock}{Corollary}
		Let $\ord_n(a)=k$.  Then $\ord_n(a^j)=k$ if and only if $\gcd(j,k)=1$.
	\end{statementblock}
}
\slide{
	\begin{defn}
		We call $a$ a \emph{primitive root} modulo $n$ if $\ord_n(a)=\phi(n)$.
	\end{defn}
	\begin{exercise}
		Modulo 11, one primitive root is $a=2$, verify this.
	\end{exercise}\vfill
	\begin{exercise}
		Use the corollary from the last slide to find all primitive roots modulo 11.
	\end{exercise}\vfill
}
\slide{
	\begin{nb}
		Next time we will prove that if $p$ is prime there will for sure be a primitive root modulo $p$.  
		
		This is not guaranteed if $n$ is not prime.
	\end{nb}
	\begin{exercise}
		\enumalph{
			\item Verify that $2$ is a primitive root modulo $9$.\vskip .5in
			\item Show that there are no primitive roots modulo $8$.
		}
	\end{exercise}
}
\slide{
	\begin{nb}
		Example 8.2 in the textbook gives an example where if $n=2^{2^m}+1$ is prime, then $2$ is not a primitive root modulo $n$.
		
		Therefore it's possible for 2 to not be a primitive root.
	\end{nb}
	\begin{exercise}
		Show that $2$ is not a primitive root modulo 7.
	\end{exercise}
}
\slide{
	\begin{statementblock}{Theorem 8.4}
		Let $\gcd(a,n)=1$ and let $b_1,b_2,\dots,b_{\phi(n)}$ be the positive integers less than or equal to $n$ that are relatively prime to $n$.
		
		If $a$ is a primitive root modulo $n$, then
			\[a^1,a^2,\dots,a^{\phi(n)}\]
		are congruent modulo $n$ to $b_1,b_2,\dots,b_{\phi(n)}$ in some order.
	\end{statementblock}
	\begin{statementblock}{Corollary}
		If $n$ has a primitive root, then it has exactly $\phi(\phi(n))$ of them.
	\end{statementblock}
	\begin{exercise}
		Verify this corollary for $n=11$.
	\end{exercise}
}
\slide{
	\begin{statementblock}{Theorem 8.6}
		If $p$ is a prime number and $d|p-1$, then there are exactly $\phi(d)$ incongruent integers having order $d$ modulo $p$.
	\end{statementblock}
	\begin{proof}[Proof (Sketch with $p=13$ and $d=4$)]
		\vskip 3in
	\end{proof}
}
\slide{
	\begin{statementblock}{Theorem 8.5}
		If $p$ is prime and 
		\[f(x)=a_nx^n+a_{n-1}x^{n-1}+\cdots+a_1x+a_0\quad a_n\not\equiv0\pmod p\]
		is a polynomial of degree $n\geq1$ with integral coefficients, then the congruence
		\[f(x)\equiv0\pmod p\]
		has at most $n$ incongruent solutions modulo $p$.
	\end{statementblock}
}
\slide{
	\begin{statementblock}{Corollary}
		If $p$ is prime and $d|p-1$, then
		\[x^d-1\equiv 0\pmod p\]
		has exactly $d$ incongruent solutions.
	\end{statementblock}
}
\slide{
	\begin{statementblock}{Corollary}
		If $p$ is prime then there are $\phi(p-1)$ incongruent primitive roots modulo $p$.
	\end{statementblock}
}
\slide{
	\begin{nb}
		This proof is not \textit{constructive}.  That is, it doesn't tell use how to find the primitive roots.
		
		Here's a table of the smallest primitive roots mod $p$ for $2\leq p\leq 197$.
		$$
		\begin{array}{|c|c|}
			\hline
			2 & 1 \\\hline
			3 & 2 \\\hline
			5 & 2 \\\hline
			7 & 3 \\\hline
			11 & 2 \\\hline
			13 & 2 \\\hline
			17 & 3 \\\hline
			19 & 2 \\\hline
			23 & 5 \\\hline
		\end{array}
		\quad
		\begin{array}{|c|c|}
			\hline
			29 & 2 \\\hline
			31 & 3 \\\hline
			37 & 2 \\\hline
			41 & 6 \\\hline
			43 & 3 \\\hline
			47 & 5 \\\hline
			53 & 2 \\\hline
			59 & 2 \\\hline
			61 & 2 \\\hline
		\end{array}
		\quad
		\begin{array}{|c|c|}
			\hline
			67 & 2 \\\hline
			71 & 7 \\\hline
			73 & 5 \\\hline
			79 & 3 \\\hline
			83 & 2 \\\hline
			89 & 3 \\\hline
			97 & 5 \\\hline
			101 & 2 \\\hline
			103 & 5 \\\hline
		\end{array}
		\quad
		\begin{array}{|c|c|}
			\hline
			107 & 2 \\\hline
			109 & 6 \\\hline
			113 & 3 \\\hline
			127 & 3 \\\hline
			131 & 2 \\\hline
			137 & 3 \\\hline
			139 & 2 \\\hline
			149 & 2 \\\hline
			151 & 6 \\\hline
		\end{array}
		\quad
		\begin{array}{|c|c|}
			\hline
			157 & 5 \\\hline
			163 & 2 \\\hline
			167 & 5 \\\hline
			173 & 2 \\\hline
			179 & 2 \\\hline
			181 & 2 \\\hline
			191 & 19 \\\hline
			193 & 5 \\\hline
			197 & 2 \\\hline
		\end{array}
		$$
	\end{nb}
}
\slide{
	\begin{statementblock}{Conjecture [Gauss]}
		There are infinitely many primes $p$ having 10 as a primitive root.
	\end{statementblock}
	\begin{statementblock}{Conjecture [Artin]}
		Let $a$ be an integer with $a\neq 1,-1$ or a perfect square. There are infinitely many primes $p$ having $a$ as a primitive root.
	\end{statementblock}
}
\slide{
	\begin{exercise}
		Find all of the primitive roots modulo 31.
	\end{exercise}
}
\end{document}

